\documentclass[a4paper, 10pt, twoside, headings=small]{scrartcl}

%%%%%%%%%%%%%%%%%%%% General %%%%%%%%%%%%%%%%%%%%%%%%%%%%%%%%%%%%%%%%%%%%%%%%%%%

\usepackage[utf8]{inputenc}
\usepackage[T1]{fontenc}
%\usepackage[protrusion=true, expansion]{microtype} 

\usepackage[hidelinks]{hyperref} % Must go after the patches


%%%%%%%%%%%%%%%%%%%% Fonts %%%%%%%%%%%%%%%%%%%%%%%%%%%%%%%%%%%%%%%%%%%%%%%%%%%%%

\usepackage{fontspec}

\usepackage{polyglossia}


\addtokomafont{disposition}{\rmfamily}


\usepackage{textcomp}

\linespread{1.1}  


\usepackage[final]{microtype}
\setmainfont[Ligatures=TeX]{XCharter}


%%%%%%%%%%%%%%%%%%%% Random Packages  %%%%%%%%%%%%%%%%%%%%%%%%%%%%%%%%%%%%%%%

\usepackage{geometry}	
\usepackage{enumerate} %Erweiterung der enumerate-Umgebung
\usepackage{ifthen,calc}
\usepackage{mathrsfs,amssymb} %Zusatzzeichen
\usepackage{wrapfig}
\usepackage[retainorgcmds]{IEEEtrantools} %Besonders geeignet für einen mehrzeilige Formelsatz
\usepackage{theorem} %Theoremlayout
\usepackage{multicol}
\setlength\columnsep{20pt}
\usepackage{csquotes}





%%%%%%%%%%%%%%%%%%%% Page %%%%%%%%%%%%%%%%%%%%%%%%%%%%%%%%%%%%%%%%%%%%%%%%%%%

% twoside
\geometry{left=2.7cm, right=2.3cm, top=2.7cm, bottom=2.2cm}


%%%%%%%%%%%% PDF %%%%%%%%%%%%%%%%%%%%%%%%%%%%%%%%%%%%%%%%%%%%%%%%%%%%%%%%%%%%

\hypersetup{
	hidelinks=true,
	%	linkcolor=black,
	%	filecolor=black,      
	%	urlcolor=black,
	%	citecolor=black,
	%	allcolors=black,
	%	allbordercolors=white,
	%pdfpagemode=FullScreen,
	%	pdftitle={\mytitle},
	%	pdfauthor={\myauthor},
	pdfkeywords={},
	%	pdfcreator={Some fancy PDF-Creator...},
	bookmarksnumbered=true
}



%%%%%%%%%%%% footnotes %%%%%%%%%%%%%%%%%%%%%%%%%%%%%%%%%%%%%%%%%%%%%%%%%%%%%%

\usepackage[flushmargin, hang]{footmisc} % flush footnote mark to left margin
\usepackage{regexpatch}
\makeatletter
% 1. remove all redefinitions about footnotes done by \maketitle
%    and add \titletrue
\regexpatchcmd{\maketitle}
{\c{def}\c{@makefnmark}.*\c{if@twocolumn}}
{\c{titletrue}\c{if@twocolumn}}
{}{}
% 2. define a conditional
\newif\iftitle
%% 3. redefine \@makefnmark to print nothing when \titletrue
%\xpretocmd{\@makefnmark}{\iftitle\else}{}{}
%\xapptocmd{\@makefnmark}{\fi}{}{}
% 4. ensure \@makefntext has \titlefalse
%    that's justified by the fact that \@makefnmark
%    in \@makefntext is set in a box
\xpretocmd{\@makefntext}{\titlefalse}{}{}

\makeatother

\renewcommand{\footnotemargin}{1em}
%\addtolength{\footnotesep}{5mm}
\skip\footins=2\bigskipamount     % Determine the space above the rule
\renewcommand*{\footnoterule}{%
	\kern-3pt%
	\hrule width 1in%
	\kern 2.6pt%
	\vspace{\smallskipamount}       % The additional space below the rule
}



%%%%%%%%%%%% captions %%%%%%%%%%%%%%%%%%%%%%%%%%%%%%%%%%%%%%%%%%%%%%%%%%%%%%%

\usepackage[textfont={small},labelfont={small, bf}]{caption}
\DeclareCaptionFont{black}{ \color{white} }
\DeclareCaptionFormat{listing}{
	\colorbox[cmyk]{0.43, 0.35, 0.35,0.01 }{
		\parbox{\textwidth}{\hspace{15pt}#1#2#3}
	}
}
\captionsetup{format=plain, singlelinecheck=true}
\captionsetup[lstlisting]{labelfont={small, bf}, textfont={small}}


%%%%%%%%%%%% Title %%%%%%%%%%%%%%%%%%%%%%%%%%%%%%%%%%%%%%%%%%%%%%%%%%%%%%%%%%

\usepackage{titling}
\setlength{\droptitle}{-5em}
\pretitle{\begin{center}\LARGE\bfseries}
	\posttitle{\par\end{center}}
\preauthor{\begin{center}}
	\postauthor{\par\end{center}}
\predate{\begin{center}}
	\postdate{\par\end{center}}


%%%%%%%%%%%% footer / header %%%%%%%%%%%%%%%%%%%%%%%%%%%%%%%%%%%%%%%%%%%%%%%%

\usepackage{fancyhdr}

% twoside with subsection
\fancyhf{}
\fancyhead[RE]{\small\nouppercase\leftmark}
\fancyhead[LO]{\small\rightmark}
\fancyhead[LE,RO]{\thepage}
\renewcommand{\headrulewidth}{0pt}


% Does not really work...
%\setotherlanguage{greek}
%\setotherlanguage{hebrew}
%\newfontfamily\greekfont[]{Linux Libertine O}
%\newfontfamily\hebrewfont[]{Linux Libertine O}

\newcommand{\bm}{\vectorbold*} % using physics package

\newcommand{\matlab}{\textsc{Matlab}\textsuperscript{\tiny{\textregistered}}}



\setmainlanguage[]{english}

\title{03 The Roots of Restlessness}

\author{Ellen G.\ White}

\date{2021/03 Rest in Christ}

\begin{document}

\maketitle

\thispagestyle{empty}

\pagestyle{fancy}

\begin{multicols}{2}

\section*{Saturday – The Roots of Restlessness}

It is a wicked pride that delights in the vanity of one’s own works, that boasts of one’s excellent qualities, seeking to make others seem inferior in order to exalt self, claiming more glory than the cold heart is willing to give to God. The disciples of Christ will heed the Master’s instruction. He has bidden us love one another even as He has loved us. Religion is founded upon love to God, which also leads us to love one another. It is full of gratitude, humility, long-suffering. It is self-sacrificing, forbearing, merciful, and forgiving. It sanctifies the whole life and extends its influence over others.

Those who love God cannot harbor hatred or envy. When the heavenly principle of eternal love fills the heart, it will flow out to others, not merely because favors are received of them, but because love is the principle of action and modifies the character, governs the impulses, controls the passions, subdues enmity, and elevates and ennobles the affections. This love is not contracted so as merely to include “me and mine,” but is as broad as the world and as high as heaven, and is in harmony with that of the angel workers. This love cherished in the soul sweetens the entire life and sheds a refining influence on all around. … This love is the spirit of God. It is the heavenly adorning that gives true nobility and dignity to the soul and assimilates our lives to that of the Master.—Testimonies for the Church, vol. 4, p. 223.

Seek to be an evergreen tree. Wear the ornament of a meek and quiet spirit, which is in the sight of God of great price. Cherish the grace of love, joy, peace, long-suffering, gentleness, goodness, faith, meekness, temperance. This is the fruit of the Christian tree. Planted by the rivers of water, it always brings forth its fruit in due season.—Ellen G. White Comments, in The SDA Bible Commentary, vol. 3, p. 1142.

Before you are two ways—the broad road of self-indulgence and the narrow path of self-sacrifice. Into the broad road you can take selfishness, pride, love of the world; but those who walk in the narrow way must lay aside every weight, and the sin which doth so easily beset. Which road have you chosen—the road which leads to everlasting death, or the road which leads to glory and immortality?

There never was a more solemn time in the history of the world than the time in which we are now living. Our eternal interests are at stake, and we should arouse to the importance of making our calling and election sure. We dare not risk our eternal interests on mere probabilities. We must be in earnest. What we are, what we are doing, what is to be our course of action in the future, are all questions of untold moment, and we cannot afford to be listless, indifferent, unconcerned. It becomes each one of us to inquire, “What is eternity to me?” Are our feet in the path that leads to heaven, or in the broad road that leads to perdition?—Our High Calling, p. 8.

\section*{Sunday – Jesus Brings Division}

The Saviour bade His disciples not to hope that the world’s enmity to the gospel would be overcome, and that after a time its opposition would cease. He said, “I came not to send peace, but a sword.” This creating of strife is not the effect of the gospel, but the result of opposition to it. Of all persecution the hardest to bear is variance in the home, the estrangement of dearest earthly friends. But Jesus declares, “He that loveth father or mother more than Me is not worthy of Me: and he that loveth son or daughter more than Me is not worthy of Me. And he that taketh not his cross, and followeth after Me, is not worthy of Me.”

The mission of Christ’s servants is a high honor, and a sacred trust. “He that receiveth you,” He says, “receiveth Me, and he that receiveth Me receiveth Him that sent Me.” No act of kindness shown to them in His name will fail to be recognized and rewarded. And in the same tender recognition He includes the feeblest and lowliest of the family of God: “Whosoever shall give to drink unto one of these little ones”—those who are as children in their faith and their knowledge of Christ—“a cup of cold water only in the name of a disciple, verily I say unto you, he shall in nowise lose his reward.”—The Desire of Ages, p. 357.

Shortly before His crucifixion Christ bequeathed to His disciples a legacy of peace. This peace is not the peace that comes through conformity with the world. It is an internal rather than an external peace. Without will be wars and fightings, through the opposition of avowed enemies, and the coldness and suspicion of those who claim to be friends. The peace of Christ is not to banish division, but it is to remain amid strife and division.

Though he bore the title of Prince of Peace, Christ said of Himself, “Think not that I am come to send peace on earth: I came not to send peace, but a sword.” Matthew 10:34. The Prince of Peace, He was yet the cause of division.

Families must be divided in order that all who call upon the name of the Lord may be saved. All who refuse His infinite love will find Christianity a sword, a disturber of their peace.—Our High Calling, p. 328.

In the matchless gift of His Son, God has encircled the whole world with an atmosphere of grace as real as the air which circulates around the globe. All who choose to breathe this life-giving atmosphere will live and grow up to the stature of men and women in Christ Jesus.

As the flower turns to the sun, that the bright beams may aid in perfecting its beauty and symmetry, so should we turn to the Sun of Righteousness, that heaven’s light may shine upon us, that our character may be developed into the likeness of Christ.—Steps to Christ, p. 68.

\section*{Monday – Selfishness}

By the parable of the foolish rich man, Christ showed the folly of those who make the world their all. This man had received everything from God. The sun had been permitted to shine upon his land; for its rays fall on the just and on the unjust. The showers of heaven descend on the evil and on the good. The Lord had caused vegetation to flourish, and the fields to bring forth abundantly. The rich man was in perplexity as to what he should do with his produce. His barns were full to overflowing, and he had no place to put the surplus of his harvest. He did not think of God, from whom all his mercies had come. He did not realize that God had made him a steward of His goods that he might help the needy. He had a blessed opportunity of being God’s almoner, but he thought only of ministering to his own comfort. …

In living for self he has rejected that divine love which would have flowed out in mercy to his fellow men. Thus he has rejected life. For God is love, and love is life. This man has chosen the earthly rather than the spiritual, and with the earthly he must pass away. …

“So is he that layeth up treasure for himself, and is not rich toward God.”—Christ’s Object Lessons, pp. 256, 258.

Among those to whom bitter disappointment will come at the day of final reckoning will be some who have been outwardly religious, and who apparently have lived Christian lives. But self is woven into all they do. They pride themselves on their morality, their influence, their ability to stand in a higher position than others, [and] their knowledge of the truth, for they think that these will win for them the commendation of Christ. “Lord,” they plead … “Have we not prophesied in thy name? and in thy name have cast out devils? and in thy name done many wonderful works?” (Matthew 7:22).

But Christ says, “I tell you, I know you not whence ye are; depart from me.” “Not everyone that saith unto me, Lord, Lord, shall enter into the kingdom of heaven; but he that doeth the will of my Father which is in heaven” (Matthew 7:21).—Selected Messages, book 1, pp. 81, 82.

Paul … was convinced that if men could be led to consider the amazing sacrifice made by the Majesty of heaven, selfishness would be banished from their hearts. … He directs the mind first to the position which Christ occupied in heaven in the bosom of His Father; he reveals Him afterward as laying aside His glory, voluntarily subjecting Himself to the humbling conditions of man’s life, assuming the responsibilities of a servant, and becoming obedient unto death, and that the most ignominious and revolting, the most agonizing—the death of the cross. Can we contemplate this wonderful manifestation of the love of God without gratitude and love, and a deep sense of the fact that we are not our own? Such a Master should not be served from grudging, selfish motives.—The Ministry of Healing, p. 501.

\section*{Tuesday – Ambition}

How often our service to Christ, our communion with one another, is marred by the secret desire to exalt self! How ready the thought of self-gratulation, and the longing for human approval! It is the love of self, the desire for an easier way than God has appointed that leads to the substitution of human theories and traditions for the divine precepts. To His own disciples the warning words of Christ are spoken, “Take heed and beware of the leaven of the Pharisees.”

The religion of Christ is sincerity itself. Zeal for God’s glory is the motive implanted by the Holy Spirit; and only the effectual working of the Spirit can implant this motive. Only the power of God can banish self-seeking and hypocrisy. This change is the sign of His working. When the faith we accept destroys selfishness and pretense, when it leads us to seek God’s glory and not our own, we may know that it is of the right order. “Father, glorify Thy name” (John 12:28), was the keynote of Christ’s life, and if we follow Him, this will be the keynote of our life.—The Desire of Ages, p. 409.

The disciples made no move toward serving one another. Jesus waited for a time to see what they would do. Then He, the divine Teacher, rose from the table. Laying aside the outer garment that would have impeded His movements, He took a towel, and girded Himself. With surprised interest the disciples looked on, and in silence waited to see what was to follow. “After that He poureth water into a basin, and began to wash the disciples’ feet, and to wipe them with the towel wherewith He was girded.” This action opened the eyes of the disciples. Bitter shame and humiliation filled their hearts. They understood the unspoken rebuke, and saw themselves in altogether a new light.

So Christ expressed His love for His disciples. Their selfish spirit filled Him with sorrow, but He entered into no controversy with them regarding their difficulty. Instead He gave them an example they would never forget. His love for them was not easily disturbed or quenched. … [Though] He had a full consciousness of His divinity … He had laid aside His royal crown and kingly robes, and had taken the form of a servant. One of the last acts of His life on earth was to gird Himself as a servant, and perform a servant’s part.—The Desire of Ages, p. 644.

Jesus has said, “I, if I be lifted up from the earth, will draw all men unto Me.” John 12:32. Christ must be revealed to the sinner as the Saviour dying for the sins of the world; and as we behold the Lamb of God upon the cross of Calvary, the mystery of redemption begins to unfold to our minds and the goodness of God leads us to repentance. In dying for sinners, Christ manifested a love that is incomprehensible; and as the sinner beholds this love, it softens the heart, impresses the mind, and inspires contrition in the soul.—Steps to Christ, p. 26.

\section*{Wednesday – Hypocrisy}

The hypocrisy of the Pharisees was the product of self-seeking. The glorification of themselves was the object of their lives. It was this that led them to pervert and misapply the Scriptures, and blinded them to the purpose of Christ’s mission. This subtle evil even the disciples of Christ were in danger of cherishing. Those who classed themselves with the followers of Jesus, but who had not left all in order to become His disciples, were influenced in a great degree by the reasoning of the Pharisees. They were often vacillating between faith and unbelief, and they did not discern the treasures of wisdom hidden in Christ. Even the disciples, though outwardly they had left all for Jesus’ sake, had not in heart ceased to seek great things for themselves. … It was this that came between them and Christ, making them so little in sympathy with His mission of self-sacrifice, so slow to comprehend the mystery of redemption.—The Desire of Ages, p. 409.

Our Saviour presented before the people of that time the character of their sins. His plain words aroused the consciences of the hearers, but Satan’s counter-working agencies were seeking for a place for their theories, to attract minds from the plainly spoken truth. As the great Teacher would speak impressive truth, the scribes and Pharisees, under pretense of being interested, would assemble around the disciples and Christ, and divert the minds of the disciples by starting questions to create controversy. They pretended that they wanted to know the truth. Christ was interrupted on this occasion as on many similar occasions. And He wished His disciples to listen to the words He had to say, and not allow anything to attract and hold their attention. Therefore He warned them, “Beware of the leaven of the Pharisees, which is hypocrisy.” They feigned a desire to get as close as possible to the inner circle. As the Lord Jesus presented truth in contrast to error, the Pharisees pretended to be desirous of understanding the truth, yet they were trying to lead His mind in other channels.

Hypocrisy is like leaven or yeast. Leaven may be hidden in the flour, and its presence is not known until it produces its effect. By insinuating itself, it soon pervades the whole mass. Hypocrisy works secretly, and if indulged, it will fill the mind with pride and vanity. There are deceptions practiced now similar to those practiced by the Pharisees. When the Saviour gave this caution, it was to warn all who believe in Him to be on guard. Watch against imbibing this spirit, and becoming like those who tried to ensnare the Saviour.—Ellen G. White Comments, in The SDA Bible Commentary, vol. 5, p. 1121.

\section*{Thursday – Uprooting Restlessness}

To live for self is to perish. Covetousness, the desire of benefit for self’s sake, cuts the soul off from life. It is the spirit of Satan to get, to draw to self. It is the spirit of Christ to give, to sacrifice self for the good of others. “And this is the record, that God hath given to us eternal life, and this life is in His Son. He that hath the Son hath life; and he that hath not the Son of God hath not life.” 1 John 5:11, 12.

Wherefore He says, “Take heed, and beware of covetousness; for a man’s life consisteth not in the abundance of the things which he possesseth.”—Christ’s Object Lessons, p. 259.

Praise the Lord, oh, my soul! He says He has gone to prepare mansions for me: “Let not your heart be troubled: ye believe in God, believe also in me. In my Father’s house are many mansions: if it were not so, I would have told you. I go to prepare a place for you. And if I go and prepare a place for you, I will come again, and receive you unto myself; that where I am, there ye may be also” (John 14:1-3).

Thank God! It is these mansions that I am looking to. It is not the earthly mansions here, for they are to be shaken down by the mighty earthquake erelong; but it is those heavenly mansions that Christ has gone to prepare for the faithful. We have no home here. We are only pilgrims and strangers here, passing to a better country, even an heavenly. May God help us to win the boon of eternal life.—In Heavenly Places, p. 354.

The humble and broken heart, subdued by genuine repentance, will appreciate something of the love of God and the cost of Calvary; and as a son confesses to a loving father, so will the truly penitent bring all his sins before God. And it is written, “If we confess our sins, He is faithful and just to forgive us our sins, and to cleanse us from all unrighteousness.” 1 John 1:9.

God’s promise is, “Ye shall seek Me, and find Me, when ye shall search for Me with all your heart.” Jeremiah 29:13. …

… When Christ dwells in the heart, the soul will be so filled with His love, with the joy of communion with Him, that it will cleave to Him; and in the contemplation of Him, self will be forgotten. Love to Christ will be the spring of action. Those who feel the constraining love of God, do not ask how little may be given to meet the requirements of God; they do not ask for the lowest standard, but aim at perfect conformity to the will of their Redeemer.—Steps to Christ, pp. 41–44.

\section*{Friday – Further Thought}

\setlength{\parindent}{0pt}This Day With God, “Ye All Are Brethren,” p. 192;

Testimonies for the Church, “Necessity of Harmony,” vol. 4, pp. 225, 226.

\end{multicols}

\end{document}

