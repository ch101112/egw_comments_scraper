\documentclass[a4paper, 10pt, twoside, headings=small]{scrartcl}

%%%%%%%%%%%%%%%%%%%% General %%%%%%%%%%%%%%%%%%%%%%%%%%%%%%%%%%%%%%%%%%%%%%%%%%%

\usepackage[utf8]{inputenc}
\usepackage[T1]{fontenc}
\usepackage[protrusion=true, expansion]{microtype} 

\usepackage[hidelinks]{hyperref} % Must go after the patches


%%%%%%%%%%%%%%%%%%%% Fonts %%%%%%%%%%%%%%%%%%%%%%%%%%%%%%%%%%%%%%%%%%%%%%%%%%%%%


\usepackage[english]{babel}



\usepackage{Charter}
\addtokomafont{disposition}{\rmfamily}


\usepackage{textcomp}

\linespread{1.1}  


%%%%%%%%%%%%%%%%%%%% Random Packages  %%%%%%%%%%%%%%%%%%%%%%%%%%%%%%%%%%%%%%%

\usepackage{geometry}	
\usepackage{enumerate} %Erweiterung der enumerate-Umgebung
\usepackage{ifthen,calc}
\usepackage{mathrsfs,amssymb} %Zusatzzeichen
\usepackage{wrapfig}
\usepackage[retainorgcmds]{IEEEtrantools} %Besonders geeignet für einen mehrzeilige Formelsatz
\usepackage{theorem} %Theoremlayout
\usepackage{multicol}
\setlength\columnsep{20pt}
\usepackage{csquotes}





%%%%%%%%%%%%%%%%%%%% Page %%%%%%%%%%%%%%%%%%%%%%%%%%%%%%%%%%%%%%%%%%%%%%%%%%%

% twoside
\geometry{left=2.7cm, right=2.3cm, top=2.7cm, bottom=2.2cm}


%%%%%%%%%%%% PDF %%%%%%%%%%%%%%%%%%%%%%%%%%%%%%%%%%%%%%%%%%%%%%%%%%%%%%%%%%%%

\hypersetup{
	hidelinks=true,
	%	linkcolor=black,
	%	filecolor=black,      
	%	urlcolor=black,
	%	citecolor=black,
	%	allcolors=black,
	%	allbordercolors=white,
	%pdfpagemode=FullScreen,
	%	pdftitle={\mytitle},
	%	pdfauthor={\myauthor},
	pdfkeywords={},
	%	pdfcreator={Some fancy PDF-Creator...},
	bookmarksnumbered=true
}



%%%%%%%%%%%% footnotes %%%%%%%%%%%%%%%%%%%%%%%%%%%%%%%%%%%%%%%%%%%%%%%%%%%%%%

\usepackage[flushmargin, hang]{footmisc} % flush footnote mark to left margin
\usepackage{regexpatch}
\makeatletter
% 1. remove all redefinitions about footnotes done by \maketitle
%    and add \titletrue
\regexpatchcmd{\maketitle}
{\c{def}\c{@makefnmark}.*\c{if@twocolumn}}
{\c{titletrue}\c{if@twocolumn}}
{}{}
% 2. define a conditional
\newif\iftitle
%% 3. redefine \@makefnmark to print nothing when \titletrue
%\xpretocmd{\@makefnmark}{\iftitle\else}{}{}
%\xapptocmd{\@makefnmark}{\fi}{}{}
% 4. ensure \@makefntext has \titlefalse
%    that's justified by the fact that \@makefnmark
%    in \@makefntext is set in a box
\xpretocmd{\@makefntext}{\titlefalse}{}{}

\makeatother

\renewcommand{\footnotemargin}{1em}
%\addtolength{\footnotesep}{5mm}
\skip\footins=2\bigskipamount     % Determine the space above the rule
\renewcommand*{\footnoterule}{%
	\kern-3pt%
	\hrule width 1in%
	\kern 2.6pt%
	\vspace{\smallskipamount}       % The additional space below the rule
}



%%%%%%%%%%%% captions %%%%%%%%%%%%%%%%%%%%%%%%%%%%%%%%%%%%%%%%%%%%%%%%%%%%%%%

\usepackage[textfont={small},labelfont={small, bf}]{caption}
\DeclareCaptionFont{black}{ \color{white} }
\DeclareCaptionFormat{listing}{
	\colorbox[cmyk]{0.43, 0.35, 0.35,0.01 }{
		\parbox{\textwidth}{\hspace{15pt}#1#2#3}
	}
}
\captionsetup{format=plain, singlelinecheck=true}
\captionsetup[lstlisting]{labelfont={small, bf}, textfont={small}}


%%%%%%%%%%%% Title %%%%%%%%%%%%%%%%%%%%%%%%%%%%%%%%%%%%%%%%%%%%%%%%%%%%%%%%%%

\usepackage{titling}
\setlength{\droptitle}{-5em}
\pretitle{\begin{center}\LARGE\normalfont\scshape}
	\posttitle{\par\end{center}}
\preauthor{\begin{center}}
	\postauthor{\par\end{center}}
\predate{\begin{center}}
	\postdate{\par\end{center}}


%%%%%%%%%%%% footer / header %%%%%%%%%%%%%%%%%%%%%%%%%%%%%%%%%%%%%%%%%%%%%%%%

\usepackage{fancyhdr}

% twoside with subsection
\fancyhf{}
\fancyhead[RE]{\small\nouppercase\leftmark}
\fancyhead[LO]{\small\rightmark}
\fancyhead[LE,RO]{\thepage}
\renewcommand{\headrulewidth}{0pt}


% Does not really work...
%\setotherlanguage{greek}
%\setotherlanguage{hebrew}
%\newfontfamily\greekfont[]{Linux Libertine O}
%\newfontfamily\hebrewfont[]{Linux Libertine O}

\newcommand{\bm}{\vectorbold*} % using physics package

\newcommand{\matlab}{\textsc{Matlab}\textsuperscript{\tiny{\textregistered}}}



\setmainlanguage[]{english}

\title{06 Finding Rest in Family Ties}

\author{Ellen G.\ White}

\date{2021/03 Rest in Christ}

\begin{document}

\maketitle

\thispagestyle{empty}

\pagestyle{fancy}

\begin{multicols}{2}

\section*{Saturday – Finding Rest in Family Ties}

Man is not what he might be and what it is God’s will that he should be. The strong power of Satan upon the human race keeps them upon a low level; but this need not be so, else Enoch could not have become so elevated and ennobled as to walk with God. Man need not cease to grow intellectually and spiritually during his life-time. But the minds of many are so occupied with themselves and their own selfish interests as to leave no room for higher and nobler thoughts. … Few realize that they have a constant work before them to develop forbearance, sympathy, charity, conscientiousness, and fidelity. …

Men cannot love God supremely and their neighbor as themselves, and be as cold as icebergs. Not only do they rob God of the love due Him, but they rob their neighbor as well. Love is a plant of heavenly growth, and it must be fostered and nourished. Affectionate hearts, truthful, loving words, will make happy families and exert an elevating influence upon all who come within the sphere of their influence.—Testimonies for the Church, vol. 4, pp. 547, 548.

When the truth of God is an abiding principle in the heart, it will be like a living spring. Attempts may be made to repress it, but it will gush forth in another place; it is there and cannot be repressed. The truth in the heart is a wellspring of life. It refreshes the weary and restrains vile thought and utterance.

Is there not enough taking place about us to show us the dangers that beset our path? Everywhere are seen wrecks of humanity, neglected family altars, broken-up families. There is a strange abandonment of principle, a lowering of the standard of morality; the sins are fast increasing which caused the judgments of God to be poured upon the earth in the Flood and in the destruction of Sodom by fire. We are nearing the end. God has borne long with the perversity of mankind, but their punishment is no less certain. Let those who profess to be the light of the world depart from all iniquity.—Testimonies for the Church, vol. 5, pp. 600, 601.

The home in which the members are polite, courteous Christians exerts a far-reaching influence for good. Other families will mark the results attained by such a home, and will follow the example set, in their turn guarding the home against satanic influences. The angels of God will often visit the home in which the will of God bears sway. Under the power of divine grace such a home becomes a place of refreshing to worn, weary pilgrims. By watchful guarding, self is kept from asserting itself. Correct habits are formed. There is a careful recognition of the rights of others. The faith that works by love and purifies the soul stands at the helm, presiding over the whole household. Under the hallowed influence of such a home, the principle of brotherhood laid down in the Word of God is more widely recognized and obeyed.—Sons and Daughters of God, p. 258.

\section*{Sunday – Dysfunction at Home}

The sin of Jacob, and the train of events to which it led, had not failed to exert an influence for evil—an influence that revealed its bitter fruit in the character and life of his sons. As these sons arrived at manhood they developed serious faults. The results of polygamy were manifest in the household. This terrible evil tends to dry up the very springs of love, and its influence weakens the most sacred ties. The jealousy of the several mothers had embittered the family relation, the children had grown up contentious and impatient of control, and the father’s life was darkened with anxiety and grief.—Patriarchs and Prophets, p. 208.

Joseph listened to his father’s instructions, and feared the Lord. He was more obedient to his father’s righteous teachings than any of his brethren. He treasured his instructions, and with integrity of heart, loved to obey God. He was grieved at the wrong conduct of some of his brethren, and meekly entreated them to pursue a righteous course, and leave off their wicked acts. This only embittered them against him. His hatred of sin was such that he could not endure to see his brethren sinning against God. He laid the matter before his father, hoping that his authority might reform them. This exposure of their wrongs enraged his brethren against him. They had observed their father’s strong love for Joseph, and were envious at him. Their envy grew into hatred, and finally to murder.—Spiritual Gifts, vol. 3, p. 138.

His mother being dead, [Joseph’s] affections clung the more closely to the father, and Jacob’s heart was bound up in this child of his old age. He “loved Joseph more than all his children.”

But even this affection was to become a cause of trouble and sorrow. Jacob unwisely manifested his preference for Joseph, and this excited the jealousy of his other sons. The father’s injudicious gift to Joseph of a costly coat, or tunic, … excited a suspicion that he intended to pass by his elder children, to bestow the birthright upon the son of Rachel. Their malice was still further increased as the boy one day told them of a dream that he had had.

As the lad stood before his brothers, his beautiful countenance lighted up with the Spirit of Inspiration, they could not withhold their admiration; but they did not choose to renounce their evil ways, and they hated the purity that reproved their sins. The same spirit that actuated Cain was kindling in their hearts.—Conflict and Courage, p. 72.

The home that is beautified by love, sympathy, and tenderness is a place that angels love to visit, and where God is glorified. The influence of a carefully guarded Christian home in the years of child-hood and youth is the surest safeguard against the corruptions of the world. In the atmosphere of such a home the children will learn to love both their earthly parents and their heavenly Father.

Every Christian family should illustrate to the world the power and excellence of Christian influence. Parents should realize their accountability to keep their homes free from every taint of moral evil.—The Adventist Home, p. 19.

\section*{Monday – Choosing a New Direction}

Joseph regarded his being sold into Egypt as the greatest calamity that could have befallen him; but he saw the necessity of trusting in God as he had never done when protected by his father’s love. …

But, in the providence of God, even this experience was to be a blessing to him. He had learned in a few hours that which years might not otherwise have taught him. His father, strong and tender as his love had been, had done him wrong by his partiality and indulgence. This unwise preference had angered his brothers and provoked them to the cruel deed that had separated him from his home. Its effects were manifest also in his own character. Faults had been encouraged that were now to be corrected.—Conflict and Courage, p. 73.

Arriving in Egypt, Joseph was sold to Potiphar, captain of the king’s guard, in whose service he remained for ten years. He was here exposed to temptations of no ordinary character. He was in the midst of idolatry. The worship of false gods was surrounded by all the pomp of royalty, supported by the wealth and culture of the most highly civilized nation then in existence. Yet Joseph preserved his simplicity and his fidelity to God. The sights and sounds of vice were all about him, but he was as one who saw and heard not. His thoughts were not permitted to linger upon forbidden subjects. The desire to gain the favor of the Egyptians could not cause him to conceal his principles. Had he attempted to do this, he would have been overcome by temptation; but he was not ashamed of the religion of his fathers, and he made no effort to hide the fact that he was a worshiper of Jehovah. Potiphar’s confidence in Joseph increased daily, and he finally promoted him to be his steward, with full control over all his possessions.—Conflict and Courage, p. 74.

There are thousands today echoing [Satan’s] rebellious complaint against God. They do not see that to deprive man of the freedom of choice would be to rob him of his prerogative as an intelligent being, and make him a mere automaton. It is not God’s purpose to coerce the will. Man was created a free moral agent. Like the inhabitants of all other worlds, he must be subjected to the test of obedience; but he is never brought into such a position that yielding to evil becomes a matter of necessity. No temptation or trial is permitted to come to him which he is unable to resist. God made such ample provision that man need never have been defeated in the conflict with Satan.—Patriarchs and Prophets, p. 331.

\section*{Tuesday – Finding True Self-Worth}

The rabbis understood Christ’s parable as applying to the publicans and sinners; but it has also a wider meaning. By the lost sheep Christ represents not only the individual sinner but the one world that has apostatized and has been ruined by sin. This world is but an atom in the vast dominions over which God presides, yet this little fallen world—the one lost sheep—is more precious in His sight than are the ninety and nine that went not astray from the fold. Christ, the loved Commander in the heavenly courts, stooped from His high estate, laid aside the glory that He had with the Father, in order to save the one lost world. For this He left the sinless worlds on high, the ninety and nine that loved Him, and came to this earth, to be “wounded for our transgressions” and “bruised for our iniquities.” (Isaiah 53:5.) God gave Himself in His Son that He might have the joy of receiving back the sheep that was lost.

“Behold, what manner of love the Father hath bestowed upon us, that we should be called the sons of God.” 1 John 3:1.—Christ’s Object Lessons, pp. 190, 191.

Christ and Him crucified should become the theme of our thoughts and stir the deepest emotions of our souls. The true followers of Christ will appreciate the great salvation which He has wrought for them; and wherever He leads the way, they will follow. They will consider it a privilege to bear whatever burdens Christ may lay upon them. It is through the cross alone that we can estimate the worth of the human soul. Such is the value of men for whom Christ died that the Father is satisfied with the infinite price which He pays for the salvation of man in yielding up His own Son to die for their redemption. What wisdom, mercy, and love in its fullness are here manifested! The worth of man is known only by going to Calvary. In the mystery of the cross of Christ we can place an estimate upon man.—Testimonies for the Church, vol. 2, p. 634.

Our God is an ever-present help in every time of need. He is perfectly acquainted with the most secret thoughts of our heart, with all the intents and purposes of our souls. When we are in perplexity, even before we open to Him our distress, He is making arrangements for our deliverance. Our sorrow is not unnoticed. He always knows much better than we do, just what is necessary for the good of His children, and He leads us as we would choose to be led if we could discern our own hearts and see our necessities and perils, as God sees them. But finite beings seldom know themselves. They do not understand their own weakness. God knows them better than they know themselves, and He understands how to lead them.—Our High Calling, p. 316.

\section*{Wednesday – Doing Relationships God’s Way}

How fierce was the assault upon Joseph’s morals. It came from one of influence, the most likely to lead astray. Yet how promptly and firmly was it resisted. He suffered for his virtue and integrity, for she who would lead him astray revenged herself upon the virtue she could not subvert, and by her influence caused him to be cast into prison, by charging him with a foul wrong. Here Joseph suffered because he would not yield his integrity. He had placed his reputation and interest in the hands of God. And although he was suffered to be afflicted for a time, to prepare him to fill an important position, yet God safely guarded that reputation that was blackened by a wicked accuser, and afterward, in His own good time, caused it to shine. God made even the prison the way to his elevation. Virtue will in time bring its own reward. The shield which covered Joseph’s heart was the fear of God, which caused him to be faithful and just to his master and true to God.—The Story of Redemption, p. 102.

Just to the degree in which the word of God is received and obeyed will it impress with its potency and touch with its life every spring of action, every phase of character. It will purify every thought, regulate every desire. Those who make God’s word their trust will quit themselves like men and be strong. They will rise above all baser things into an atmosphere free from defilement.

When man is in fellowship with God, that unswerving purpose which preserved Joseph and Daniel amidst the corruption of heathen courts will make his a life of unsullied purity. His robes of character will be spotless. In his life the light of Christ will be undimmed. The bright and morning Star will appear shining steadfastly above him in changeless glory.—The Ministry of Healing, p. 136.

The Saviour’s lesson, “Resist not him that is evil,” was a hard saying for the revengeful Jews, and they murmured against it among themselves. But Jesus now made a still stronger declaration:

“Ye have heard that it hath been said, Thou shalt love thy neighbor, and hate thine enemy. But I say unto you, Love your enemies, bless them that curse you, do good to them that hate you, and pray for them which despitefully use you and persecute you; that ye may be the children of your Father which is in heaven.” …

The children of God are those who are partakers of His nature. It is not earthly rank, nor birth, nor nationality, nor religious privilege, which proves that we are members of the family of God; it is love, a love that embraces all humanity. Even sinners whose hearts are not utterly closed to God’s Spirit, will respond to kindness; while they may give hate for hate, they will also give love for love. But it is only the Spirit of God that gives love for hatred. To be kind to the unthankful and to the evil, to do good hoping for nothing again, is the insignia of the royalty of heaven, the sure token by which the children of the Highest reveal their high estate.—Thoughts From the Mount of Blessing, pp. 73, 75.

\section*{Thursday – The Great Controversy, Up Close and Personal}

Joseph was one of the few who could withstand temptation. He showed that he had an eye single to the glory or God. He evidenced a lofty regard for God’s will, alike when occupying the prisoner’s cell and when standing next the throne. He carried his religion with him wherever he went and in whatever situation he was placed. True religion has an all-pervading power. It gives tone to everything man does. You need not go out of the world in order to be a Christian, but you may carry your religion, with all its sanctifying influences, into all you do and say. You may discharge well the duties belonging to the situation where God has placed you, by keeping the heart fixed upon heavenly things.—Testimonies for the Church, vol. 5, p. 124.

Life in this stormy world, where moral darkness triumphs over truth and virtue, will be to the Christian a continual conflict. He will find that he must keep the armor on, for he will have to fight against forces that never tire and foes that never sleep. We shall find ourselves beset with countless temptations, and we must find strength in Christ to overcome them or be overcome by them and lose our souls. We have a great and solemn work to do, and how terrible will be our loss if we fail.—Testimonies for the Church, vol. 3, p. 453.

There are many who do not understand the conflict that is going on between Christ and Satan over the souls of men. They do not realize that if they would stand under the blood-stained banner of Prince Emmanuel they must be willing to be partakers of His conflicts and wage a determined war against the powers of darkness.

When thinking on the conflict, Paul writes to his Ephesian brethren exhorting them to … “Put on the whole armour of God, that ye may be able to stand against the wiles of the devil. For we wrestle not against flesh and blood, but against principalities, against powers, against the rulers of the darkness of this world, against spiritual wickedness in high places.” (verses 10-13).

The gaining of eternal life will ever involve a struggle, a conflict. We are continually to be found fighting the good fight of faith. We are soldiers of Christ; and those who enlist in His army are expected to do difficult work, work which will tax their energies to the utmost. …

Victories are not gained by ceremonies or display but by simple obedience to the highest General, the Lord God of heaven. He who trusts in this Leader will never know defeat. Obedience to God is liberty from the thraldom of sin, deliverance from human passion and impulse. Man may stand conqueror of himself, conqueror of his own inclinations, conqueror of principalities and powers, and of the “rulers of the darkness of this world,” and of “spiritual wickedness in high places.”—In Heavenly Places, p. 259.

\section*{Friday – Further Thought}

\setlength{\parindent}{0pt}The Upward Look, “Get Acquainted With God,” p. 248;

In Heavenly Places, “No Exemption From Sorrow,” p. 268.

\end{multicols}

\end{document}

