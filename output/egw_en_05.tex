\documentclass[a4paper, 10pt, twoside, headings=small]{scrartcl}

%%%%%%%%%%%%%%%%%%%% General %%%%%%%%%%%%%%%%%%%%%%%%%%%%%%%%%%%%%%%%%%%%%%%%%%%

\usepackage[utf8]{inputenc}
\usepackage[T1]{fontenc}
\usepackage[protrusion=true, expansion]{microtype} 

\usepackage[hidelinks]{hyperref} % Must go after the patches


%%%%%%%%%%%%%%%%%%%% Fonts %%%%%%%%%%%%%%%%%%%%%%%%%%%%%%%%%%%%%%%%%%%%%%%%%%%%%


\usepackage[english]{babel}



\usepackage{Charter}
\addtokomafont{disposition}{\rmfamily}


\usepackage{textcomp}

\linespread{1.1}  


%%%%%%%%%%%%%%%%%%%% Random Packages  %%%%%%%%%%%%%%%%%%%%%%%%%%%%%%%%%%%%%%%

\usepackage{geometry}	
\usepackage{enumerate} %Erweiterung der enumerate-Umgebung
\usepackage{ifthen,calc}
\usepackage{mathrsfs,amssymb} %Zusatzzeichen
\usepackage{wrapfig}
\usepackage[retainorgcmds]{IEEEtrantools} %Besonders geeignet für einen mehrzeilige Formelsatz
\usepackage{theorem} %Theoremlayout
\usepackage{multicol}
\setlength\columnsep{20pt}
\usepackage{csquotes}





%%%%%%%%%%%%%%%%%%%% Page %%%%%%%%%%%%%%%%%%%%%%%%%%%%%%%%%%%%%%%%%%%%%%%%%%%

% twoside
\geometry{left=2.7cm, right=2.3cm, top=2.7cm, bottom=2.2cm}


%%%%%%%%%%%% PDF %%%%%%%%%%%%%%%%%%%%%%%%%%%%%%%%%%%%%%%%%%%%%%%%%%%%%%%%%%%%

\hypersetup{
	hidelinks=true,
	%	linkcolor=black,
	%	filecolor=black,      
	%	urlcolor=black,
	%	citecolor=black,
	%	allcolors=black,
	%	allbordercolors=white,
	%pdfpagemode=FullScreen,
	%	pdftitle={\mytitle},
	%	pdfauthor={\myauthor},
	pdfkeywords={},
	%	pdfcreator={Some fancy PDF-Creator...},
	bookmarksnumbered=true
}



%%%%%%%%%%%% footnotes %%%%%%%%%%%%%%%%%%%%%%%%%%%%%%%%%%%%%%%%%%%%%%%%%%%%%%

\usepackage[flushmargin, hang]{footmisc} % flush footnote mark to left margin
\usepackage{regexpatch}
\makeatletter
% 1. remove all redefinitions about footnotes done by \maketitle
%    and add \titletrue
\regexpatchcmd{\maketitle}
{\c{def}\c{@makefnmark}.*\c{if@twocolumn}}
{\c{titletrue}\c{if@twocolumn}}
{}{}
% 2. define a conditional
\newif\iftitle
%% 3. redefine \@makefnmark to print nothing when \titletrue
%\xpretocmd{\@makefnmark}{\iftitle\else}{}{}
%\xapptocmd{\@makefnmark}{\fi}{}{}
% 4. ensure \@makefntext has \titlefalse
%    that's justified by the fact that \@makefnmark
%    in \@makefntext is set in a box
\xpretocmd{\@makefntext}{\titlefalse}{}{}

\makeatother

\renewcommand{\footnotemargin}{1em}
%\addtolength{\footnotesep}{5mm}
\skip\footins=2\bigskipamount     % Determine the space above the rule
\renewcommand*{\footnoterule}{%
	\kern-3pt%
	\hrule width 1in%
	\kern 2.6pt%
	\vspace{\smallskipamount}       % The additional space below the rule
}



%%%%%%%%%%%% captions %%%%%%%%%%%%%%%%%%%%%%%%%%%%%%%%%%%%%%%%%%%%%%%%%%%%%%%

\usepackage[textfont={small},labelfont={small, bf}]{caption}
\DeclareCaptionFont{black}{ \color{white} }
\DeclareCaptionFormat{listing}{
	\colorbox[cmyk]{0.43, 0.35, 0.35,0.01 }{
		\parbox{\textwidth}{\hspace{15pt}#1#2#3}
	}
}
\captionsetup{format=plain, singlelinecheck=true}
\captionsetup[lstlisting]{labelfont={small, bf}, textfont={small}}


%%%%%%%%%%%% Title %%%%%%%%%%%%%%%%%%%%%%%%%%%%%%%%%%%%%%%%%%%%%%%%%%%%%%%%%%

\usepackage{titling}
\setlength{\droptitle}{-5em}
\pretitle{\begin{center}\LARGE\normalfont\scshape}
	\posttitle{\par\end{center}}
\preauthor{\begin{center}}
	\postauthor{\par\end{center}}
\predate{\begin{center}}
	\postdate{\par\end{center}}


%%%%%%%%%%%% footer / header %%%%%%%%%%%%%%%%%%%%%%%%%%%%%%%%%%%%%%%%%%%%%%%%

\usepackage{fancyhdr}

% twoside with subsection
\fancyhf{}
\fancyhead[RE]{\small\nouppercase\leftmark}
\fancyhead[LO]{\small\rightmark}
\fancyhead[LE,RO]{\thepage}
\renewcommand{\headrulewidth}{0pt}


% Does not really work...
%\setotherlanguage{greek}
%\setotherlanguage{hebrew}
%\newfontfamily\greekfont[]{Linux Libertine O}
%\newfontfamily\hebrewfont[]{Linux Libertine O}

\newcommand{\bm}{\vectorbold*} % using physics package

\newcommand{\matlab}{\textsc{Matlab}\textsuperscript{\tiny{\textregistered}}}



\setmainlanguage[]{english}

\title{05 “ ‘Come to Me . . .’ ”}

\author{Ellen G.\ White}

\date{2021/03 Rest in Christ}

\begin{document}

\maketitle

\thispagestyle{empty}

\pagestyle{fancy}

\begin{multicols}{2}

\section*{Saturday – “ ‘Come to Me . . .’ ”}

It is not work but overwork, without periods of rest, that breaks people down, endangering the life-forces. Those who overwork soon reach the place where they work in a hopeless way.

The work done to the Lord is done in cheerfulness and with courage. God wants us to bring spirit and life and hopefulness into our work. Brain workers should give due attention to every part of the human machinery, equalizing the taxation. Physical and mental effort, wisely combined, will keep the whole man in a condition that makes him acceptable to God.

Bring into the day’s work hopefulness, courage, and amiability. Do not overwork. Better far leave undone some of the things planned for the day’s work than to undo oneself and become overtaxed, losing the courage necessary for the performance of the tasks of the next day. Do not today violate the laws of nature, lest you lose your strength for the day to follow.—Mind, Character, and Personality, vol. 2, pp. 375, 376.

Christ longs to have care-worn, weary, oppressed human beings come to Him. He longs to give them the light and joy and peace that are to be found nowhere else. The veriest sinners are the objects of His deep, earnest pity and love. He sends His Holy Spirit to yearn over them with tenderness, seeking to draw them to Himself.—Christ’s Object Lessons, p. 225.

“Come unto Me, all ye that labor and are heavy-laden, and I will give you rest.”

These words of comfort were spoken to the multitude that followed Jesus. The Saviour had said that only through Himself could men receive a knowledge of God. He had spoken of His disciples as the ones to whom a knowledge of heavenly things had been given. But He left none to feel themselves shut out from His care and love. All who labor and are heavy-laden may come unto Him.—The Desire of Ages, p. 328.

As through Jesus we enter into rest, heaven begins here. We respond to His invitation, Come, learn of Me, and in thus coming we begin the life eternal. Heaven is a ceaseless approaching to God through Christ. The longer we are in the heaven of bliss, the more and still more of glory will be opened to us; and the more we know of God, the more intense will be our happiness. As we walk with Jesus in this life, we may be filled with His love, satisfied with His presence. All that human nature can bear, we may receive here. But what is this compared with the hereafter? There “are they before the throne of God, and serve Him day and night in His temple: and He that sitteth on the throne shall dwell among them. They shall hunger no more, neither thirst any more; neither shall the sun light on them, nor any heat. For the Lamb which is in the midst of the throne shall feed them, and shall lead them unto living fountains of waters: and God shall wipe away all tears from their eyes.” Revelation 7:15-17.—The Desire of Ages, pp. 331, 332.

\section*{Sunday – “ ‘I Will Give You Rest’ ”}

I feel urged by the Spirit of the Lord to tell you that now is your day of privilege, of trust, of blessing. Will you improve it? Are you working for the glory of God, or for selfish interests? Are you keeping before your mind’s eye brilliant prospects of worldly success, whereby you may obtain self-gratification and financial gain? If so, you will be most bitterly disappointed. But if you seek to live a pure and holy life, to learn daily in the school of Christ the lessons that He has invited you to learn, to be meek and lowly in heart, then you have a peace which no worldly circumstances can change.

A life in Christ is a life of restfulness. Uneasiness, dissatisfaction, and restlessness reveal the absence of the Saviour. If Jesus is brought into the life, that life will be filled with good and noble works for the Master. You will forget to be self-serving, and will live closer and still closer to the dear Saviour; your character will become Christlike, and all around you will take knowledge that you have been with Jesus and learned of Him.—Testimonies for the Church, vol. 5, p. 487.

Christ is the wellspring of life. That which many need is to have a clearer knowledge of Him; they need to be patiently and kindly, yet earnestly, taught how the whole being may be thrown open to the healing agencies of heaven. When the sunlight of God’s love illuminates the darkened chambers of the soul, restless weariness and dissatisfaction will cease, and satisfying joys will give vigor to the mind and health and energy to the body. …

We are not to let the future, with its hard problems, its unsatisfying prospects, make our hearts faint, our knees tremble, our hands hang down. “Let him take hold of My strength,” says the Mighty One, “that he may make peace with Me; and he shall make peace with Me.” Isaiah 27:5. Those who surrender their lives to His guidance and to His service will never be placed in a position for which He has not made provision. Whatever our situation, if we are doers of His word, we have a Guide to direct our way; whatever our perplexity, we have a sure Counselor; whatever our sorrow, bereavement, or loneliness, we have a sympathizing Friend.—The Ministry of Healing, pp. 247, 248.

Have you, have I, fully comprehended the gracious call, “Come unto me”? He says, “Abide in me,” not Abide with Me. “Do understand My call. Come to Me to stay with Me.” He will freely bestow all blessings connected with Himself upon all who come to Him for life. … You are privileged with His abiding presence in the place of a short-lived privilege that is not lasting as you engage in the duties of life. Will anxiety, perplexity, and cares drive you away from Christ? Are we less dependent upon God when in the workshop, in the field, in the market-place? … The Lord Jesus will abide with you and you with Him in every place.—In Heavenly Places, p. 55.

\section*{Monday – “ ‘Take My Yoke Upon You’ ”}

It was to speak to all men, and draw them across the gulf that sin had made, to unite finite man with the infinite God. It is the power of the cross alone that can separate man from the strong confederacy of sin. Christ gave Himself for the saving of the sinner. Those whose sins are forgiven, who love Jesus, will be united with Him. They will bear the yoke of Christ. This yoke is not to hamper them, not to make their religious life one of unsatisfying toil. No; the yoke of Christ is to be the very means by which the Christian life is to become one of pleasure and joy. The Christian is to be joyful in contemplation of that which the Lord has done in giving His only-begotten Son to die for the world, “that whosoever believeth in Him should not perish, but have everlasting life.—Messages to Young People, p. 138.

Jesus says, “Abide in Me.” These words convey the idea of rest, stability, confidence. Again He invites, “Come unto Me, … and I will give you rest.” Matthew 11:28. The words of the psalmist express the same thought: “Rest in the Lord, and wait patiently for Him.” And Isaiah gives the assurance, “In quietness and in confidence shall be your strength.” Psalm 37:7; Isaiah 30:15. This rest is not found in inactivity; for in the Saviour’s invitation the promise of rest is united with the call to labor: “Take My yoke upon you: … and ye shall find rest.” Matthew 11:29. The heart that rests most fully upon Christ will be most earnest and active in labor for Him.—Steps to Christ, p. 71.

Jesus invites the weary and heavy-laden with promises of rest if they will come to Him. He invites them to exchange the galling yoke of selfishness and covetousness, which makes them slaves to mammon, for His yoke, which He declares is easy, and His burden, which is light. … Those who refuse to accept the relief which Christ offers them, and continue to wear the galling yoke of selfishness, tasking their souls to the utmost in plans to accumulate money for selfish gratification, have not experienced the peace and rest found in bearing the yoke of Christ and lifting the burdens of self-denial and disinterested benevolence which Christ has borne in their behalf.—Testimonies for the Church, vol. 3, p. 384.

Wearing the yoke with Christ, means to work in His lines, to be a copartner with Him in His sufferings and toils for lost humanity. It means to be a wise instructor of souls. We shall be what we are willing to be made by Christ in these precious hours of probation. We shall be the sort of a vessel that we allow ourselves to be molded into. We must unite with God in the molding and fashioning work, having our wills submitted to the divine will.—Ellen G. White Comments, in The SDA Bible Commentary, vol. 5, p. 1092.

\section*{Tuesday – “ ‘I Am Gentle and Lowly in Heart’ ”}

“Learn of Me,” says Jesus; “for I am meek and lowly in heart: and ye shall find rest.” We are to enter the school of Christ, to learn from Him meekness and lowliness. Redemption is that process by which the soul is trained for heaven. This training means a knowledge of Christ. It means emancipation from ideas, habits, and practices that have been gained in the school of the prince of darkness. The soul must be delivered from all that is opposed to loyalty to God.

In the heart of Christ, where reigned perfect harmony with God, there was perfect peace. He was never elated by applause, nor dejected by censure or disappointment. Amid the greatest opposition and the most cruel treatment, He was still of good courage. But many who profess to be His followers have an anxious, troubled heart, because they are afraid to trust themselves with God. They do not make a complete surrender to Him; for they shrink from the consequences that such a surrender may involve. Unless they do make this surrender, they cannot find peace.—The Desire of Ages, p. 330.

You are safe only as, in perfect submission and obedience, you connect yourselves with Christ. The yoke is easy, for Christ carries the weight. As you lift the burden of the cross, it will become light; and that cross is to you a pledge of eternal life. It is the privilege of each to follow gladly after Christ, exclaiming at every step, “Thy gentleness hath made me great.” But if we would travel heavenward, we must take the Word of God as our lesson-book. In the words of inspiration we must read our lessons day by day.

The humiliation of the man Christ Jesus is incomprehensible to the human mind; but His divinity and His existence before the world was formed can never be doubted by those who believe the Word of God. The apostle Paul speaks of our Mediator, the only begotten Son of God, who in a state of glory was in the form of God, the Commander of all the heavenly hosts, and who, when He clothed His divinity with humanity, took upon Him the form of a servant.—Sons and Daughters of God, p. 81.

In consenting to become man, Christ manifested a humility that is the marvel of the heavenly intelligences. The act of consenting to be a man would be no humiliation were it not for the fact of Christ’s exalted pre-existence. We must open our understanding to realize that Christ laid aside His royal robe, His kingly crown, His high command, and clothed His divinity with humanity, that He might meet man where he was, and bring to the human family moral power to become the sons and daughters of God. To redeem man, Christ became obedient unto death, even the death of the cross.—The Youth’s Instructor, October 13, 1898.

\section*{Wednesday – “ ‘For My Yoke Is Easy’ ”}

“Take My yoke upon you,” Jesus says. The yoke is an instrument of service. Cattle are yoked for labor, and the yoke is essential that they may labor effectually. By this illustration Christ teaches us that we are called to service as long as life shall last. We are to take upon us His yoke, that we may be co-workers with Him.

The yoke that binds to service is the law of God. The great law of love revealed in Eden, proclaimed upon Sinai, and in the new covenant written in the heart, is that which binds the human worker to the will of God. If we were left to follow our own inclinations, to go just where our will would lead us, we should fall into Satan’s ranks and become possessors of his attributes. Therefore God confines us to His will, which is high, and noble, and elevating. He desires that we shall patiently and wisely take up the duties of service. The yoke of service Christ Himself has borne in humanity. He said, “I delight to do Thy will, O My God: yea, Thy law is within My heart.” Psalm 40:8. “I came down from heaven, not to do Mine own will, but the will of Him that sent Me.” John 6:38. Love for God, zeal for His glory, and love for fallen humanity, brought Jesus to earth to suffer and to die. This was the controlling power of His life. This principle He bids us adopt.—The Desire of Ages, p. 329.

Men need to learn that the blessings of obedience, in their fullness, can be theirs only as they receive the grace of Christ. It is His grace that gives man power to obey the laws of God. It is this that enables him to break the bondage of evil habit. This is the only power that can make him and keep him steadfast in the right path.

When the gospel is received in its purity and power, it is a cure for the maladies that originated in sin. The Sun of Righteousness arises, “with healing in His wings.” Malachi 4:2. Not all this world bestows can heal a broken heart, or impart peace of mind, or remove care, or banish disease. Fame, genius, talent—all are powerless to gladden the sorrowful heart or to restore the wasted life. The life of God in the soul is man’s only hope.

The love which Christ diffuses through the whole being is a vitalizing power. Every vital part—the brain, the heart, the nerves—it touches with healing. By it the highest energies of the being are roused to activity. It frees the soul from the guilt and sorrow, the anxiety and care, that crush the life forces. With it come serenity and composure. It implants in the soul, joy that nothing earthly can destroy,—joy in the Holy Spirit,—health-giving, life-giving joy.—The Ministry of Healing, p. 115.

\section*{Thursday – “ ‘My Burden Is Light’ ”}

There are many whose hearts are aching under a load of care because they seek to reach the world’s standard. They have chosen its service, accepted its perplexities, adopted its customs. Thus their character is marred, and their life made a weariness. In order to gratify ambition and worldly desires, they wound the conscience, and bring upon themselves an additional burden of remorse. The continual worry is wearing out the life forces. Our Lord desires them to lay aside this yoke of bondage. He invites them to accept His yoke; He says, “My yoke is easy, and My burden is light.” He bids them seek first the kingdom of God and His righteousness, and His promise is that all things needful to them for this life shall be added. Worry is blind, and cannot discern the future; but Jesus sees the end from the beginning. In every difficulty He has His way prepared to bring relief. Our heavenly Father has a thousand ways to provide for us, of which we know nothing. Those who accept the one principle of making the service and honor of God supreme will find perplexities vanish, and a plain path before their feet.—The Desire of Ages, p. 330.

Jesus invites you to lay down the yoke you have been bearing, which has been galling your neck, and take His yoke, which is easy, and His burden, which is light. How wearisome is the load of self-love, covetousness, pride, passion, jealousy, and evil surmising. Yet how closely do men clasp these curses, and how loath are they to give them up. Christ understands how grievous are these self-imposed burdens, and He invites us to lay them down. The heavy-laden and weary souls He invites to come to Him, and take His burden, which is light, in exchange for the burdens which they bind upon themselves. He says: “Ye shall find rest unto your souls. For My yoke is easy, and My burden is light.” The requirements of our Saviour are all consistent and harmonious, and if cheerfully borne will bring peace and rest to the soul.—Testimonies for the Church, vol. 4, p. 240.

It was to save sinners that Christ left His home in heaven and came to earth to suffer and to die. For this He toiled and agonized and prayed, until, heartbroken and deserted by those He came to save, He poured out His life on Calvary.

Many shrink from such a life as our Saviour lived. They feel that it requires too great a sacrifice to imitate the Pattern, to bring forth fruit in good works, and then patiently endure the pruning of God that they may bring forth more fruit. But when the Christian regards himself as only a humble instrument in the hands of Christ, and endeavors to faithfully perform every duty, relying upon the help which God has promised, then he will wear the yoke of Christ and find it easy; then he will bear burdens for Christ, and pronounce them light. He can look up with courage and with confidence, and say, “I know whom I have believed, and am persuaded that he is able to keep that which I have committed unto him” (2 Timothy 1:12).—The Sanctified Life, p. 82.

\section*{Friday – Further Thought}

\setlength{\parindent}{0pt}My Life Today, “Being Good and Doing Good,” p. 168;

Sons and Daughters of God, “He Is Near to All That Call Upon Him,” p. 19.

\end{multicols}

\end{document}

