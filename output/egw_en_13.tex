\documentclass[a4paper, 10pt, twoside, headings=small]{scrartcl}

%%%%%%%%%%%%%%%%%%%% General %%%%%%%%%%%%%%%%%%%%%%%%%%%%%%%%%%%%%%%%%%%%%%%%%%%

\usepackage[utf8]{inputenc}
\usepackage[T1]{fontenc}
%\usepackage[protrusion=true, expansion]{microtype} 

\usepackage[hidelinks]{hyperref} % Must go after the patches


%%%%%%%%%%%%%%%%%%%% Fonts %%%%%%%%%%%%%%%%%%%%%%%%%%%%%%%%%%%%%%%%%%%%%%%%%%%%%

\usepackage{fontspec}

\usepackage{polyglossia}


\addtokomafont{disposition}{\rmfamily}


\usepackage{textcomp}

\linespread{1.1}  


\usepackage[final]{microtype}
\setmainfont[Ligatures=TeX]{XCharter}


%%%%%%%%%%%%%%%%%%%% Random Packages  %%%%%%%%%%%%%%%%%%%%%%%%%%%%%%%%%%%%%%%

\usepackage{geometry}	
\usepackage{enumerate} %Erweiterung der enumerate-Umgebung
\usepackage{ifthen,calc}
\usepackage{mathrsfs,amssymb} %Zusatzzeichen
\usepackage{wrapfig}
\usepackage[retainorgcmds]{IEEEtrantools} %Besonders geeignet für einen mehrzeilige Formelsatz
\usepackage{theorem} %Theoremlayout
\usepackage{multicol}
\setlength\columnsep{20pt}
\usepackage{csquotes}





%%%%%%%%%%%%%%%%%%%% Page %%%%%%%%%%%%%%%%%%%%%%%%%%%%%%%%%%%%%%%%%%%%%%%%%%%

% twoside
\geometry{left=2.7cm, right=2.3cm, top=2.7cm, bottom=2.2cm}


%%%%%%%%%%%% PDF %%%%%%%%%%%%%%%%%%%%%%%%%%%%%%%%%%%%%%%%%%%%%%%%%%%%%%%%%%%%

\hypersetup{
	hidelinks=true,
	%	linkcolor=black,
	%	filecolor=black,      
	%	urlcolor=black,
	%	citecolor=black,
	%	allcolors=black,
	%	allbordercolors=white,
	%pdfpagemode=FullScreen,
	%	pdftitle={\mytitle},
	%	pdfauthor={\myauthor},
	pdfkeywords={},
	%	pdfcreator={Some fancy PDF-Creator...},
	bookmarksnumbered=true
}



%%%%%%%%%%%% footnotes %%%%%%%%%%%%%%%%%%%%%%%%%%%%%%%%%%%%%%%%%%%%%%%%%%%%%%

\usepackage[flushmargin, hang]{footmisc} % flush footnote mark to left margin
\usepackage{regexpatch}
\makeatletter
% 1. remove all redefinitions about footnotes done by \maketitle
%    and add \titletrue
\regexpatchcmd{\maketitle}
{\c{def}\c{@makefnmark}.*\c{if@twocolumn}}
{\c{titletrue}\c{if@twocolumn}}
{}{}
% 2. define a conditional
\newif\iftitle
%% 3. redefine \@makefnmark to print nothing when \titletrue
%\xpretocmd{\@makefnmark}{\iftitle\else}{}{}
%\xapptocmd{\@makefnmark}{\fi}{}{}
% 4. ensure \@makefntext has \titlefalse
%    that's justified by the fact that \@makefnmark
%    in \@makefntext is set in a box
\xpretocmd{\@makefntext}{\titlefalse}{}{}

\makeatother

\renewcommand{\footnotemargin}{1em}
%\addtolength{\footnotesep}{5mm}
\skip\footins=2\bigskipamount     % Determine the space above the rule
\renewcommand*{\footnoterule}{%
	\kern-3pt%
	\hrule width 1in%
	\kern 2.6pt%
	\vspace{\smallskipamount}       % The additional space below the rule
}



%%%%%%%%%%%% captions %%%%%%%%%%%%%%%%%%%%%%%%%%%%%%%%%%%%%%%%%%%%%%%%%%%%%%%

\usepackage[textfont={small},labelfont={small, bf}]{caption}
\DeclareCaptionFont{black}{ \color{white} }
\DeclareCaptionFormat{listing}{
	\colorbox[cmyk]{0.43, 0.35, 0.35,0.01 }{
		\parbox{\textwidth}{\hspace{15pt}#1#2#3}
	}
}
\captionsetup{format=plain, singlelinecheck=true}
\captionsetup[lstlisting]{labelfont={small, bf}, textfont={small}}


%%%%%%%%%%%% Title %%%%%%%%%%%%%%%%%%%%%%%%%%%%%%%%%%%%%%%%%%%%%%%%%%%%%%%%%%

\usepackage{titling}
\setlength{\droptitle}{-5em}
\pretitle{\begin{center}\LARGE\bfseries}
	\posttitle{\par\end{center}}
\preauthor{\begin{center}}
	\postauthor{\par\end{center}}
\predate{\begin{center}}
	\postdate{\par\end{center}}


%%%%%%%%%%%% footer / header %%%%%%%%%%%%%%%%%%%%%%%%%%%%%%%%%%%%%%%%%%%%%%%%

\usepackage{fancyhdr}

% twoside with subsection
\fancyhf{}
\fancyhead[RE]{\small\nouppercase\leftmark}
\fancyhead[LO]{\small\rightmark}
\fancyhead[LE,RO]{\thepage}
\renewcommand{\headrulewidth}{0pt}


% Does not really work...
%\setotherlanguage{greek}
%\setotherlanguage{hebrew}
%\newfontfamily\greekfont[]{Linux Libertine O}
%\newfontfamily\hebrewfont[]{Linux Libertine O}

\newcommand{\bm}{\vectorbold*} % using physics package

\newcommand{\matlab}{\textsc{Matlab}\textsuperscript{\tiny{\textregistered}}}



\setmainlanguage[]{english}

\title{13 The Ultimate Rest}

\author{Ellen G.\ White}

\date{2021/03 Rest in Christ}

\begin{document}

\maketitle

\thispagestyle{empty}

\pagestyle{fancy}

\begin{multicols}{2}

\section*{Saturday – The Ultimate Rest}

Christ entered upon His mission of mercy, and from the manger to the cross was beset by the enemy. Satan contested every inch of ground, exerting his utmost power to overcome Him. Like a tempest temptation after temptation beat upon Him. But the more mercilessly they fell, the more firmly did the Son of God cling to the hand of His Father, and press on in the bloodstained path.

The severity of the conflict through which Christ passed was proportionate to the vastness of the interests involved in His success or failure…. Satan sought to overthrow Christ, in order that he himself might continue to reign in this world as supreme….The Father, the Son, and Lucifer have been revealed in their true relation to one another. God has given unmistakable evidence of His justice and His love.—Reflecting Christ, p. 58.

We see [in God’s Word] the great plan of human redemption, the means devised to free mankind from the power of Satan. We see Christ, the Captain of our salvation, meeting the prince of darkness in open battle, and single-handed, obtaining the victory in our behalf. We learn too that by this victory was opened to us a door of hope, a source of strength, and that we may, as faithful soldiers, fight our own battles with the wily foe, and conquer in the name of Jesus. The powers of darkness must be met by every soul. The young as well as the old will be assailed, and all should understand the nature of the great controversy between Christ and Satan, and should realize that it concerns themselves.

It is not enough to have an intellectual knowledge of the truth…. There must be an entrance of the Word into the heart. It must be set home by the power of the Holy Spirit. The will must be brought into harmony with its requirements. Not only the intellect but the heart and conscience must concur in the acceptance of the truth.—That I May Know Him, p. 192.

Satan is a vigilant, untiring foe, and he sleeps not. He knows that his time is short, and he will work until the end with every species of deception to draw souls into his snare and ruin them. I have a message for you—“Watch and pray, lest ye enter into temptation.” Give no place to the devil to stand between you and Christ, lest you savor of the things that be of men and not of God. If your faith is genuine, it must and will produce obedience. God commands us to do nothing which we cannot do. He will give strength to every believing, trusting soul.

Cherish the love of Jesus in the heart, respect each other, for Christ has given His life for you. Every soul is precious in the sight of God. It is a wonderful thing to be remembered and cared for every hour by God.—The Upward Look, p. 20.

\section*{Sunday – A Vision of the End}

Patmos, a barren, rocky island in the Aegean Sea, had been chosen by the Roman government as a place of banishment for criminals; but to the servant of God this gloomy abode became the gate of heaven. Here, shut away from the busy scenes of life, and from the active labors of former years, he had the companionship of God and Christ and the heavenly angels, and from them he received instruction for the church for all future time. …

Among the cliffs and rocks of Patmos, John held communion with his Maker. He reviewed his past life, and at thought of the blessings he had received, peace filled his heart. He had lived the life of a Christian, and he could say in faith, “We know that we have passed from death unto life.” 1 John 3:14. Not so the emperor who had banished him. He could look back only on fields of warfare and carnage, on desolated homes, on weeping widows and orphans, the fruit of his ambitious desire for pre-eminence.—The Acts of the Apostles, pp. 570, 571.

John [remembers] the wonderful incidents that he has witnessed in the life of Christ. In imagination he again enjoys the precious opportunities with which he was once favored, and is greatly comforted. Suddenly his meditation is broken in upon; he is addressed in tones distinct and clear. He turns to see from whence the voice proceeds, and, lo! he beholds his Lord, whom he has loved. … no longer “a man of sorrows, and acquainted with grief” (Isaiah 53:3).…

John, who has so loved his Lord, and who has steadfastly adhered to the truth in the face of imprisonment, stripes, and threatened death, cannot endure the excellent glory of Christ’s presence, and falls to the earth as one stricken dead. Jesus then lays His hand upon the prostrate form of His servant, saying, “Fear not; … I am he that liveth, and was dead; and, behold, I am alive for evermore” (Revelation 1:17, 18). John was strengthened to live in the presence of his glorified Lord, and then were presented before him in holy vision the purposes of God for future ages.—The Sanctified Life, pp. 77, 78.

“Jesus came and spake unto them, saying, … lo, I am with you alway, even unto the end of the world.” Here is our power, our comfort. Of ourselves we have no strength. But He says, “I am with you alway,” helping you to perform your duty, guiding, comforting, sanctifying, and sustaining you, giving you success in speaking words that will draw the attention of others to Christ and awaken in their minds the desire to understand the hope and meaning of the truth, turning them from darkness to light. …

… The lapse of time has wrought no change in His parting promise. He is with us today as truly as He was with the disciples, and He will be with us “even unto the end.”—In Heavenly Places, p. 188.

\section*{Monday – The Countdown}

The future was mercifully veiled from the disciples. Had they at that time fully comprehended the two awful facts—the Redeemer’s sufferings and death, and the destruction of their city and temple—they would have been overwhelmed with horror. Christ presented before them an outline of the prominent events to take place before the close of time. His words were not then fully understood; but their meaning was to be unfolded as His people should need the instruction therein given. The prophecy which He uttered was twofold in its meaning; while foreshadowing the destruction of Jerusalem, it prefigured also the terrors of the last great day.—The Great Controversy, p. 25.

Turning to the disciples, Christ said, “Take heed that no man deceive you. For many shall come in My name, saying, I am Christ; and shall deceive many.” Many false messiahs will appear, claiming to work miracles, and declaring that the time of the deliverance of the Jewish nation has come. These will mislead many. Christ’s words were fulfilled. Between His death and the siege of Jerusalem many false messiahs appeared. But this warning was given also to those who live in this age of the world. The same deceptions practiced prior to the destruction of Jerusalem have been practiced through the ages, and will be practiced again.—The Desire of Ages, p. 628.

It is a fatal mistake to suppose that the work of soul-saving depends alone upon the ministry. The humble, consecrated believer upon whom the Master of the vineyard places a burden for souls is to be given encouragement by the men upon whom the Lord has laid larger responsibilities. Those who stand as leaders in the church of God are to realize that the Saviour’s commission is given to all who believe in His name. God will send forth into His vineyard many who have not been dedicated to the ministry by the laying on of hands.

Hundreds, yea, thousands, who have heard the message of salvation are still idlers in the market place, when they might be engaged in some line of active service. To these Christ is saying, “Why stand ye here all the day idle?” and He adds, “Go ye also into the vineyard.” Matthew 20:6, 7. Why is it that many more do not respond to the call? Is it because they think themselves excused in that they do not stand in the pulpit? Let them understand that there is a large work to be done outside the pulpit by thousands of consecrated lay members.

Long has God waited for the spirit of service to take possession of the whole church so that everyone shall be working for Him according to his ability. When the members of the church of God do their appointed work in the needy fields at home and abroad, in fulfillment of the gospel commission, the whole world will soon be warned and the Lord Jesus will return to this earth with power and great glory. “This gospel of the kingdom shall be preached in all the world for a witness unto all nations; and then shall the end come.” Matthew 24:14.—The Acts of the Apostles, pp. 110, 111.

\section*{Tuesday – Marching Orders}

In God’s great plan for the redemption of a lost race, He has placed Himself under the necessity of using human agencies as His helping hand. He must have a helping hand, in order to reach humanity. He must have the cooperation of those who will be active, quick to see opportunities, quick to discern what must be done for their fellow men. …

All around us are afflicted souls. Let us search out these suffering ones, and speak a word in season to comfort their hearts. Here and there—everywhere—we shall find them. Let us ever be channels through which may flow to them the refreshing waters of compassion. …

Many are in obscurity. They have lost their bearings. They know not what course to pursue. Let the perplexed ones search out others who are in perplexity, and speak to them words of hope and encouragement. When they begin to do this work, the light of heaven will reveal to them the path that they should follow. By their words of consolation to the afflicted they themselves will be consoled. By helping others, they themselves will be helped out of their difficulties. Joy takes the place of sadness and gloom. The heart, filled with the Spirit of God, glows with warmth toward every fellow being.—Ellen G. White Comments, in The SDA Bible Commentary, vol. 4, p. 1151.

The soul-saving message, the third angel’s message, is the message to be given to the world. The commandments of God and the faith of Jesus are both important, immensely important, and must be given with equal force and power. The first part of the message has been dwelt upon mostly, the last part casually. The faith of Jesus is not comprehended. …

Why are our lips so silent upon the subject of Christ’s righteousness and His love for the world? Why do we not give to the people that which will revive and quicken them into a new life? …

The character of Christ is an infinitely perfect character, and He must be lifted up, He must be brought prominently into view, for He is the power, the might, the sanctification and righteousness of all who believe in Him.—Reflecting Christ, p. 82.

John in the Revelation foretells the proclamation of the gospel message just before Christ’s second coming. He beholds an angel flying “in the midst of heaven, having the everlasting gospel to preach unto them that dwell on the earth, and to every nation, and kindred, and tongue, and people, saying with a loud voice, Fear God, and give glory to Him; for the hour of His judgment is come.” Revelation 14:6, 7.

In the prophecy this warning of the judgment, with its connected messages, is followed by the coming of the Son of man in the clouds of heaven. The proclamation of the judgment is an announcement of Christ’s second coming as at hand. And this proclamation is called the everlasting gospel. Thus the preaching of Christ’s second coming, the announcement of its nearness, is shown to be an essential part of the gospel message.—Christ’s Object Lessons, p. 227.

\section*{Wednesday – Rest in Peace}

“Our friend Lazarus sleepeth.” How touching [Jesus’] words! how full of sympathy! In the thought of the peril their Master was about to incur by going to Jerusalem, the disciples had almost forgotten the bereaved family at Bethany. But not so Christ. The disciples felt rebuked. They had been disappointed because Christ did not respond more promptly to the message. They had been tempted to think that He had not the tender love for Lazarus and his sisters that they had thought He had, or He would have hastened back with the messenger. But the words, “Our friend Lazarus sleepeth,” awakened right feelings in their minds. They were convinced that Christ had not forgotten His suffering friends.

“Then said His disciples, Lord, if he sleep, he shall do well. Howbeit Jesus spake of his death: but they thought that He had spoken of taking of rest in sleep.” Christ represents death as a sleep to His believing children. Their life is hid with Christ in God, and until the last trump shall sound those who die will sleep in Him.—The Desire of Ages, p. 527.

In consequence of Adam’s sin, death passed upon the whole human race. All alike go down into the grave. And through the provisions of the plan of salvation, all are to be brought forth from their graves. “There shall be a resurrection of the dead, both of the just and unjust;” “for as in Adam all die, even so in Christ shall all be made alive.” Acts 24:15; 1 Corinthians 15:22. But a distinction is made between the two classes that are brought forth. “All that are in the graves shall hear His voice, and shall come forth; they that have done good, unto the resurrection of life; and they that have done evil, unto the resurrection of damnation.” John 5:28, 29. They who have been “accounted worthy” of the resurrection of life are “blessed and holy.” “On such the second death hath no power.” Revelation 20:6.—The Great Controversy, p. 544.

Christ is coming with clouds and with great glory. A multitude of shining angels will attend Him. He will come to raise the dead, and to change the living saints from glory to glory. He will come to honor those who have loved Him, and kept His commandments, and to take them to Himself. He has not forgotten them nor His promise. There will be a relinking of the family chain. When we look upon our dead, we may think of the morning when the trump of God shall sound, when “the dead shall be raised incorruptible, and we shall be changed.” 1 Corinthians 15:52. A little longer, and we shall see the King in His beauty. A little longer, and He will wipe all tears from our eyes. A little longer, and He will present us “faultless before the presence of His glory with exceeding joy.” Jude 1:24. Wherefore, when He gave the signs of His coming He said, “When these things begin to come to pass, then look up, and lift up your heads; for your redemption draweth nigh.”—The Desire of Ages, p. 632.

\section*{Thursday – Rejoice in the Lord Always}

It may seem difficult to rejoice in the Lord when in trouble, but we lose a great deal by giving way to a spirit of complaint. It is our privilege to have in our hearts, at all times, the peace of Christ. We should not allow ourselves to be easily disturbed. It is to test us that God brings us through trials and difficulties, and if we are patient and trustful under His proving, He will purify us from all dross, and at last bring us forth with triumph and rejoicing. Great blessings are reserved for those who uncomplainingly submit to the yoke that God wishes them to bear. …

“In everything give thanks” (1 Thessalonians 5:18) for the keeping power of God through Jesus Christ. … At the moment when you are offering your prayer for help you may not feel all the joy and blessing that you would like to feel, but if you believe that Christ will hear and answer your petition, the peace of Christ will come.—Our High Calling, p. 326.

My head is weary this morning. Mist and clouds hang over my mind; but the suggestions of the enemy to distrust the Lord shall not be cherished. Now is my time to fight the good fight of faith. Now is the very occasion that needs the steady faith that works by love and purifies my soul. I seek the Lord more earnestly.

In 1 Chronicles 28:9 David gives his charge to Solomon.

The message was brought to Asa by the Lord’s prophet: “The Lord is with you, while ye be with him; and if ye seek him, he will be found of you; but if ye forsake him, he will forsake you” (2 Chronicles 15:2; cf. Jeremiah 29:11-13). My heart goes out in faith. Faith is not feeling; faith is not sight. …

… I believe the promise is for me, and I appropriate the same personally. The promise itself is of no value unless I fully believe that He that has made the promise is abundantly able to fulfill, and infinite in power to do all that He has said.—This Day With God, p. 156.

O how privileged we are that we may come to Jesus just as we are and cast ourselves upon His love! We have no hope but in Jesus. He alone can reach us with His hand to lift us up out of the depths of discouragement and hopelessness and place our feet upon the Rock. Although the human soul may cling to Jesus with all the desperate sense of his great need, Jesus will cling to the souls bought by His own blood with a firmer grasp than the sinner clings to Him. …

What a Saviour we have—a risen Saviour, One who can save all who come unto Him!—That I May Know Him, p. 80.

\section*{Friday – Further Thought}

\setlength{\parindent}{0pt}This Day With God, “Complete Commitment,” p. 128;

The Upward Look, “Search the Scriptures,” p. 368.

\end{multicols}

\end{document}

