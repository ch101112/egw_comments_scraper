\documentclass[a4paper, 10pt, twoside, headings=small]{scrartcl}

%%%%%%%%%%%%%%%%%%%% General %%%%%%%%%%%%%%%%%%%%%%%%%%%%%%%%%%%%%%%%%%%%%%%%%%%

\usepackage[utf8]{inputenc}
\usepackage[T1]{fontenc}
%\usepackage[protrusion=true, expansion]{microtype} 

\usepackage[hidelinks]{hyperref} % Must go after the patches


%%%%%%%%%%%%%%%%%%%% Fonts %%%%%%%%%%%%%%%%%%%%%%%%%%%%%%%%%%%%%%%%%%%%%%%%%%%%%

\usepackage{fontspec}

\usepackage{polyglossia}


\addtokomafont{disposition}{\rmfamily}


\usepackage{textcomp}

\linespread{1.1}  


\usepackage[final]{microtype}
\setmainfont[Ligatures=TeX]{XCharter}


%%%%%%%%%%%%%%%%%%%% Random Packages  %%%%%%%%%%%%%%%%%%%%%%%%%%%%%%%%%%%%%%%

\usepackage{geometry}	
\usepackage{enumerate} %Erweiterung der enumerate-Umgebung
\usepackage{ifthen,calc}
\usepackage{mathrsfs,amssymb} %Zusatzzeichen
\usepackage{wrapfig}
\usepackage[retainorgcmds]{IEEEtrantools} %Besonders geeignet für einen mehrzeilige Formelsatz
\usepackage{theorem} %Theoremlayout
\usepackage{multicol}
\setlength\columnsep{20pt}
\usepackage{csquotes}





%%%%%%%%%%%%%%%%%%%% Page %%%%%%%%%%%%%%%%%%%%%%%%%%%%%%%%%%%%%%%%%%%%%%%%%%%

% twoside
\geometry{left=2.7cm, right=2.3cm, top=2.7cm, bottom=2.2cm}


%%%%%%%%%%%% PDF %%%%%%%%%%%%%%%%%%%%%%%%%%%%%%%%%%%%%%%%%%%%%%%%%%%%%%%%%%%%

\hypersetup{
	hidelinks=true,
	%	linkcolor=black,
	%	filecolor=black,      
	%	urlcolor=black,
	%	citecolor=black,
	%	allcolors=black,
	%	allbordercolors=white,
	%pdfpagemode=FullScreen,
	%	pdftitle={\mytitle},
	%	pdfauthor={\myauthor},
	pdfkeywords={},
	%	pdfcreator={Some fancy PDF-Creator...},
	bookmarksnumbered=true
}



%%%%%%%%%%%% footnotes %%%%%%%%%%%%%%%%%%%%%%%%%%%%%%%%%%%%%%%%%%%%%%%%%%%%%%

\usepackage[flushmargin, hang]{footmisc} % flush footnote mark to left margin
\usepackage{regexpatch}
\makeatletter
% 1. remove all redefinitions about footnotes done by \maketitle
%    and add \titletrue
\regexpatchcmd{\maketitle}
{\c{def}\c{@makefnmark}.*\c{if@twocolumn}}
{\c{titletrue}\c{if@twocolumn}}
{}{}
% 2. define a conditional
\newif\iftitle
%% 3. redefine \@makefnmark to print nothing when \titletrue
%\xpretocmd{\@makefnmark}{\iftitle\else}{}{}
%\xapptocmd{\@makefnmark}{\fi}{}{}
% 4. ensure \@makefntext has \titlefalse
%    that's justified by the fact that \@makefnmark
%    in \@makefntext is set in a box
\xpretocmd{\@makefntext}{\titlefalse}{}{}

\makeatother

\renewcommand{\footnotemargin}{1em}
%\addtolength{\footnotesep}{5mm}
\skip\footins=2\bigskipamount     % Determine the space above the rule
\renewcommand*{\footnoterule}{%
	\kern-3pt%
	\hrule width 1in%
	\kern 2.6pt%
	\vspace{\smallskipamount}       % The additional space below the rule
}



%%%%%%%%%%%% captions %%%%%%%%%%%%%%%%%%%%%%%%%%%%%%%%%%%%%%%%%%%%%%%%%%%%%%%

\usepackage[textfont={small},labelfont={small, bf}]{caption}
\DeclareCaptionFont{black}{ \color{white} }
\DeclareCaptionFormat{listing}{
	\colorbox[cmyk]{0.43, 0.35, 0.35,0.01 }{
		\parbox{\textwidth}{\hspace{15pt}#1#2#3}
	}
}
\captionsetup{format=plain, singlelinecheck=true}
\captionsetup[lstlisting]{labelfont={small, bf}, textfont={small}}


%%%%%%%%%%%% Title %%%%%%%%%%%%%%%%%%%%%%%%%%%%%%%%%%%%%%%%%%%%%%%%%%%%%%%%%%

\usepackage{titling}
\setlength{\droptitle}{-5em}
\pretitle{\begin{center}\LARGE\bfseries}
	\posttitle{\par\end{center}}
\preauthor{\begin{center}}
	\postauthor{\par\end{center}}
\predate{\begin{center}}
	\postdate{\par\end{center}}


%%%%%%%%%%%% footer / header %%%%%%%%%%%%%%%%%%%%%%%%%%%%%%%%%%%%%%%%%%%%%%%%

\usepackage{fancyhdr}

% twoside with subsection
\fancyhf{}
\fancyhead[RE]{\small\nouppercase\leftmark}
\fancyhead[LO]{\small\rightmark}
\fancyhead[LE,RO]{\thepage}
\renewcommand{\headrulewidth}{0pt}


% Does not really work...
%\setotherlanguage{greek}
%\setotherlanguage{hebrew}
%\newfontfamily\greekfont[]{Linux Libertine O}
%\newfontfamily\hebrewfont[]{Linux Libertine O}

\newcommand{\bm}{\vectorbold*} % using physics package

\newcommand{\matlab}{\textsc{Matlab}\textsuperscript{\tiny{\textregistered}}}



\setmainlanguage[]{english}

\title{10 Sabbath Rest}

\author{Ellen G.\ White}

\date{2021/03 Rest in Christ}

\begin{document}

\maketitle

\thispagestyle{empty}

\pagestyle{fancy}

\begin{multicols}{2}

\section*{Saturday – Sabbath Rest}

God gave to men the memorial of His creative power, that they might discern Him in the works of His hand. The Sabbath bids us behold in His created works the glory of the Creator. And it was because He desired us to do this that Jesus bound up His precious lessons with the beauty of natural things. On the holy rest day, above all other days, we should study the messages that God has written for us in nature. We should study the Saviour’s parables where He spoke them, in the fields and groves, under the open sky, among the grass and flowers. As we come close to the heart of nature, Christ makes His presence real to us, and speaks to our hearts of His peace and love.—Christ’s Object Lessons, p. 25.

God said, “The seventh day is the sabbath of the Lord thy God.” He placed His sanctity upon this day and blessed it and hallowed it as a day of rest. It is the only commandment in the whole Decalogue that tells who God is. It places God in distinction with every other god. It says the God that made the heaven and the earth, the God that made the trees and the flowers and that created man; this is the God that you are to keep before your children, and you have only to point to the flowers and tell them that He made these and that He rested on the seventh day from all His labors. The seventh day is a God-given memorial.

Pointing to God as the maker of the heavens and the earth, it distinguishes the true God from all false gods. All who keep the seventh day, signify by this act that they are worshippers of Jehovah. Thus the Sabbath is the sign of man’s allegiance to God as long as there are any upon the earth to serve Him.—Sons and Daughters of God, p. 59.

The law of God was not given to the Jews alone. It is of world-wide and perpetual obligation. “He that offendeth in one point is guilty of all.” Its ten precepts are like a chain of ten links. If one link is broken, the chain becomes worthless. Not a single precept can be revoked or changed to save the transgressor. While families and nations exist; while property, life, and character must be guarded; while good and evil are antagonistic, and a blessing or a curse must follow the acts of men—so long must the divine law control us. When God no longer requires men to love Him supremely, to reverence His name, and to keep holy the Sabbath; when He permits them to disregard the rights of their fellow-men, to hate and injure one another—then, and not till then, will the moral law lose its force.—The Signs of the Times, January 19, 1882.

\section*{Sunday – Sabbath and Creation}

The Sabbath calls our thoughts to nature, and brings us into communion with the Creator. In the song of the bird, the sighing of the trees, and the music of the sea, we still may hear His voice who talked with Adam in Eden in the cool of the day. And as we behold His power in nature we find comfort, for the word that created all things is that which speaks life to the soul. He “who commanded the light to shine out of darkness, hath shined in our hearts, to give the light of the knowledge of the glory of God in the face of Jesus Christ.” 2 Corinthians 4:6.—The Desire of Ages, p. 281.

God created man in His own image. Here is no mystery. There is no ground for the supposition that man was evolved by slow degrees of development from the lower forms of animal or vegetable life. Such teaching lowers the great work of the Creator to the level of man’s narrow, earthly conceptions. Men are so intent upon excluding God from the sovereignty of the universe that they degrade man and defraud him of the dignity of his origin. He who set the starry worlds on high and tinted with delicate skill the flowers of the field, who filled the earth and the heavens with the wonders of His power, when He came to crown His glorious work, to place one in the midst to stand as ruler of the fair earth, did not fail to create a being worthy of the hand that gave him life.

The genealogy of our race, as given by inspiration, traces back its origin, not to a line of developing germs, mollusks, and quadrupeds, but to the great Creator. Though formed from the dust, Adam was “the son of God.” He was placed, as God’s representative, over the lower orders of being. They cannot understand or acknowledge the sovereignty of God, yet they were made capable of loving and serving man. The psalmist says, “Thou madest him to have dominion over the works of Thy hands; Thou hast put all things under his feet: … the beasts of the field; the fowl of the air, … and whatsoever passeth through the paths of the seas.” Psalm 8:6-8.—Patriarchs and Prophets, pp. 44, 45.

At his creation Adam was placed in dominion over the earth. But by yielding to temptation, he was brought under the power of Satan. “Of whom a man is overcome, of the same is he brought in bondage.” 2 Peter 2:19. When man became Satan’s captive, the dominion which he held, passed to his conqueror. Thus Satan became “the god of this world.” 2 Corinthians 4:4. He had usurped that dominion over the earth which had been originally given to Adam. But Christ, by His sacrifice paying the penalty of sin, would not only redeem man, but recover the dominion which he had forfeited. All that was lost by the first Adam will be restored by the second.—Patriarchs and Prophets, p. 67.

\section*{Monday – Celebrating Freedom}

All men are of one family by creation, and all are one through redemption. Christ came to demolish every wall of partition, to throw open every compartment of the temple courts, that every soul may have free access to God. His love is so broad, so deep, so full, that it penetrates everywhere. It lifts out of Satan’s influence those who have been deluded by his deceptions, and places them within reach of the throne of God, the throne encircled by the rainbow of promise. In Christ there is neither Jew nor Greek, bond nor free.—Prophets and Kings, p. 369.

No other institution which was committed to the Jews tended so fully to distinguish them from surrounding nations as did the Sabbath. God designed that its observance should designate them as His worshipers. It was to be a token of their separation from idolatry, and their connection with the true God. … When the command was given to Israel, “Remember the Sabbath day, to keep it holy,” the Lord said also to them, “Ye shall be holy men unto Me.” Exodus 20:8; 22:31. Only thus could the Sabbath distinguish Israel as the worshipers of God.—The Desire of Ages, p. 283.

After the war in the heavenly courts Satan and his followers were cast out. As human beings, we are subject to the crafty wiles and temptations of this fallen foe. And unless we are kept by the power of Christ, we shall certainly be led away by the satanic sophistries by which the world is flooded. Our safety is to lean not on human power, on the arm of flesh, but upon the divine arm. Those who are partakers of the divine nature will not be beguiled by Satan.

Everyone will be tested. … We are God’s property. In Jesus Christ we are to behold a pattern of what we should be. Every soul should be educated to look not to his fellow men, but unto Christ. He is the Author and the Finisher of our Faith.—The Upward Look, p. 149.

To be sanctified is to become a partaker of the divine nature, catching the spirit and mind of Jesus, ever learning in the school of Christ. … [But] it is impossible for any of us by our own power or our own efforts to work this change in ourselves. It is the Holy Spirit, the Comforter, which Jesus said He would send into the world, that changes our character into the image of Christ; and when this is accomplished, we reflect, as in a mirror, the glory of the Lord. That is, the character of the one who thus beholds Christ is so like His, that one looking at him sees Christ’s own character shining out as from a mirror. Imperceptibly to ourselves, we are changed day by day from our ways and will into the ways and will of Christ, into the loveliness of His character.—Reflecting Christ, p. 20.

\section*{Tuesday – The Stranger in Your Gates}

It was in order that the Israelites might be a blessing to the nations, and that God’s name might be made known “throughout all the earth” (Exodus 9:16), that they were delivered from Egyptian bondage. If obedient to His requirements, they were to be placed far in advance of other peoples in wisdom and understanding; but this supremacy was to be reached and maintained only in order that through them the purpose of God for “all nations of the earth” might be fulfilled.

The marvelous providences connected with Israel’s deliverance from Egyptian bondage and with their occupancy of the Promised Land led many of the heathen to recognize the God of Israel as the Supreme Ruler. “The Egyptians shall know,” had been the promise, “that I am the Lord, when I stretch forth Mine hand upon Egypt, and bring out the children of Israel from among them.” Exodus 7:5. Even proud Pharaoh was constrained to acknowledge Jehovah’s power. “Go, serve the Lord,” he urged Moses and Aaron, “and bless me also.” Exodus 12:31, 32.—Prophets and Kings, pp. 368, 369.

The Israelites had lately been servants themselves, and now that they were to have servants under them, they were to beware of indulging the spirit of cruelty and exaction from which they had suffered under their Egyptian taskmasters. The memory of their own bitter servitude should enable them to put themselves in the servant’s place, leading them to be kind and compassionate, to deal with others as they would wish to be dealt with.

The rights of widows and orphans were especially guarded, and a tender regard for their helpless condition was enjoined. “If thou afflict them in any wise,” the Lord declared, “and they cry at all unto Me, I will surely hear their cry; and My wrath shall wax hot, and I will kill you with the sword; and your wives shall be widows, and your children fatherless.” Aliens who united themselves with Israel were to be protected from wrong or oppression. “Thou shalt not oppress a stranger: for ye know the heart of a stranger, seeing ye were strangers in the land of Egypt.”—Patriarchs and Prophets, p. 310.

Great blessings are promised to those who place a high estimate upon the Sabbath and realize the obligations resting upon them in regard to its observance: “If thou turn away thy foot from the Sabbath [from trampling upon it, setting it at nought], from doing thy pleasure on My holy day; and call the Sabbath a delight, the holy of the Lord, honorable; and shalt honor Him, not doing thine own ways, nor finding thine own pleasure, nor speaking thine own words: then shalt thou delight thyself in the Lord; and I will cause thee to ride upon the high places of the earth, and feed thee with the heritage of Jacob thy father: for the mouth of the Lord hath spoken it.”—Testimonies for the Church, vol. 2, p. 701.

\section*{Wednesday – Serving Others Honors God’s Sabbath}

Jesus stated … that the work of relieving the afflicted was in harmony with the Sabbath law. It was in harmony with the work of God’s angels, who are ever descending and ascending between heaven and earth to minister to suffering humanity. Jesus declared, “My Father worketh hitherto, and I work.” All days are God’s, in which to carry out His plans for the human race. …

Should God forbid the sun to perform its office upon the Sabbath, cut off its genial rays from warming the earth and nourishing vegetation? Must the system of worlds stand still through that holy day? …

Nature must continue her unvarying course. God could not for a moment stay His hand, or man would faint and die. And man also has a work to perform on this day. The necessities of life must be attended to, the sick must be cared for, the wants of the needy must be supplied. He will not be held guiltless who neglects to relieve suffering on the Sabbath. God’s holy rest day was made for man, and acts of mercy are in perfect harmony with its intent. God does not desire His creatures to suffer an hour’s pain that may be relieved upon the Sabbath or any other day. —The Desire of Ages, pp. 206, 207.

According to the fourth commandment the Sabbath was dedicated to rest and religious worship. All secular employment was to be suspended, but works of mercy and benevolence were in accordance with the purpose of the Lord. They were not to be limited by time or place. To relieve the afflicted, to comfort the sorrowing, is a labor of love that does honor to God’s holy day.

The necessities of life must be attended to, the sick must be cared for, the wants of the needy must be supplied. He will not be held guiltless who neglects to relieve suffering on the Sabbath. God’s holy rest day was made for man, and acts of mercy are in perfect harmony with its intent. God does not desire His creatures to suffer an hour’s pain that may be relieved upon the Sabbath or any other day….

Labor to relieve the suffering was pronounced by our Saviour a work of mercy and no violation of the Sabbath. The needs of suffering humanity are never to be neglected. The Saviour, by His example, has shown us that it is right to relieve suffering on the Sabbath.—My Life Today, p. 231.

Had Israel been true to her trust, all the nations of earth would have shared in her blessings. But the hearts of those to whom had been entrusted a knowledge of saving truth, were untouched by the needs of those around them. As God’s purpose was lost sight of, the heathen came to be looked upon as beyond the pale of His mercy. The light of truth was withheld, and darkness prevailed. The nations were overspread with a veil of ignorance; the love of God was little known; error and superstition flourished.—Prophets and Kings, p. 179.

\section*{Thursday – The Sign That We Belong to God}

One writer has likened the attempt to change the law of God to an ancient mischievous practice of turning in a wrong direction a signpost erected at an important junction where two roads met. The perplexity and hardship which this practice often caused was great.

A signpost was erected by God for those journeying through this world. One arm of this signpost pointed out willing obedience to the Creator as the road to felicity and life, while the other arm indicated disobedience as the path to misery and death. The way to happiness was as clearly defined as was the way to the city of refuge under the Jewish dispensation. But in an evil hour for our race, the great enemy of all good turned the signpost around, and multitudes have mistaken the way.

Through Moses the Lord instructed the Israelites: “Verily My Sabbaths ye shall keep: for it is a sign between Me and you throughout your generations; that ye may know that I am the Lord that doth sanctify you. … It is a sign between Me and the children of Israel forever: for in six days the Lord made heaven and earth, and on the seventh day He rested, and was refreshed.” Exodus 31:13, 17.

In these words the Lord clearly defined obedience as the way to the City of God; but the man of sin has changed the signpost, making it point in the wrong direction. He has set up a false sabbath and has caused men and women to think that by resting on it they were obeying the command of the Creator.—Prophets and Kings, pp. 179, 180.

The Sabbath is a sign of the relationship existing between God and His people, a sign that they are His obedient subjects, that they keep holy His law. The observance of the Sabbath is the means ordained by God of preserving a knowledge of Himself and of distinguishing between His loyal subjects and the transgressors of His law. This is the faith once delivered to the saints, who stand in moral power before the world, firmly maintaining this faith.—Testimonies for the Church, vol. 8, p. 198.

When the Lord delivered His people Israel from Egypt and committed to them His law, He taught them that by the observance of the Sabbath they were to be distinguished from idolaters. It was this that made the distinction between those who acknowledge the sovereignty of God and those who refuse to accept Him as their Creator and King. …

As the Sabbath was the sign that distinguished Israel when they came out of Egypt to enter the earthly Canaan, so it is the sign that now distinguishes God’s people as they come out from the world to enter the heavenly rest. The Sabbath is a sign of the relationship existing between God and His people, a sign that they honor His law. It distinguishes between His loyal subjects and transgressors.—Testimonies for the Church, vol. 6, p. 349.

\section*{Friday – Further Thought}

\setlength{\parindent}{0pt}The Desire of Ages, “The Sabbath,” pp. 281–289;

The Sanctified Life, “No Sanctification Without Obedience,” pp. 66, 67.

\end{multicols}

\end{document}

