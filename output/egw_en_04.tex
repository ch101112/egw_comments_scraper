\documentclass[a4paper, 10pt, twoside, headings=small]{scrartcl}

%%%%%%%%%%%%%%%%%%%% General %%%%%%%%%%%%%%%%%%%%%%%%%%%%%%%%%%%%%%%%%%%%%%%%%%%

\usepackage[utf8]{inputenc}
\usepackage[T1]{fontenc}
\usepackage[protrusion=true, expansion]{microtype} 

\usepackage[hidelinks]{hyperref} % Must go after the patches


%%%%%%%%%%%%%%%%%%%% Fonts %%%%%%%%%%%%%%%%%%%%%%%%%%%%%%%%%%%%%%%%%%%%%%%%%%%%%


\usepackage[english]{babel}



\usepackage{Charter}
\addtokomafont{disposition}{\rmfamily}


\usepackage{textcomp}

\linespread{1.1}  


%%%%%%%%%%%%%%%%%%%% Random Packages  %%%%%%%%%%%%%%%%%%%%%%%%%%%%%%%%%%%%%%%

\usepackage{geometry}	
\usepackage{enumerate} %Erweiterung der enumerate-Umgebung
\usepackage{ifthen,calc}
\usepackage{mathrsfs,amssymb} %Zusatzzeichen
\usepackage{wrapfig}
\usepackage[retainorgcmds]{IEEEtrantools} %Besonders geeignet für einen mehrzeilige Formelsatz
\usepackage{theorem} %Theoremlayout
\usepackage{multicol}
\setlength\columnsep{20pt}
\usepackage{csquotes}





%%%%%%%%%%%%%%%%%%%% Page %%%%%%%%%%%%%%%%%%%%%%%%%%%%%%%%%%%%%%%%%%%%%%%%%%%

% twoside
\geometry{left=2.7cm, right=2.3cm, top=2.7cm, bottom=2.2cm}


%%%%%%%%%%%% PDF %%%%%%%%%%%%%%%%%%%%%%%%%%%%%%%%%%%%%%%%%%%%%%%%%%%%%%%%%%%%

\hypersetup{
	hidelinks=true,
	%	linkcolor=black,
	%	filecolor=black,      
	%	urlcolor=black,
	%	citecolor=black,
	%	allcolors=black,
	%	allbordercolors=white,
	%pdfpagemode=FullScreen,
	%	pdftitle={\mytitle},
	%	pdfauthor={\myauthor},
	pdfkeywords={},
	%	pdfcreator={Some fancy PDF-Creator...},
	bookmarksnumbered=true
}



%%%%%%%%%%%% footnotes %%%%%%%%%%%%%%%%%%%%%%%%%%%%%%%%%%%%%%%%%%%%%%%%%%%%%%

\usepackage[flushmargin, hang]{footmisc} % flush footnote mark to left margin
\usepackage{regexpatch}
\makeatletter
% 1. remove all redefinitions about footnotes done by \maketitle
%    and add \titletrue
\regexpatchcmd{\maketitle}
{\c{def}\c{@makefnmark}.*\c{if@twocolumn}}
{\c{titletrue}\c{if@twocolumn}}
{}{}
% 2. define a conditional
\newif\iftitle
%% 3. redefine \@makefnmark to print nothing when \titletrue
%\xpretocmd{\@makefnmark}{\iftitle\else}{}{}
%\xapptocmd{\@makefnmark}{\fi}{}{}
% 4. ensure \@makefntext has \titlefalse
%    that's justified by the fact that \@makefnmark
%    in \@makefntext is set in a box
\xpretocmd{\@makefntext}{\titlefalse}{}{}

\makeatother

\renewcommand{\footnotemargin}{1em}
%\addtolength{\footnotesep}{5mm}
\skip\footins=2\bigskipamount     % Determine the space above the rule
\renewcommand*{\footnoterule}{%
	\kern-3pt%
	\hrule width 1in%
	\kern 2.6pt%
	\vspace{\smallskipamount}       % The additional space below the rule
}



%%%%%%%%%%%% captions %%%%%%%%%%%%%%%%%%%%%%%%%%%%%%%%%%%%%%%%%%%%%%%%%%%%%%%

\usepackage[textfont={small},labelfont={small, bf}]{caption}
\DeclareCaptionFont{black}{ \color{white} }
\DeclareCaptionFormat{listing}{
	\colorbox[cmyk]{0.43, 0.35, 0.35,0.01 }{
		\parbox{\textwidth}{\hspace{15pt}#1#2#3}
	}
}
\captionsetup{format=plain, singlelinecheck=true}
\captionsetup[lstlisting]{labelfont={small, bf}, textfont={small}}


%%%%%%%%%%%% Title %%%%%%%%%%%%%%%%%%%%%%%%%%%%%%%%%%%%%%%%%%%%%%%%%%%%%%%%%%

\usepackage{titling}
\setlength{\droptitle}{-5em}
\pretitle{\begin{center}\LARGE\normalfont\scshape}
	\posttitle{\par\end{center}}
\preauthor{\begin{center}}
	\postauthor{\par\end{center}}
\predate{\begin{center}}
	\postdate{\par\end{center}}


%%%%%%%%%%%% footer / header %%%%%%%%%%%%%%%%%%%%%%%%%%%%%%%%%%%%%%%%%%%%%%%%

\usepackage{fancyhdr}

% twoside with subsection
\fancyhf{}
\fancyhead[RE]{\small\nouppercase\leftmark}
\fancyhead[LO]{\small\rightmark}
\fancyhead[LE,RO]{\thepage}
\renewcommand{\headrulewidth}{0pt}


% Does not really work...
%\setotherlanguage{greek}
%\setotherlanguage{hebrew}
%\newfontfamily\greekfont[]{Linux Libertine O}
%\newfontfamily\hebrewfont[]{Linux Libertine O}

\newcommand{\bm}{\vectorbold*} % using physics package

\newcommand{\matlab}{\textsc{Matlab}\textsuperscript{\tiny{\textregistered}}}



\setmainlanguage[]{english}

\title{04 The Cost of Rest}

\author{Ellen G.\ White}

\date{2021/03 Rest in Christ}

\begin{document}

\maketitle

\thispagestyle{empty}

\pagestyle{fancy}

\begin{multicols}{2}

\section*{Saturday – The Cost of Rest}

It is impossible to estimate too largely the work that the Lord will accomplish through His proposed vessels in carrying out His mind and purpose. …

The Holy Spirit never reveals itself in … a bedlam of noise. This is an invention of Satan to cover up his ingenious methods for making of none effect the pure, sincere, elevating, ennobling, sanctifying truth for this time. … A bedlam of noise shocks the senses and perverts that which if conducted aright might be a blessing.—Selected Messages, book 2, p. 36.

Those who take Christ at His word, and surrender their souls to His keeping, their lives to His ordering, will find peace and quietude. Nothing of the world can make them sad when Jesus makes them glad by His presence. In perfect acquiescence there is perfect rest. The Lord says, “Thou wilt keep him in perfect peace, whose mind is stayed on Thee: because he trusteth in Thee.” Isaiah 26:3. Our lives may seem a tangle; but as we commit ourselves to the wise Master Worker, He will bring out the pattern of life and character that will be to His own glory. And that character which expresses the glory—character—of Christ will be received into the Paradise of God. A renovated race shall walk with Him in white, for they are worthy.—The Desire of Ages, p. 331.

[W]hen you are weary or perplexed … the enemy works to make you grumble and murmur, look unto Jesus, trust in your Saviour. This is the only cure .… If you allow your mind to be occupied with these things, the enemy will see that you are kept busy. He puts his magnifying glass before your eyes, and mole hills of difficulty are made to appear as mountains. You need to understand how to repose in God. A wise heart, molded by the Holy Spirit, it is your privilege to have; and this is the foundation of all true happiness.

God would have you trust in His love, and be constantly guarding your soul by locking the gate of your thoughts, that they shall not become unmanageable; for when you allow your mind to indulge these thoughts of self-pity, the enemy comes in to suggest the most unkind and unreasonable things in regard to those who would do you good, and only good.

Listen to Jesus, follow His counsel and you will not go astray from the wise and mighty Counsellor, the only true Guide, the only One who can give you peace, happiness, and fullness of joy. Whatever others may think of us or may do to us, it need not disturb this oneness with Christ, this fellowship of the Spirit. You know we cannot find rest anywhere out of Christ.—Sons and Daughters of God, p. 298.

\section*{Sunday – Worn and Weary}

[I]t is a perilous thing to praise or exalt men; for if one comes to lose sight of his entire dependence on God, and to trust to his own strength, he is sure to fall. Man is contending with foes who are stronger than he. … It is impossible for us in our own strength to maintain the conflict; and whatever diverts the mind from God, whatever leads to self-exaltation or to self-dependence, is surely preparing the way for our overthrow. …

It was the spirit of self-confidence and self-exaltation that prepared the way for David’s fall. Flattery and the subtle allurements of power and luxury were not without effect upon him. … All this tended to lessen David’s sense of the exceeding sinfulness of sin. And instead of relying in humility upon the power of Jehovah, he began to trust to his own wisdom and might. As soon as Satan can separate the soul from God, the only Source of strength, he will seek to arouse the unholy desires of man’s carnal nature. The work of the enemy is not abrupt; it is not, at the outset, sudden and startling; it is a secret undermining of the strongholds of principle. It begins in apparently small things—the neglect to be true to God and to rely upon Him wholly, the disposition to follow the customs and practices of the world.—Patriarchs and Prophets, p. 717.

… David yielded to Satan and brought upon his soul the stain of guilt. He, the Heaven-appointed leader of the nation, chosen by God to execute His law, himself trampled upon its precepts. …

[N]ow, guilty and unrepentant, he did not ask help and guidance from Heaven, but sought to extricate himself from the dangers in which sin had involved him. Bathsheba, whose fatal beauty had proved a snare to the king, was the wife of Uriah the Hittite, one of David’s bravest and most faithful officers. None could foresee what would be the result should the crime become known. The law of God pronounced the adulterer guilty of death, and the proud-spirited soldier, so shamefully wronged, might avenge himself by taking the life of the king or by exciting the nation to revolt.

Every effort which David made to conceal his guilt proved unavailing. He had betrayed himself into the power of Satan; danger surrounded him, dishonor more bitter than death was before him. There appeared but one way of escape, and in his desperation he was hurried on to add murder to adultery. He who had compassed the destruction of Saul was seeking to lead David also to ruin. Though the temptations were different, they were alike in leading to transgression of God’s law. David reasoned that if Uriah were slain by the hand of enemies in battle, the guilt of his death could not be traced home to the king, Bathsheba would be free to become David’s wife, suspicion could be averted, and the royal honor would be maintained.

\section*{Monday – Wake-Up Call}

Nathan the prophet was bidden to bear a message of reproof to David. It was a message terrible in its severity. To few sovereigns could such a reproof be given but at the price of certain death to the reprover. Nathan delivered the divine sentence unflinchingly, yet with such heaven-born wisdom as to engage the sympathies of the king, to arouse his conscience, and to call from his lips the sentence of death upon himself. Appealing to David as the divinely appointed guardian of his people’s rights, the prophet repeated a story of wrong and oppression that demanded redress. …

The prophet’s rebuke touched the heart of David; conscience was aroused; his guilt appeared in all its enormity. His soul was bowed in penitence before God. With trembling lips he said, “I have sinned against the Lord.” All wrong done to others reaches back from the injured one to God. David had committed a grievous sin, toward both Uriah and Bathsheba, and he keenly felt this. But infinitely greater was his sin against God.—Patriarchs and Prophets, pp. 720, 722.

It was when the Israelites were in a condition of outward ease and security that they were led into sin. They failed to keep God ever before them, they neglected prayer and cherished a spirit of self-confidence. Ease and self-indulgence left the citadel of the soul unguarded, and debasing thoughts found entrance. It was the traitors within the walls that overthrew the strongholds of principle and betrayed Israel into the power of Satan. It is thus that Satan still seeks to compass the ruin of the soul. A long preparatory process, unknown to the world, goes on in the heart before the Christian commits open sin. The mind does not come down at once from purity and holiness to depravity, corruption, and crime. It takes time to degrade those formed in the image of God to the brutal or the satanic. By beholding we become changed. By the indulgence of impure thoughts man can so educate his mind that sin which he once loathed will become pleasant to him.—Patriarchs and Prophets, p. 459.

[S]tore the mind with useful knowledge, committing to memory portions of Scripture, tracing out the fulfillment of the prophecies, and learning the lessons which Christ gave to His disciples. … In this way an effectual door will be closed against a thousand temptations. Had King David been engaged in some useful employment, he would not have been guilty of the murder of Uriah. Satan is ever ready to employ him who does not employ himself. The mind which is continually striving to rise to the height of intellectual greatness will find no time for cheap, foolish thoughts, which are the parent of evil actions.—Testimonies for the Church, vol. 4, p. 412.

\section*{Tuesday – Forgiven and Forgotten?}

This experience was most painful to David, but it was most beneficial. But for the mirror which Nathan held up before him, in which he so clearly recognized his own likeness, he would have gone on unconvicted of his heinous sin, and would have been ruined. The conviction of his guilt was the saving of his soul. He saw himself in another light, as the Lord saw him, and as long as he lived he repented of his sin. …

Though David repented of his sin, and was forgiven and accepted by the Lord, he reaped the baleful harvest of the seed he himself had sown. His authority in his own household, his claim to respect and obedience from his sons, was weakened. A sense of his guilt kept him silent when he should have condemned sin; it made his arm feeble to execute justice in his house.

Those who, by pointing to the example of David, try to lessen the guilt of their own sins, should learn from the Bible record that the way of transgression is hard. Though like David they should turn from their evil course, the results of sin, even in this life, will be found bitter and hard to bear.—Conflict and Courage, pp. 179, 180.

We should remember that all make mistakes; even men and women who have had years of experience sometimes err; but God does not cast them off because of their errors; to every erring son and daughter of Adam He gives the privilege of another trial.

Jesus loves to have us come to Him just as we are, sinful, helpless, dependent. We may come with all our weakness, our folly, our sinfulness, and fall at His feet in penitence. It is His glory to encircle us in the arms of His love, and to bind up our wounds, to cleanse us from all impurity. …

Put away the suspicion that God’s promises are not meant for you. They are for every repentant transgressor. Strength and grace have been provided through Christ to be brought by ministering angels to every believing soul. None are so sinful that they cannot find strength, purity, and righteousness in Jesus, who died for them. He is waiting to strip them of their garments stained and polluted with sin, and to put upon them the white robes of righteousness; He bids them live and not die.—The Faith I Live By, p. 134.

True confession is always of a specific character, and acknowledges particular sins. They may be of such a nature as only to be brought before God, they may be wrongs that should be confessed before individuals who have suffered injury through them, or they may be of a general kind that should be made known in the congregation of the people. But all confession should be definite and to the point, acknowledging the very sins of which you are guilty. …

Confession will not be acceptable to God without sincere repentance and reformation. There must be decided changes in the life; everything offensive to God must be put away. This will be the result of genuine sorrow for sin.—Testimonies for the Church, vol. 5, pp. 639, 640.

\section*{Wednesday – Something New}

The prayer of David after his fall, illustrates the nature of true sorrow for sin. His repentance was sincere and deep. There was no effort to palliate his guilt; no desire to escape the judgment threatened, inspired his prayer. David saw the enormity of his transgression; he saw the defilement of his soul; he loathed his sin. It was not for pardon only that he prayed, but for purity of heart. He longed for the joy of holiness—to be restored to harmony and communion with God. This was the language of his soul: … “Purge me with hyssop, and I shall be clean: wash me, and I shall be whiter than snow. … Create in me a clean heart, O God; And renew a right spirit within me” Psalm 51:7, 10.—Steps to Christ, pp. 24, 25.

God’s forgiveness is not merely a judicial act by which He sets us free from condemnation. It is not only forgiveness for sin, but reclaiming from sin. It is the outflow of redeeming love that transforms the heart. David [has] the true conception of forgiveness when he. … says, “As far as the east is from the west, so far hath He removed our transgressions from us.” Psalm 103:12. …

Let Christ, the divine Life, dwell in you and through you reveal the heaven-born love that will inspire hope in the hopeless and bring heaven’s peace to the sin-stricken heart. As we come to God, this is the condition which meets us at the threshold, that, receiving mercy from Him, we yield ourselves to reveal His grace to others.—Thoughts From the Mount of Blessing, p. 114.

Rest yourself wholly in the hands of Jesus. Contemplate His great love, and while you meditate upon His self-denial, His infinite sacrifice made in our behalf in order that we should believe in Him, your heart will be filled with holy joy, calm peace, and indescribable love. As we talk of Jesus, as we call upon Him in prayer, our confidence that He is our personal, loving Saviour will strengthen and His character will appear more and more lovely. … Wait upon the Lord in faith. The Lord draws out the soul in prayer, and gives us to feel His precious love. We have a nearness to Him, and can hold sweet communion with Him. We obtain distinct views of His tenderness and compassion, and our hearts are broken and melted with contemplation of the love that is given to us. …

… Our peace is like a river, wave after wave of glory rolls into the heart, and indeed we sup with Jesus and He with us. We have a realizing sense of the love of God, and we rest in His love. No language can describe it, it is beyond knowledge. We are one with Christ, our life is hid with Christ in God. We have the assurance that when He who is our life shall appear, then shall we also appear with Him in glory. With strong confidence, we can call God our Father.—Letter 52, 1894.

\section*{Thursday – Reflectors of God’s Light}

Thus in a sacred song, [Psalm 51], to be sung in the public assemblies of his people, in the presence of the court—priests and judges, princes and men of war—and which would preserve to the latest generation the knowledge of his fall, the king of Israel recounted his sin, his repentance, and his hope of pardon through the mercy of God. Instead of endeavoring to conceal his guilt he desired that others might be instructed by the sad history of his fall.—Patriarchs and Prophets, p. 725.

We are to come to God in faith, and pour out our supplications before Him, believing that He will work in our behalf, and in the behalf of those we are seeking to save. We are to devote more time to earnest prayer. With the trusting faith of a little child, we are to come to our heavenly Father, telling Him of all our needs. He is always ready to pardon and help. The supply of divine wisdom is inexhaustible, and the Lord encourages us to draw largely from it. The longing that we should have for spiritual blessings is described in the words, “As the hart panteth after the water brooks, so panteth my soul after thee, O God.” We need a deeper soul-hunger for the rich gifts that heaven has to bestow. We are to hunger and thirst after righteousness.

O that we might have a consuming desire to know God by an experimental knowledge, to come into the audience chamber of the Most High, reaching up the hand of faith, and casting our helpless souls upon the One mighty to save. His loving kindness is better than life.—Ellen G. White Comments, in The SDA Bible Commentary, vol. 3, pp. 1146, 1147.

Jesus, dwelling in you, desires to speak to the hearts of those who are not acquainted with Him. Perhaps they do not read the Bible, or do not hear the voice that speaks to them in its pages; they do not see the love of God through His works. But if you are a true representative of Jesus, it may be that through you they will be led to understand something of His goodness and be won to love and serve Him.

Christians are set as light bearers on the way to heaven. They are to reflect to the world the light shining upon them from Christ. Their life and character should be such that through them others will get a right conception of Christ and of His service.—Steps to Christ, p. 115.

[No] man can impart that which he himself has not received. In the work of God, humanity can originate nothing. No man can by his own effort make himself a light bearer for God. … It is the love of God continually transferred to man that enables him to impart light. Into the hearts of all who are united to God by faith the golden oil of love flows freely, to shine out again in good works, in real, heartfelt service for God.—Christ’s Object Lessons, p. 418.

\section*{Friday – Further Thought}

\setlength{\parindent}{0pt}Conflict and Courage, “One Sin Leads to Another,” p. 178;

Sons and Daughters of God, “We Learn of Christ,” p. 68.

\end{multicols}

\end{document}

