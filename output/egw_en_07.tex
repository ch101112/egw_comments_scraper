\documentclass[a4paper, 10pt, twoside, headings=small]{scrartcl}

%%%%%%%%%%%%%%%%%%%% General %%%%%%%%%%%%%%%%%%%%%%%%%%%%%%%%%%%%%%%%%%%%%%%%%%%

\usepackage[utf8]{inputenc}
\usepackage[T1]{fontenc}
\usepackage[protrusion=true, expansion]{microtype} 

\usepackage[hidelinks]{hyperref} % Must go after the patches


%%%%%%%%%%%%%%%%%%%% Fonts %%%%%%%%%%%%%%%%%%%%%%%%%%%%%%%%%%%%%%%%%%%%%%%%%%%%%


\usepackage[english]{babel}



\usepackage{Charter}
\addtokomafont{disposition}{\rmfamily}


\usepackage{textcomp}

\linespread{1.1}  


%%%%%%%%%%%%%%%%%%%% Random Packages  %%%%%%%%%%%%%%%%%%%%%%%%%%%%%%%%%%%%%%%

\usepackage{geometry}	
\usepackage{enumerate} %Erweiterung der enumerate-Umgebung
\usepackage{ifthen,calc}
\usepackage{mathrsfs,amssymb} %Zusatzzeichen
\usepackage{wrapfig}
\usepackage[retainorgcmds]{IEEEtrantools} %Besonders geeignet für einen mehrzeilige Formelsatz
\usepackage{theorem} %Theoremlayout
\usepackage{multicol}
\setlength\columnsep{20pt}
\usepackage{csquotes}





%%%%%%%%%%%%%%%%%%%% Page %%%%%%%%%%%%%%%%%%%%%%%%%%%%%%%%%%%%%%%%%%%%%%%%%%%

% twoside
\geometry{left=2.7cm, right=2.3cm, top=2.7cm, bottom=2.2cm}


%%%%%%%%%%%% PDF %%%%%%%%%%%%%%%%%%%%%%%%%%%%%%%%%%%%%%%%%%%%%%%%%%%%%%%%%%%%

\hypersetup{
	hidelinks=true,
	%	linkcolor=black,
	%	filecolor=black,      
	%	urlcolor=black,
	%	citecolor=black,
	%	allcolors=black,
	%	allbordercolors=white,
	%pdfpagemode=FullScreen,
	%	pdftitle={\mytitle},
	%	pdfauthor={\myauthor},
	pdfkeywords={},
	%	pdfcreator={Some fancy PDF-Creator...},
	bookmarksnumbered=true
}



%%%%%%%%%%%% footnotes %%%%%%%%%%%%%%%%%%%%%%%%%%%%%%%%%%%%%%%%%%%%%%%%%%%%%%

\usepackage[flushmargin, hang]{footmisc} % flush footnote mark to left margin
\usepackage{regexpatch}
\makeatletter
% 1. remove all redefinitions about footnotes done by \maketitle
%    and add \titletrue
\regexpatchcmd{\maketitle}
{\c{def}\c{@makefnmark}.*\c{if@twocolumn}}
{\c{titletrue}\c{if@twocolumn}}
{}{}
% 2. define a conditional
\newif\iftitle
%% 3. redefine \@makefnmark to print nothing when \titletrue
%\xpretocmd{\@makefnmark}{\iftitle\else}{}{}
%\xapptocmd{\@makefnmark}{\fi}{}{}
% 4. ensure \@makefntext has \titlefalse
%    that's justified by the fact that \@makefnmark
%    in \@makefntext is set in a box
\xpretocmd{\@makefntext}{\titlefalse}{}{}

\makeatother

\renewcommand{\footnotemargin}{1em}
%\addtolength{\footnotesep}{5mm}
\skip\footins=2\bigskipamount     % Determine the space above the rule
\renewcommand*{\footnoterule}{%
	\kern-3pt%
	\hrule width 1in%
	\kern 2.6pt%
	\vspace{\smallskipamount}       % The additional space below the rule
}



%%%%%%%%%%%% captions %%%%%%%%%%%%%%%%%%%%%%%%%%%%%%%%%%%%%%%%%%%%%%%%%%%%%%%

\usepackage[textfont={small},labelfont={small, bf}]{caption}
\DeclareCaptionFont{black}{ \color{white} }
\DeclareCaptionFormat{listing}{
	\colorbox[cmyk]{0.43, 0.35, 0.35,0.01 }{
		\parbox{\textwidth}{\hspace{15pt}#1#2#3}
	}
}
\captionsetup{format=plain, singlelinecheck=true}
\captionsetup[lstlisting]{labelfont={small, bf}, textfont={small}}


%%%%%%%%%%%% Title %%%%%%%%%%%%%%%%%%%%%%%%%%%%%%%%%%%%%%%%%%%%%%%%%%%%%%%%%%

\usepackage{titling}
\setlength{\droptitle}{-5em}
\pretitle{\begin{center}\LARGE\normalfont\scshape}
	\posttitle{\par\end{center}}
\preauthor{\begin{center}}
	\postauthor{\par\end{center}}
\predate{\begin{center}}
	\postdate{\par\end{center}}


%%%%%%%%%%%% footer / header %%%%%%%%%%%%%%%%%%%%%%%%%%%%%%%%%%%%%%%%%%%%%%%%

\usepackage{fancyhdr}

% twoside with subsection
\fancyhf{}
\fancyhead[RE]{\small\nouppercase\leftmark}
\fancyhead[LO]{\small\rightmark}
\fancyhead[LE,RO]{\thepage}
\renewcommand{\headrulewidth}{0pt}


% Does not really work...
%\setotherlanguage{greek}
%\setotherlanguage{hebrew}
%\newfontfamily\greekfont[]{Linux Libertine O}
%\newfontfamily\hebrewfont[]{Linux Libertine O}

\newcommand{\bm}{\vectorbold*} % using physics package

\newcommand{\matlab}{\textsc{Matlab}\textsuperscript{\tiny{\textregistered}}}



\setmainlanguage[]{english}

\title{07 Rest, Relationships, and Healing}

\author{Ellen G.\ White}

\date{2021/03 Rest in Christ}

\begin{document}

\maketitle

\thispagestyle{empty}

\pagestyle{fancy}

\begin{multicols}{2}

\section*{Saturday – Rest, Relationships, and Healing}

For those who are convicted of sin and weighed down with a sense of their unworthiness, there are lessons of faith and encouragement in this record. The Bible faithfully presents the result of Israel’s apostasy; but it portrays also the deep humiliation and repentance, the earnest devotion and generous sacrifice, that marked their seasons of return to the Lord.

Every true turning to the Lord brings abiding joy into the life. When a sinner yields to the influence of the Holy Spirit, he sees his own guilt and defilement in contrast with the holiness of the great Searcher of hearts. He sees himself condemned as a transgressor. But he is not, because of this, to give way to despair; for his pardon has already been secured. He may rejoice in the sense of sins forgiven, in the love of a pardoning heavenly Father. It is God’s glory to encircle sinful, repentant human beings in the arms of His love, to bind up their wounds, to cleanse them from sin, and to clothe them with the garments of salvation.—Prophets and Kings, pp. 667, 668.

The love of God is something more than a mere negation; it is a positive and active principle, a living spring, ever flowing to bless others. If the love of Christ dwells in us, we shall not only cherish no hatred toward our fellows, but we shall seek in every way to manifest love toward them.

When one who professes to serve God wrongs or injures a brother, he misrepresents the character of God to that brother, and the wrong must be confessed, he must acknowledge it to be sin, in order to be in harmony with God. Our brother may have done us a greater wrong than we have done him, but this does not lessen our responsibility. If when we come before God we remember that another has aught against us, we are to leave our gift of prayer, of thanksgiving, of freewill offering, and go to the brother with whom we are at variance, and in humility confess our own sin and ask to be forgiven.—Thoughts From the Mount of Blessing, p. 59.

If we have in any manner defrauded or injured our brother, we should make restitution. If we have unwittingly borne false witness, if we have misstated his words, if we have injured his influence in any way, we should go to the ones with whom we have conversed about him, and take back all our injurious misstatements.

If matters of difficulty between brethren were not laid open before others, but frankly spoken of between themselves in the spirit of Christian love, how much evil might be prevented! How many roots of bitterness whereby many are defiled would be destroyed, and how closely and tenderly might the followers of Christ be united in His love!

\section*{Sunday – Facing the Past}

Hearing of the abundant provision made by the king of Egypt, ten of Jacob’s sons journeyed thither to purchase grain. On their arrival they were directed to the king’s deputy, and with other applicants they came to present themselves before the ruler of the land. And they “bowed down themselves before him with their faces to the earth.” … As Joseph saw his brothers stooping and making obeisance, his dreams came to his mind, and the scenes of the past rose vividly before him. His keen eye, surveying the group, discovered that Benjamin was not among them. Had he also fallen a victim to the treacherous cruelty of those savage men? He determined to learn the truth. …

… He wished to learn if they possessed the same haughty spirit as when he was with them, and also to draw from them some information in regard to their home; yet he well knew how deceptive their statements might be.—Patriarchs and Prophets, pp. 224, 225.

The servants of Christ are not to act out the dictates of the natural heart. They need to have close communion with God, lest, under provocation, self rise up, and they pour forth a torrent of words that are unbefitting, that are not as dew or the still showers that refresh the withering plants. This is what Satan wants them to do; for these are his methods. It is the dragon that is wroth; it is the spirit of Satan that is revealed in anger and accusing. But God’s servants are to be representatives of Him. He desires them to deal only in the currency of heaven, the truth that bears His own image and superscription. The power by which they are to overcome evil is the power of Christ. The glory of Christ is their strength. They are to fix their eyes upon His loveliness. … And the spirit that is kept gentle under provocation will speak more effectively in favor of the truth than will any argument, however forcible.—The Desire of Ages, p. 353.

Christ is our only hope. We may look to Him, for He is our Saviour. We may take Him at His word, and make Him our dependence. He knows just the help we need, and we can safely put our trust in Him. If we depend on merely human wisdom to guide us, we shall find ourselves on the losing side. But we may come direct to the Lord Jesus, for He has said: “Come unto Me, all ye that labor and are heavy-laden, and I will give you rest. Take My yoke upon you, and learn of Me; for I am meek and lowly in heart: and ye shall find rest unto your souls.” It is our privilege to be taught of Him. …

We have a divine audience to which to present our requests. Then let nothing prevent us from offering our petitions in the name of Jesus, believing with unwavering faith that God hears us, and that He will answer us. Let us carry our difficulties to God, humbling ourselves before Him.—Testimonies to Ministers and Gospel Workers, pp. 486, 487.

\section*{Monday – Setting the Stage}

[Joseph’s] brothers stood motionless, dumb with fear and amazement. The ruler of Egypt their brother Joseph, whom they had envied and would have murdered, and finally sold as a slave! All their ill treatment of him passed before them. They remembered how they had despised his dreams and had labored to prevent their fulfillment. …

Seeing their confusion, he said kindly, “Come near to me, I pray you;” and as they came near, he continued, “I am Joseph your brother, whom ye sold into Egypt. Now therefore be not grieved, nor angry with yourselves, that ye sold me hither: for God did send me before you to preserve life.” Feeling that they had already suffered enough for their cruelty toward him, he nobly sought to banish their fears and lessen the bitterness of their self-reproach. …

… “And he fell upon his brother Benjamin’s neck, and wept; and Benjamin wept upon his neck. Moreover he kissed all his brethren, and wept upon them: and after that his brethren talked with him.” They humbly confessed their sin and entreated his forgiveness. They had long suffered anxiety and remorse, and now they rejoiced that he was still alive.—Patriarchs and Prophets, pp. 230, 231.

Although Joseph was exalted as a ruler over all the land, yet he did not forget God. He knew that he was a stranger in a strange land, separated from his father and his brethren, which often caused him sadness, but he firmly believed that God’s hand had overruled his course, to place him in an important position. And depending on God continually, he performed all the duties of his office, as ruler over the land of Egypt with faithfulness. …

[When his brothers] humbly confessed their wrongs which they had committed against Joseph, and entreated his forgiveness, [they] greatly rejoiced to find that he was alive; for they had suffered remorse, and great distress of mind, since their cruelty toward him. And now as they knew that they were not guilty of his blood, their troubled minds were relieved.

Joseph gladly forgave his brethren, and sent them away abundantly provided with provisions, and carriages, and everything necessary for the removal of their father’s family and their own to Egypt. Joseph gave his brother Benjamin more valuable presents than to his other brethren. As he sent them away he charged them, “See that ye fall not out by the way.” He was afraid that they might enter into a dispute, and charge upon one another the cause of their guilt in regard to their cruel treatment of himself. With joy they returned to their father.—Spiritual Gifts, vol. 3, pp. 152, 167.

Jesus knows the circumstances of every soul. You may say, I am sinful, very sinful. You may be; but the worse you are, the more you need Jesus. He turns no weeping, contrite one away. He does not tell to any all that He might reveal, but He bids every trembling soul take courage. Freely will He pardon all who come to Him for forgiveness and restoration.—The Desire of Ages, p. 568.

\section*{Tuesday – Forgive and Forget?}

Peter had come to Christ with the question, “How oft shall my brother sin against me, and I forgive him? till seven times?” … Christ taught that we are never to become weary of forgiving. Not “Until seven times,” He said, “but, Until seventy times seven.”

Then He showed the true ground upon which forgiveness is to be granted and the danger of cherishing an unforgiving spirit. In a parable He told of a king’s dealing with the officers who administered the affairs of his government. Some of these officers were in receipt of vast sums of money belonging to the state. As the king investigated their administration of this trust, there was brought before him one man whose account showed a debt to his Lord for the immense sum of ten thousand talents. He had nothing to pay, and according to the custom, the king ordered him to be sold, with all that he had, that payment might be made. But the terrified man fell at his feet and besought him, saying, “Have patience with me, and I will pay thee all. Then the Lord of that servant was moved with compassion, and loosed him, and forgave him the debt. …

The pardon granted by this king represents a divine forgiveness of all sin. Christ is represented by the king, who, moved with compassion, forgave the debt of his servant. Man was under the condemnation of the broken law. He could not save himself, and for this reason Christ came to this world, clothed His divinity with humanity, and gave His life, the just for the unjust. He gave Himself for our sins, and to every soul He freely offers the blood-bought pardon.—Christ’s Object Lessons, pp. 243, 244.

If your brethren err, you are to forgive them. When they come to you with confession, you should not say, I do not think they are humble enough. I do not think they feel their confession. What right have you to judge them, as if you could read the heart? The word of God says, “If he repent, forgive him. And if he trespasses against thee seven times in a day, and seven times in a day turn again to thee, saying, I repent; thou shalt forgive him.” Luke 17:3, 4. …

Give the erring one no occasion for discouragement. Suffer not a Pharisaical hardness to come in and hurt your brother. Let no bitter sneer rise in mind or heart. Let no tinge of scorn be manifest in the voice. If you speak a word of your own, if you take an attitude of indifference, or show suspicion or distrust, it may prove the ruin of a soul. He needs a brother with the Elder Brother’s heart of sympathy to touch his heart of humanity. Let him feel the strong clasp of a sympathizing hand, and hear the whisper, Let us pray. God will give a rich experience to you both. Prayer unites us with one another and with God.—Christ’s Object Lessons, pp. 249, 250.

\section*{Wednesday – Making It Practical}

The cross of Calvary appeals to us in power, affording a reason why we should love our Saviour, and why we should make Him first and last and best in everything. We should take our fitting place in humble penitence at the foot of the cross. Here, as we see our Saviour in agony, the Son of God dying, the just for the unjust, we may learn lessons of meekness and lowliness of mind. Behold Him who with one word could summon legions of angels to His assistance, a subject of jest and merriment, of reviling and hatred. He gives Himself a sacrifice for sin. When reviled, He threatens not; when falsely accused, He opens not His mouth. He prays on the cross for His murderers. He is dying for them; He is paying an infinite price for every one of them. He bears the penalty of man’s sins without a murmur.—Lift Him Up, p. 233.

Heaven viewed with grief and amazement Christ hanging upon the cross, blood flowing from His wounded temples, and sweat tinged with blood standing upon His brow. From His hands and feet the blood fell, drop by drop, upon the rock drilled for the foot of the cross. The wounds made by the nails gaped as the weight of His body dragged upon His hands. His labored breath grew quick and deep, as His soul panted under the burden of the sins of the world. All heaven was filled with wonder when the prayer of Christ was offered in the midst of His terrible suffering,—“Father, forgive them; for they know not what they do.” Luke 23:34.—The Desire of Ages, p. 760.

The Teacher from heaven, no less a personage than the Son of God, came to earth to reveal the character of the Father to men, that they might worship Him in spirit and in truth. Christ revealed to men the fact that the strictest adherence to ceremony and form would not save them; for the kingdom of God was spiritual in its nature. … He presented to men that which was exactly contrary to the representations of the enemy in regard to the character of God, and sought to impress upon men the paternal love of the Father, who “so loved the world, that He gave His only-begotten Son, that whosoever believeth in Him should not perish but have everlasting life.”

He urged upon men the necessity of prayer, repentance, confession, and the abandonment of sin. He taught them honesty, forbearance, mercy, and compassion, enjoining upon them to love not only those who loved them, but those who hated them, who treated them despitefully. In this He was revealing to them the character of the Father, who is long-suffering, merciful, and gracious, slow to anger, and full of goodness and truth. Those who accepted His teaching were under the guardian care of angels, who were commissioned to strengthen, to enlighten, that the truth might renew and sanctify the soul.—Fundamentals of Christian Education, p. 177.

\section*{Thursday – Finding Rest After Forgiveness}

Called from a dungeon, a servant of captives, a prey of ingratitude and malice, Joseph proved true to his allegiance to the God of heaven. And all Egypt marveled at the wisdom of the man whom God instructed. Pharaoh made him lord of his house, and ruler of all his substance: to bind his princes at his pleasure; and teach his senators wisdom.” Psalm 105:21, 22. Not to the people of Egypt alone, but to all the nations connected with that powerful kingdom, God manifested Himself through Joseph. He desired to make him a light bearer to all peoples, and He placed him next the throne of the world’s greatest empire, that the heavenly illumination might extend far and near. By his wisdom and justice, by the purity and benevolence of his daily life, by his devotion to the interests of the people,—and that people a nation of idolaters,—Joseph was a representative of Christ. In their benefactor, to whom all Egypt turned with gratitude and praise, that heathen people, and through them all the nations with which they were connected, were to behold the love of their Creator and Redeemer.—Testimonies for the Church, vol. 6, p. 219.

The heart where love reigns will be guided to a gentle, courteous, compassionate course of conduct toward others, whether they suit our fancy or not, whether they respect us or treat us ill. Love is an active principle; it keeps the good of others continually before us, thus restraining us from inconsiderate actions lest we fail of our object in winning souls to Christ. Love seeks not its own. It will not prompt men to seek their own ease and indulgence of self. It is the respect we render to I that so often hinders the growth of love. …

Another striking point in the character of Joseph, worthy of imitation by all … is his deep filial reverence. As he meets his father with tears streaming from his eyes, he hangs upon his neck in an affectionate, loving embrace. He seems to feel that he cannot do enough for his parent’s comfort and watches over his declining years with a love as tender as a mother’s. No pains is spared to show his respect and love upon all occasions. Joseph is an example of what a [son] should be.—Testimonies for the Church, vol. 5, pp. 123, 124.

[If] we come to God, feeling helpless and dependent, as we really are, and in humble, trusting faith make known our wants to Him whose knowledge is infinite, who sees everything in creation, and who governs everything by His will and word, He can and will attend to our cry, and will let light shine into our hearts. Through sincere prayer we are brought into connection with the mind of the Infinite. We may have no remarkable evidence at the time that the face of our Redeemer is bending over us in compassion and love, but this is even so. We may not feel His visible touch, but His hand is upon us in love and pitying tenderness.

When we come to ask mercy and blessing from God we should have a spirit of love and forgiveness in our … hearts.—Steps to Christ, pp. 96, 97.

\section*{Friday – Further Thought}

\setlength{\parindent}{0pt}Gospel Workers, “How God Trains His Workers,” pp. 269, 270;

Sons and Daughters of God, “In Forgiveness,” p. 153.

\end{multicols}

\end{document}

