\documentclass[a4paper, 10pt, twoside, headings=small]{scrartcl}

%%%%%%%%%%%%%%%%%%%% General %%%%%%%%%%%%%%%%%%%%%%%%%%%%%%%%%%%%%%%%%%%%%%%%%%%

\usepackage[utf8]{inputenc}
\usepackage[T1]{fontenc}
%\usepackage[protrusion=true, expansion]{microtype} 

\usepackage[hidelinks]{hyperref} % Must go after the patches


%%%%%%%%%%%%%%%%%%%% Fonts %%%%%%%%%%%%%%%%%%%%%%%%%%%%%%%%%%%%%%%%%%%%%%%%%%%%%

\usepackage{fontspec}

\usepackage{polyglossia}


\addtokomafont{disposition}{\rmfamily}


\usepackage{textcomp}

\linespread{1.1}  


\usepackage[final]{microtype}
\setmainfont[Ligatures=TeX]{XCharter}


%%%%%%%%%%%%%%%%%%%% Random Packages  %%%%%%%%%%%%%%%%%%%%%%%%%%%%%%%%%%%%%%%

\usepackage{geometry}	
\usepackage{enumerate} %Erweiterung der enumerate-Umgebung
\usepackage{ifthen,calc}
\usepackage{mathrsfs,amssymb} %Zusatzzeichen
\usepackage{wrapfig}
\usepackage[retainorgcmds]{IEEEtrantools} %Besonders geeignet für einen mehrzeilige Formelsatz
\usepackage{theorem} %Theoremlayout
\usepackage{multicol}
\setlength\columnsep{20pt}
\usepackage{csquotes}





%%%%%%%%%%%%%%%%%%%% Page %%%%%%%%%%%%%%%%%%%%%%%%%%%%%%%%%%%%%%%%%%%%%%%%%%%

% twoside
\geometry{left=2.7cm, right=2.3cm, top=2.7cm, bottom=2.2cm}


%%%%%%%%%%%% PDF %%%%%%%%%%%%%%%%%%%%%%%%%%%%%%%%%%%%%%%%%%%%%%%%%%%%%%%%%%%%

\hypersetup{
	hidelinks=true,
	%	linkcolor=black,
	%	filecolor=black,      
	%	urlcolor=black,
	%	citecolor=black,
	%	allcolors=black,
	%	allbordercolors=white,
	%pdfpagemode=FullScreen,
	%	pdftitle={\mytitle},
	%	pdfauthor={\myauthor},
	pdfkeywords={},
	%	pdfcreator={Some fancy PDF-Creator...},
	bookmarksnumbered=true
}



%%%%%%%%%%%% footnotes %%%%%%%%%%%%%%%%%%%%%%%%%%%%%%%%%%%%%%%%%%%%%%%%%%%%%%

\usepackage[flushmargin, hang]{footmisc} % flush footnote mark to left margin
\usepackage{regexpatch}
\makeatletter
% 1. remove all redefinitions about footnotes done by \maketitle
%    and add \titletrue
\regexpatchcmd{\maketitle}
{\c{def}\c{@makefnmark}.*\c{if@twocolumn}}
{\c{titletrue}\c{if@twocolumn}}
{}{}
% 2. define a conditional
\newif\iftitle
%% 3. redefine \@makefnmark to print nothing when \titletrue
%\xpretocmd{\@makefnmark}{\iftitle\else}{}{}
%\xapptocmd{\@makefnmark}{\fi}{}{}
% 4. ensure \@makefntext has \titlefalse
%    that's justified by the fact that \@makefnmark
%    in \@makefntext is set in a box
\xpretocmd{\@makefntext}{\titlefalse}{}{}

\makeatother

\renewcommand{\footnotemargin}{1em}
%\addtolength{\footnotesep}{5mm}
\skip\footins=2\bigskipamount     % Determine the space above the rule
\renewcommand*{\footnoterule}{%
	\kern-3pt%
	\hrule width 1in%
	\kern 2.6pt%
	\vspace{\smallskipamount}       % The additional space below the rule
}



%%%%%%%%%%%% captions %%%%%%%%%%%%%%%%%%%%%%%%%%%%%%%%%%%%%%%%%%%%%%%%%%%%%%%

\usepackage[textfont={small},labelfont={small, bf}]{caption}
\DeclareCaptionFont{black}{ \color{white} }
\DeclareCaptionFormat{listing}{
	\colorbox[cmyk]{0.43, 0.35, 0.35,0.01 }{
		\parbox{\textwidth}{\hspace{15pt}#1#2#3}
	}
}
\captionsetup{format=plain, singlelinecheck=true}
\captionsetup[lstlisting]{labelfont={small, bf}, textfont={small}}


%%%%%%%%%%%% Title %%%%%%%%%%%%%%%%%%%%%%%%%%%%%%%%%%%%%%%%%%%%%%%%%%%%%%%%%%

\usepackage{titling}
\setlength{\droptitle}{-5em}
\pretitle{\begin{center}\LARGE\bfseries}
	\posttitle{\par\end{center}}
\preauthor{\begin{center}}
	\postauthor{\par\end{center}}
\predate{\begin{center}}
	\postdate{\par\end{center}}


%%%%%%%%%%%% footer / header %%%%%%%%%%%%%%%%%%%%%%%%%%%%%%%%%%%%%%%%%%%%%%%%

\usepackage{fancyhdr}

% twoside with subsection
\fancyhf{}
\fancyhead[RE]{\small\nouppercase\leftmark}
\fancyhead[LO]{\small\rightmark}
\fancyhead[LE,RO]{\thepage}
\renewcommand{\headrulewidth}{0pt}


% Does not really work...
%\setotherlanguage{greek}
%\setotherlanguage{hebrew}
%\newfontfamily\greekfont[]{Linux Libertine O}
%\newfontfamily\hebrewfont[]{Linux Libertine O}

\newcommand{\bm}{\vectorbold*} % using physics package

\newcommand{\matlab}{\textsc{Matlab}\textsuperscript{\tiny{\textregistered}}}



\title{01 Living in a 24-7 Society}

\author{Ellen G.\ White}

\date{2021/03 Rest in Christ}

\begin{document}

\maketitle

\thispagestyle{empty}

\pagestyle{fancy}

\begin{multicols}{2}

\section*{Saturday – Living in a 24-7 Society}

In this speck of a world, the heavenly universe manifests the greatest interest. Yet we come in contact with the busy activity of our cities, we mingle with the multitude in the crowded thoroughfares, we enter marts of trade and walk the streets; and through all, from morning till evening, the people act as if business, sport, and pleasure were all there is to life,—as if this world were all there is to occupy the mind. How few contemplate the unseen agencies!

All heaven is intensely interested in the human beings who are so full of activity, and yet have no thought for the unseen. Sometimes the heavenly intelligences draw aside the curtain that hides the unseen world, that our minds may be withdrawn from the hurry and rush, and consider that there are witnesses to all we do and say, when engaged in business, or when we think ourselves alone.—Sons and Daughters of God, p. 37.

My soul longeth, yea, even fainteth for the courts of the Lord: my heart and my flesh crieth out for the living God. Psalm 84:2.

When God’s people take their eyes off the things of this world and place them on heaven and heavenly things they will be a peculiar people, because they will see the mercy and goodness and compassion that God has shown to the children of men. His love will call forth a response from them, and their lives will show to those around them that the Spirit of God is controlling them, that they are setting their affections on things above, not on the things of the earth.—In Heavenly Places, p. 368.

We need to appreciate more fully the meaning of the words: “I sat down under His shadow with great delight.” Song of Solomon 2:3. These words do not bring to our minds the picture of hasty transit, but of quiet rest. There are many professing Christians who are anxious and depressed, many who are so full of busy activity that they cannot find time to rest quietly in the promises of God, who act as if they could not afford to have peace and quietness. To all such Christ’s invitation is: “Come unto Me, … and I will give you rest.” Matthew 11:28.

Let us turn from the dusty, heated thoroughfares of life to rest in the shadow of Christ’s love. Here we gain strength for conflict. Here we learn how to lessen toil and worry, and how to speak and sing to the praise of God. Let the weary and the heavy-laden learn from Christ the lesson of quiet trust. They must sit under His shadow if they would be possessors of His peace and rest.—Testimonies for the Church, vol, 7, p. 69.

\section*{Sunday – Worn and Weary}

In Eden, God set up the memorial of His work of creation, in placing His blessing upon the seventh day. The Sabbath was committed to Adam, the father and representative of the whole human family. Its observance was to be an act of grateful acknowledgment, on the part of all who should dwell upon the earth, that God was their Creator and their rightful Sovereign; that they were the work of His hands and the subjects of His authority. Thus the institution was wholly commemorative, and given to all mankind. There was nothing in it shadowy or of restricted application to any people.

God saw that a Sabbath was essential for man, even in Paradise. He needed to lay aside his own interests and pursuits for one day of the seven, that he might more fully contemplate the works of God and meditate upon His power and goodness. He needed a Sabbath to remind him more vividly of God and to awaken gratitude because all that he enjoyed and possessed came from the beneficent hand of the Creator.—Patriarchs and Prophets, p. 48.

We are sustained every moment by God’s care, and upheld by His power. He spreads our tables with food. He gives us peaceful and refreshing sleep. Weekly He brings to us the Sabbath, that we may rest from our temporal labors, and worship Him in His own house. He has given us His word to be a lamp to our feet and a light to our path. In its sacred pages we find the counsels of wisdom; and as oft as we lift our hearts to Him in penitence and faith, He grants us the blessings of His grace. Above all else is the infinite gift of God’s dear Son, through whom flow all other blessings for this life and for the life to come.—Counsels on Stewardship, p. 18.

Overwork sometimes causes a loss of self-control. But the Lord never compels hurried, complicated movements. Many gather to themselves burdens that the merciful heavenly Father did not place on them. Duties He never designed them to perform chase one another wildly. God desires us to realize that we do not glorify His name when we take so many burdens that we are overtasked and, becoming heart weary and brain weary, chafe and fret and scold. We are to bear only the responsibilities that the Lord gives us, trusting in Him, and thus keeping our hearts pure and sweet and sympathetic.—My Life Today, p. 81.

Each day brings its responsibilities and duties, but the work of tomorrow must not be crowded into the hours of today. God is merciful, full of compassion, reasonable in His requirements. He does not ask us to pursue a course of action that will result in the loss of physical health or the enfeebling of the mental powers. He would not have us work under a pressure and strain until exhaustion follows, with prostration of the nerves.—Gospel Workers, p. 244.

\section*{Monday – Running on Empty}

The servants of Christ are not to treat their health indifferently. Let no one labor to the point of exhaustion, thereby disqualifying himself for future effort. Do not try to crowd into one day the work of two. At the end, those who work carefully and wisely will be found to have accomplished as much as those who so expend their physical and mental strength that they have no deposit from which to draw in time of need.

God’s work is world-wide; it calls for every jot and tittle of the ability and power that we have. … After His servants have done their best, they may say, The harvest truly is great, and the laborers are few; but God “knoweth our frame; He remembereth that we are dust.” [Psalm 103:14.]—Gospel Workers, p. 244.

There are three ways in which the Lord reveals His will to us, to guide us, and to fit us to guide others. How may we know His voice from that of a stranger? How shall we distinguish it from the voice of a false shepherd? God reveals His will to us in His word, the Holy Scriptures. His voice is also revealed in His providential workings; and it will be recognized if we do not separate our souls from Him by walking in our own ways, doing according to our own wills, and following the promptings of an unsanctified heart, until the senses have become so confused that eternal things are not discerned, and the voice of Satan is so disguised that it is accepted as the voice of God.

Another way in which God’s voice is heard is through the appeals of His Holy Spirit, making impressions upon the heart, which will be wrought out in the character. If you are in doubt upon any subject you must first consult the Scriptures. If you have truly begun the life of faith you have given yourself to the Lord to be wholly His, and He has taken you to mold and fashion according to His purpose, that you may be a vessel unto honor.—Testimonies for the Church, vol. 5, p. 512.

Let all who are afflicted or unjustly used, cry to God. Turn away from those whose hearts are as steel, and make your requests known to your Maker. Never is one repulsed who comes to Him with a contrite heart. Not one sincere prayer is lost. Amid the anthems of the celestial choir, God hears the cries of the weakest human being. We pour out our heart’s desire in our closets, we breathe a prayer as we walk by the way, and our words reach the throne of the Monarch of the universe. They may be inaudible to any human ear, but they cannot die away into silence, nor can they be lost through the activities of business that are going on. Nothing can drown the soul’s desire. It rises above the din of the street, above the confusion of the multitude, to the heavenly courts. It is God to whom we are speaking, and our prayer is heard.—Christ’s Object Lessons, p. 174.

\section*{Tuesday – Defining Rest in the Old Testament}

In all ages God’s appointed witnesses have exposed themselves to reproach and persecution for the truth’s sake. Joseph was maligned and persecuted because he preserved his virtue and integrity. David, the chosen messenger of God, was hunted like a beast of prey by his enemies. Daniel was cast into a den of lions because he was true to his allegiance to heaven. Job was deprived of his worldly possessions, and so afflicted in body that he was abhorred by his relatives, and friends; yet he maintained his integrity. Jeremiah could not be deterred from speaking the words that God had given him to speak; and his testimony so enraged the king and princes that he was cast into a loathsome pit. Stephen was stoned because he preached Christ and Him crucified. Paul was imprisoned, beaten with rods, stoned, and finally put to death because he was a faithful messenger for God to the Gentiles. And John was banished to the Isle of Patmos “for the word of God, and for the testimony of Jesus Christ.”

These examples of human steadfastness bear witness to the faithfulness of God’s promises—of His abiding presence and sustaining grace. They testify to the power of faith to withstand the powers of the world. It is the work of faith to rest in God in the darkest hour, to feel, however sorely tried and tempest-tossed, that our Father is at the helm. The eye of faith alone can look beyond the things of time to estimate aright the worth of the eternal riches.—The Acts of the Apostles, p. 575.

Adam’s life was one of sorrow, humility, and continual repentance. As he taught his children and grandchildren the fear of the Lord, he was often bitterly reproached for his sin which resulted in so much misery upon his posterity. When he left the beautiful Eden, the thought that he must die thrilled him with horror. He looked upon death as a dreadful calamity. He was first made acquainted with the dreadful reality of death in the human family by his own son Cain slaying his brother Abel. Filled with the bitterest remorse for his own transgression, and deprived of his son Abel, and looking upon Cain as his murderer, and knowing the curse God pronounced upon him, bowed down Adam’s heart with grief. Most bitterly did he reproach himself for his first great transgression. He entreated pardon from God through the promised Sacrifice. Deeply had he felt the wrath of God for his crime committed in Paradise. He witnessed the general corruption which afterward finally provoked God to destroy the inhabitants of the earth by a flood. The sentence of death pronounced upon him by his Maker, which at first appeared so terrible to him, after he had lived some hundreds of years, looked just and merciful in God, to bring to an end a miserable life.—The Story of Redemption, p. 55.

\section*{Wednesday – Rest in the New Testament}

Christ’s words of compassion are spoken to His workers today just as surely as they were spoken to His disciples. “Come ye yourselves apart, … and rest awhile,” He says to those who are worn and weary. It is not wise to be always under the strain of work and excitement, even in ministering to men’s spiritual needs; for in this way personal piety is neglected, and the powers of mind and soul and body are overtaxed. Self-denial is required of the disciples of Christ, and sacrifices must be made; but care must also be exercised lest through their overzeal Satan take advantage of the weakness of humanity, and the work of God be marred.

In the estimation of the rabbis it was the sum of religion to be always in a bustle of activity. They depended upon some outward performance to show their superior piety. Thus they separated their souls from God, and built themselves up in self-sufficiency. The same dangers still exist. As activity increases and men become successful in doing any work for God, there is danger of trusting to human plans and methods. There is a tendency to pray less, and to have less faith. Like the disciples, we are in danger of losing sight of our dependence on God, and seeking to make a savior of our activity. We need to look constantly to Jesus, realizing that it is His power which does the work. While we are to labor earnestly for the salvation of the lost, we must also take time for meditation, for prayer, and for the study of the word of God.—The Desire of Ages, p. 362.

Those who learn of Jesus His meekness and lowliness find rest in the experience of practicing His lessons. It is not in indolence, in selfish ease and pleasure-seeking, that rest is obtained. Those who are unwilling to give the Lord faithful, earnest, loving service will not find spiritual rest in this life or in the life to come. Only from earnest labor comes peace and joy in the Holy Spirit—happiness on earth and glory hereafter.—Ellen G. White Comments, in The SDA Bible Commentary, vol. 7, p. 928.

There will be peace, constant peace, flowing into the soul, for the rest is found in perfect submission to Jesus Christ. Obedience to God’s will finds the rest. The disciple that treads in the meek and lowly steps of the Redeemer finds rest which the world cannot give, and the world cannot take away. “Thou wilt keep him in perfect peace whose mind is stayed on thee: because he trusteth in thee.” Isaiah 26:3.—Our High Calling, p. 98.

A life in Christ is a life of restfulness. There may be no ecstasy of feeling, but there should be an abiding, peaceful trust. Your hope is not in yourself; it is in Christ. Your weakness is united to His strength, your ignorance to His wisdom, your frailty to His enduring might. So you are not to look to yourself, not to let the mind dwell upon self, but look to Christ.—Steps to Christ, p. 70.

\section*{Thursday – A Restless Wanderer}

[Cain and Abel] had been instructed in regard to the provision made for the salvation of the human race. They were required to carry out a system of humble obedience, showing their reverence for God and their faith and dependence upon the promised Redeemer, by slaying the firstlings of the flock and solemnly presenting them with the blood as a burnt offering to God. This sacrifice would lead them to continually keep in mind their sin and the Redeemer to come, who was to be the great sacrifice for man.

Cain … was unwilling to strictly follow the plan of obedience and procure a lamb and offer it with the fruit of the ground. He merely took of the ground and disregarded the requirement of God. … Abel advised his brother not to come before the Lord without the blood of sacrifice. Cain, being the eldest, would not listen to his brother. He despised his counsel, and with doubt and murmuring in regard to the necessity of the ceremonial offerings, he presented his offering. But God did not accept it.—The Story of Redemption, p. 52.

When Cain saw that his offering was rejected, he was angry with the Lord and with Abel; he was angry that God did not accept man’s substitute in place of the sacrifice divinely ordained, and angry with his brother for choosing to obey God instead of joining in rebellion against Him. … Through an angel messenger the divine warning was conveyed: “If thou doest well, shalt thou not be accepted? And if thou doest not well, sin lieth at the door.” The choice lay with Cain himself. If he would trust to the merits of the promised Saviour, and would obey God’s requirements, he would enjoy His favor. But should he persist in unbelief and transgression, he would have no ground for complaint because he was rejected by the Lord. …

Cain hated and killed his brother, not for any wrong that Abel had done, but “because his own works were evil, and his brother’s righteous.” 1 John 3:12. So in all ages the wicked have hated those who were better than themselves. Abel’s life of obedience and unswerving faith was to Cain a perpetual reproof. … The brighter the heavenly light that is reflected from the character of God’s faithful servants, the more clearly the sins of the ungodly are revealed, and the more determined will be their efforts to destroy those who disturb their peace.—Patriarchs and Prophets, pp. 73, 74.

It is the love of self that brings unrest. When we are born from above, the same mind will be in us that was in Jesus, the mind that led Him to humble Himself that we might be saved. Then we shall not be seeking the highest place. We shall desire to sit at the feet of Jesus, and learn of Him. We shall understand that the value of our work does not consist in making a show and noise in the world, and in being active and zealous in our own strength. The value of our work is in proportion to the impartation of the Holy Spirit. Trust in God brings holier qualities of mind, so that in patience we may possess our souls.—The Desire of Ages, p. 330.

\section*{Friday – Further Thought}

\setlength{\parindent}{0pt}My Life Today, “Rest,” p. 133;

Steps to Christ, “Growing Up Into Christ,” p. 72.

\end{multicols}

\end{document}

