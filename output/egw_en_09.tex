\documentclass[a4paper, 10pt, twoside, headings=small]{scrartcl}

%%%%%%%%%%%%%%%%%%%% General %%%%%%%%%%%%%%%%%%%%%%%%%%%%%%%%%%%%%%%%%%%%%%%%%%%

\usepackage[utf8]{inputenc}
\usepackage[T1]{fontenc}
%\usepackage[protrusion=true, expansion]{microtype} 

\usepackage[hidelinks]{hyperref} % Must go after the patches


%%%%%%%%%%%%%%%%%%%% Fonts %%%%%%%%%%%%%%%%%%%%%%%%%%%%%%%%%%%%%%%%%%%%%%%%%%%%%

\usepackage{fontspec}

\usepackage{polyglossia}


\addtokomafont{disposition}{\rmfamily}


\usepackage{textcomp}

\linespread{1.1}  


\usepackage[final]{microtype}
\setmainfont[Ligatures=TeX]{XCharter}


%%%%%%%%%%%%%%%%%%%% Random Packages  %%%%%%%%%%%%%%%%%%%%%%%%%%%%%%%%%%%%%%%

\usepackage{geometry}	
\usepackage{enumerate} %Erweiterung der enumerate-Umgebung
\usepackage{ifthen,calc}
\usepackage{mathrsfs,amssymb} %Zusatzzeichen
\usepackage{wrapfig}
\usepackage[retainorgcmds]{IEEEtrantools} %Besonders geeignet für einen mehrzeilige Formelsatz
\usepackage{theorem} %Theoremlayout
\usepackage{multicol}
\setlength\columnsep{20pt}
\usepackage{csquotes}





%%%%%%%%%%%%%%%%%%%% Page %%%%%%%%%%%%%%%%%%%%%%%%%%%%%%%%%%%%%%%%%%%%%%%%%%%

% twoside
\geometry{left=2.7cm, right=2.3cm, top=2.7cm, bottom=2.2cm}


%%%%%%%%%%%% PDF %%%%%%%%%%%%%%%%%%%%%%%%%%%%%%%%%%%%%%%%%%%%%%%%%%%%%%%%%%%%

\hypersetup{
	hidelinks=true,
	%	linkcolor=black,
	%	filecolor=black,      
	%	urlcolor=black,
	%	citecolor=black,
	%	allcolors=black,
	%	allbordercolors=white,
	%pdfpagemode=FullScreen,
	%	pdftitle={\mytitle},
	%	pdfauthor={\myauthor},
	pdfkeywords={},
	%	pdfcreator={Some fancy PDF-Creator...},
	bookmarksnumbered=true
}



%%%%%%%%%%%% footnotes %%%%%%%%%%%%%%%%%%%%%%%%%%%%%%%%%%%%%%%%%%%%%%%%%%%%%%

\usepackage[flushmargin, hang]{footmisc} % flush footnote mark to left margin
\usepackage{regexpatch}
\makeatletter
% 1. remove all redefinitions about footnotes done by \maketitle
%    and add \titletrue
\regexpatchcmd{\maketitle}
{\c{def}\c{@makefnmark}.*\c{if@twocolumn}}
{\c{titletrue}\c{if@twocolumn}}
{}{}
% 2. define a conditional
\newif\iftitle
%% 3. redefine \@makefnmark to print nothing when \titletrue
%\xpretocmd{\@makefnmark}{\iftitle\else}{}{}
%\xapptocmd{\@makefnmark}{\fi}{}{}
% 4. ensure \@makefntext has \titlefalse
%    that's justified by the fact that \@makefnmark
%    in \@makefntext is set in a box
\xpretocmd{\@makefntext}{\titlefalse}{}{}

\makeatother

\renewcommand{\footnotemargin}{1em}
%\addtolength{\footnotesep}{5mm}
\skip\footins=2\bigskipamount     % Determine the space above the rule
\renewcommand*{\footnoterule}{%
	\kern-3pt%
	\hrule width 1in%
	\kern 2.6pt%
	\vspace{\smallskipamount}       % The additional space below the rule
}



%%%%%%%%%%%% captions %%%%%%%%%%%%%%%%%%%%%%%%%%%%%%%%%%%%%%%%%%%%%%%%%%%%%%%

\usepackage[textfont={small},labelfont={small, bf}]{caption}
\DeclareCaptionFont{black}{ \color{white} }
\DeclareCaptionFormat{listing}{
	\colorbox[cmyk]{0.43, 0.35, 0.35,0.01 }{
		\parbox{\textwidth}{\hspace{15pt}#1#2#3}
	}
}
\captionsetup{format=plain, singlelinecheck=true}
\captionsetup[lstlisting]{labelfont={small, bf}, textfont={small}}


%%%%%%%%%%%% Title %%%%%%%%%%%%%%%%%%%%%%%%%%%%%%%%%%%%%%%%%%%%%%%%%%%%%%%%%%

\usepackage{titling}
\setlength{\droptitle}{-5em}
\pretitle{\begin{center}\LARGE\bfseries}
	\posttitle{\par\end{center}}
\preauthor{\begin{center}}
	\postauthor{\par\end{center}}
\predate{\begin{center}}
	\postdate{\par\end{center}}


%%%%%%%%%%%% footer / header %%%%%%%%%%%%%%%%%%%%%%%%%%%%%%%%%%%%%%%%%%%%%%%%

\usepackage{fancyhdr}

% twoside with subsection
\fancyhf{}
\fancyhead[RE]{\small\nouppercase\leftmark}
\fancyhead[LO]{\small\rightmark}
\fancyhead[LE,RO]{\thepage}
\renewcommand{\headrulewidth}{0pt}


% Does not really work...
%\setotherlanguage{greek}
%\setotherlanguage{hebrew}
%\newfontfamily\greekfont[]{Linux Libertine O}
%\newfontfamily\hebrewfont[]{Linux Libertine O}

\newcommand{\bm}{\vectorbold*} % using physics package

\newcommand{\matlab}{\textsc{Matlab}\textsuperscript{\tiny{\textregistered}}}



\setmainlanguage[]{english}

\title{09 The Rhythms of Rest}

\author{Ellen G.\ White}

\date{2021/03 Rest in Christ}

\begin{document}

\maketitle

\thispagestyle{empty}

\pagestyle{fancy}

\begin{multicols}{2}

\section*{Saturday – The Rhythms of Rest}

Christ sought to draw the attention of His disciples away from the artificial to the natural: “If God so clothe the grass of the field, which today is, and tomorrow is cast into the oven, shall he not much more clothe you, O ye of little faith!” Why did not our heavenly Father carpet the earth with brown or gray? He chose the color that was most restful, the most acceptable to the senses. How it cheers the heart and refreshes the weary spirit to look upon the earth, clad in its garments of living green! Without this covering the air would be filled with dust, and the earth would appear like a desert.

Every spire of grass, every opening bud and blooming flower is a token of God’s love, and should teach us a lesson of faith and trust in Him. Christ calls our attention to their natural loveliness, and assures us that the most gorgeous array of the greatest king that ever wielded an earthly scepter was not equal to that worn by the humblest flower. You who are sighing for the artificial splendor which wealth alone can purchase, for costly paintings, furniture, and dress, listen to the voice of the divine Teacher. He points you to the flower of the field, the simple design of which cannot be equaled by human skill.—Sons and Daughters of God, p. 75.

Christ has linked His teaching, not only with the day of rest, but with the week of toil. He has wisdom for him who drives the plow and sows the seed. In the plowing and sowing, the tilling and reaping, He teaches us to see an illustration of His work of grace in the heart. So in every line of useful labor and every association of life, He desires us to find a lesson of divine truth. Then our daily toil will no longer absorb our attention and lead us to forget God; it will continually remind us of our Creator and Redeemer. The thought of God will run like a thread of gold through all our homely cares and occupations. For us the glory of His face will again rest upon the face of nature. We shall ever be learning new lessons of heavenly truth, and growing into the image of His purity. Thus shall we “be taught of the Lord”; and in the lot wherein we are called, we shall “abide with God.” Isaiah 54:13; 1 Corinthians 7:24.—Christ’s Object Lessons, pp. 25, 26.

All the heavenly beings are in constant activity, and the Lord Jesus, in His lifework, has given an example for everyone. He went about “doing good.” God has established the law of obedient action. Silent but ceaseless, the objects of His creation do their appointed work. The ocean is in constant motion. The springing grass, which today is, and tomorrow is cast into the oven, does its errand, clothing the fields with beauty. The leaves are stirred to motion, and yet no hand is seen to touch them. The sun, moon, and stars are useful and glorious in fulfilling their mission.—My Life Today, p. 130.

\section*{Sunday – Prelude to Rest}

The earth came forth from the hand of the Creator exceedingly beautiful. There were mountains and hills and plains; and interspersed among them were rivers and bodies of water. The earth was not one extensive plain, but the monotony of the scenery was broken by hills and mountains, not high and ragged as they now are, but regular and beautiful in shape. The bare, high rocks were never seen upon them, but lay beneath the surface, answering as bones to the earth. The waters were regularly dispersed. The hills, mountains, and very beautiful plains were adorned with plants and flowers, and tall, majestic trees of every description, which were many times larger and much more beautiful than trees now are. The air was pure and healthful, and the earth seemed like a noble palace. Angels beheld and rejoiced at the wonderful and beautiful works of God.—Lift Him Up, p. 47.

Man was the crowning act of the creation of God, made in the image of God, and designed to be a counterpart of God; but Satan has labored to obliterate the image of God in man, and to imprint upon him his own image. Man is very dear to God, because he was formed in His own image. …

In order to understand the value which God places upon man, we need to comprehend the plan of redemption, the costly sacrifice which our Saviour made to save the human race from eternal ruin. Jesus died to regain possession of the one pearl of great price. …

The Lord gave His only begotten Son to ransom us from sin. We are His workmanship, we are His representatives in the world, and He expects that we shall reveal the true value of man by our purity of life, and the earnest efforts put forth to recover the pearl of great price. Our character is to be modeled after the divine similitude, and to be reformed by that faith that works by love and purifies the soul. The grace of God will beautify, ennoble, and sanctify the character. The servant of the Lord who works intelligently will be successful.—Lift Him Up, p. 48.

Jesus pointed His hearers back to the marriage institution as ordained at creation. … He referred them to the blessed days of Eden, when God pronounced all things “very good.” Then marriage and the Sabbath had their origin, twin institutions for the glory of God in the benefit of humanity. Then, as the Creator joined the hands of the holy pair in wedlock, saying, A man shall “leave his father and his mother, and shall cleave unto his wife: and they shall be one” (Genesis 2:24), He enunciated the law of marriage for all the children of Adam to the close of time. That which the Eternal Father Himself had pronounced good was the law of highest blessing and development for man.—Thoughts From the Mount of Blessing, p. 63.

\section*{Monday – The Command to Rest}

God reserved the seventh day as a period of rest for man, for the good of man as well as for His own glory. He saw that the wants of man required a day of rest from toil and care, that his health and life would be endangered without a period of relaxation from the labor and anxiety of the six days. …

The Sabbath bids us behold in His created works the glory of the Creator. And it is because He desired us to do this that Jesus bound up His precious lessons with the beauty of natural things. On the holy rest day, above all other days, we should study the messages that God has written for us in nature. We should study the Saviour’s parables where He spoke them, in the fields and groves, under the open sky, among the grass and flowers. As we come close to the heart of nature, Christ makes His presence real to us and speaks to our hearts of His peace and love.—My Life Today, p. 140.

The Sabbath was to be a sign between God and his people forever. In this manner was it to be a sign—all who should observe the Sabbath signified by such observance that they were worshipers of the living God, the Creator of the Heavens and the earth. The Sabbath was to be a sign between God and his people as long as he should have a people upon the earth to serve him.—Spiritual Gifts, vol. 3, p. 267.

As the tree of knowledge was placed in the midst of the Garden of Eden, so the Sabbath command is placed in the midst of the Decalogue. In regard to the fruit of the tree of knowledge, the restriction was made, “Ye shall not eat of it, … lest ye die.” Genesis 3:3. Of the Sabbath God said, Ye shall not defile it, but keep it holy. “Remember the sabbath day, to keep it holy.” Exodus 20:8. As the tree of knowledge was the test of Adam’s obedience, so the fourth command is the test that God has given to prove the loyalty of all His people.

The Sabbath is a token between God and His people. It is a holy day, given by the Creator to man as a day upon which to rest, and reflect upon sacred things. God designed it to be observed through every age as a perpetual covenant. It was to be regarded as a peculiar treasure, a trust to be carefully cherished.

As we observe the Sabbath let us remember that it is the sign which heaven has given to man that he is accepted in the Beloved; that if he is obedient, he may enter the city of God, and partake of the fruit of the tree of life. As we refrain from labor on the seventh day, we testify to the world that we are on God’s side, and are striving to live in perfect conformity to His commandments. Thus we recognize as our sovereign the God who made the world in six days and rested on the seventh.—Our High Calling, p. 343.

\section*{Tuesday – New Circumstances}

God manifested his great care and love for his people in sending them bread from Heaven. “Man did eat angels’ food.” That is, food provided for them by the angels. In the three-fold miracle of the manna, a double quantity on the sixth day, and none upon the seventh, and its keeping fresh through the Sabbath, while upon other days it would become unfit for use, was designed to impress them with the sacredness of the Sabbath. After they were abundantly supplied with food, they were ashamed of their unbelief and murmurings, and promised to trust the Lord for the future. But they soon forgot their promise, and failed at the first trial of their faith.—Spiritual Gifts, vol. 3, p. 255.

Before entering the Promised Land, the Israelites were admonished by Moses to “keep the Sabbath day to sanctify it.” Deuteronomy 5:12. The Lord designed that by a faithful observance of the Sabbath command, Israel should continually be reminded of their accountability to Him as their Creator and their Redeemer. While they should keep the Sabbath in the proper spirit, idolatry could not exist; but should the claims of this precept of the Decalogue be set aside as no longer binding, the Creator would be forgotten and men would worship other gods. “I gave them My Sabbaths,” God declared, “to be a sign between Me and them, that they might know that I am the Lord that sanctify them.” Ezekiel 20:12.—Prophets and Kings, p. 181.

Do we believe with all the heart that Christ is soon coming and that we are now having the last message of mercy that is ever to be given to a guilty world? Is our example what it should be? Do we, by our lives and holy conversation, show to those around us that we are looking for the glorious appearing of our Lord and Saviour Jesus Christ, who shall change these vile bodies and fashion them like unto His glorious body? I fear that we do not believe and realize these things as we should. Those who believe the important truths that we profess, should act out their faith.—Early Writings, p. 111.

Christ is coming with power and great glory. He is coming with His own glory and with the glory of the Father. He is coming with all the holy angels with Him. While all the world is plunged in darkness, there will be light in every dwelling of the saints. They will catch the first light of His second appearing. The unsullied light will shine from His splendor, and Christ the Redeemer will be admired by all who have served Him. … To His faithful followers Christ has been a daily companion and familiar friend. They have lived in close contact, in constant communion with God. Upon them the glory of the Lord has risen. In them the light of the knowledge of the glory of God in the face of Jesus Christ has been reflected. Now they rejoice in the undimmed rays of the brightness and glory of the King in His majesty.—Christ’s Object Lessons, p. 420.

\section*{Wednesday – Another Reason to Rest}

The Lord God of heaven is our Leader. He is a leader whom we can safely follow; for He never makes a mistake. Let us honor God and His Son Jesus Christ, through whom He communicates with the world. It was Christ who gave to Moses the instruction that He gave to the children of Israel. It was Christ who delivered the Israelites from Egyptian bondage. Moses and Aaron were the visible leaders of the people. To Moses instruction was given by their invisible Leader, to be repeated to them.

Had Israel obeyed the directions given them by Moses, not one of those who started on the journey from Egypt would in the wilderness have fallen a prey to disease or death. They were under a safe Guide. Christ had pledged Himself to lead them safely to the promised land if they would follow His guidance. This vast multitude, numbering more than a million people, was under His direct rule. They were His family. In every one of them He was interested.—Ellen G. White Comments, in The SDA Bible Commentary, vol. 1, pp. 1117, 1118.

Redemption is an inexhaustible theme, worthy of our closest contemplation. It passes the comprehension of the deepest thought, the stretch of the most vivid imagination. Who by searching can find out God? The treasures of wisdom and knowledge are opened to all men, and were thousands of the most gifted men to devote their whole time to setting forth Jesus always before us, studying how they might portray His matchless charms, they would never exhaust the subject.

Although great and talented authors have made known wonderful truths, and have presented increased light to the people, still in our day we shall find new ideas, and ample fields in which to work, for the theme of salvation is inexhaustible. The work has gone forward from century to century, setting forth the life and character of Christ, and the love of God as manifested in the atoning sacrifice. The theme of redemption will employ the minds of the redeemed through all eternity. There will be new and rich developments made manifest in the plan of salvation throughout eternal ages.—Selected Messages, book 1, p. 403.

Christ, the Light of the world, veiled the dazzling splendor of His divinity and came to live as a man among men, that they might, without being consumed, become acquainted with their Creator. Christ came to teach human beings what God desires them to know. In the heavens above, in the earth, in the broad waters of the ocean, we see the handiwork of God. All created things testify to His power, His wisdom, His love. But not from the stars or the ocean or the cataract can we learn of the personality of God as it is revealed in Christ.

Tender, compassionate, sympathetic, ever considerate of others, He represented the character of God, and was constantly engaged in service for God and man.—Sons and Daughters of God, p. 21.

\section*{Thursday – Keeping the Sabbath}

The Sabbath was not for Israel merely, but for the world. It had been made known to man in Eden, and, like the other precepts of the Decalogue, it is of imperishable obligation. Of that law of which the fourth commandment forms a part, Christ declares, “Till heaven and earth pass, one jot or one tittle shall in nowise pass from the law.” So long as the heavens and the earth endure, the Sabbath will continue as a sign of the Creator’s power. And when Eden shall bloom on earth again, God’s holy rest day will be honored by all beneath the sun. “From one Sabbath to another” the inhabitants of the glorified new earth shall go up “to worship before Me, saith the Lord.” Matthew 5:18; Isaiah 66:23.—The Desire of Ages, p. 283.

The Sabbath of the Lord is to be made a blessing to us and to our children. They can be pointed to the blooming flowers and the opening buds, the lofty trees and beautiful spires of grass, and taught that God made all these in six days, and rested on the seventh day, and hallowed it. Thus the parents may bind up their lessons of instruction to their children, so that when these children look upon the things of nature, they will call to mind the great Creator of them all. Their thoughts will be carried up to nature’s God—back to the creation of our world, when the foundation of the Sabbath was laid, and all the sons of God shouted for joy.

Happy is the family who can go to the place of worship on the Sabbath as Jesus and His disciples went to the synagogue—across the fields, along the shores of the lake, or through the groves.—My Life Today, p. 140.

The Sabbath is a sign of creative and redeeming power; it points to God as the source of life and knowledge; it recalls man’s primeval glory, and thus witnesses to God’s purpose to re-create us in His own image.

The Sabbath and the family were alike instituted in Eden, and in God’s purpose they are indissolubly linked together. On this day more than on any other, it is possible for us to live the life of Eden. It was God’s plan for the members of the family to be associated in work and study, in worship and recreation, the father as priest of his household, and both father and mother as teachers and companions of their children. But the results of sin, having changed the conditions of life, to a great degree prevent this association. Often the father hardly sees the faces of his children throughout the week. He is almost wholly deprived of opportunity for companionship or instruction. But God’s love has set a limit to the demands of toil. Over the Sabbath He places His merciful hand. In His own day He preserves for the family opportunity for communion with Him, with nature, and with one another.—Education, p. 250.

\section*{Friday – Further Thought}

\setlength{\parindent}{0pt}My Life Today, “Reverence for the Sabbath,” p. 287;

Spiritual Gifts, “Disguised Infidelity,” vol. 3, pp. 90–95.

\end{multicols}

\end{document}

