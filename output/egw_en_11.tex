\documentclass[a4paper, 10pt, twoside, headings=small]{scrartcl}

%%%%%%%%%%%%%%%%%%%% General %%%%%%%%%%%%%%%%%%%%%%%%%%%%%%%%%%%%%%%%%%%%%%%%%%%

\usepackage[utf8]{inputenc}
\usepackage[T1]{fontenc}
%\usepackage[protrusion=true, expansion]{microtype} 

\usepackage[hidelinks]{hyperref} % Must go after the patches


%%%%%%%%%%%%%%%%%%%% Fonts %%%%%%%%%%%%%%%%%%%%%%%%%%%%%%%%%%%%%%%%%%%%%%%%%%%%%

\usepackage{fontspec}

\usepackage{polyglossia}


\addtokomafont{disposition}{\rmfamily}


\usepackage{textcomp}

\linespread{1.1}  


\usepackage[final]{microtype}
\setmainfont[Ligatures=TeX]{XCharter}


%%%%%%%%%%%%%%%%%%%% Random Packages  %%%%%%%%%%%%%%%%%%%%%%%%%%%%%%%%%%%%%%%

\usepackage{geometry}	
\usepackage{enumerate} %Erweiterung der enumerate-Umgebung
\usepackage{ifthen,calc}
\usepackage{mathrsfs,amssymb} %Zusatzzeichen
\usepackage{wrapfig}
\usepackage[retainorgcmds]{IEEEtrantools} %Besonders geeignet für einen mehrzeilige Formelsatz
\usepackage{theorem} %Theoremlayout
\usepackage{multicol}
\setlength\columnsep{20pt}
\usepackage{csquotes}





%%%%%%%%%%%%%%%%%%%% Page %%%%%%%%%%%%%%%%%%%%%%%%%%%%%%%%%%%%%%%%%%%%%%%%%%%

% twoside
\geometry{left=2.7cm, right=2.3cm, top=2.7cm, bottom=2.2cm}


%%%%%%%%%%%% PDF %%%%%%%%%%%%%%%%%%%%%%%%%%%%%%%%%%%%%%%%%%%%%%%%%%%%%%%%%%%%

\hypersetup{
	hidelinks=true,
	%	linkcolor=black,
	%	filecolor=black,      
	%	urlcolor=black,
	%	citecolor=black,
	%	allcolors=black,
	%	allbordercolors=white,
	%pdfpagemode=FullScreen,
	%	pdftitle={\mytitle},
	%	pdfauthor={\myauthor},
	pdfkeywords={},
	%	pdfcreator={Some fancy PDF-Creator...},
	bookmarksnumbered=true
}



%%%%%%%%%%%% footnotes %%%%%%%%%%%%%%%%%%%%%%%%%%%%%%%%%%%%%%%%%%%%%%%%%%%%%%

\usepackage[flushmargin, hang]{footmisc} % flush footnote mark to left margin
\usepackage{regexpatch}
\makeatletter
% 1. remove all redefinitions about footnotes done by \maketitle
%    and add \titletrue
\regexpatchcmd{\maketitle}
{\c{def}\c{@makefnmark}.*\c{if@twocolumn}}
{\c{titletrue}\c{if@twocolumn}}
{}{}
% 2. define a conditional
\newif\iftitle
%% 3. redefine \@makefnmark to print nothing when \titletrue
%\xpretocmd{\@makefnmark}{\iftitle\else}{}{}
%\xapptocmd{\@makefnmark}{\fi}{}{}
% 4. ensure \@makefntext has \titlefalse
%    that's justified by the fact that \@makefnmark
%    in \@makefntext is set in a box
\xpretocmd{\@makefntext}{\titlefalse}{}{}

\makeatother

\renewcommand{\footnotemargin}{1em}
%\addtolength{\footnotesep}{5mm}
\skip\footins=2\bigskipamount     % Determine the space above the rule
\renewcommand*{\footnoterule}{%
	\kern-3pt%
	\hrule width 1in%
	\kern 2.6pt%
	\vspace{\smallskipamount}       % The additional space below the rule
}



%%%%%%%%%%%% captions %%%%%%%%%%%%%%%%%%%%%%%%%%%%%%%%%%%%%%%%%%%%%%%%%%%%%%%

\usepackage[textfont={small},labelfont={small, bf}]{caption}
\DeclareCaptionFont{black}{ \color{white} }
\DeclareCaptionFormat{listing}{
	\colorbox[cmyk]{0.43, 0.35, 0.35,0.01 }{
		\parbox{\textwidth}{\hspace{15pt}#1#2#3}
	}
}
\captionsetup{format=plain, singlelinecheck=true}
\captionsetup[lstlisting]{labelfont={small, bf}, textfont={small}}


%%%%%%%%%%%% Title %%%%%%%%%%%%%%%%%%%%%%%%%%%%%%%%%%%%%%%%%%%%%%%%%%%%%%%%%%

\usepackage{titling}
\setlength{\droptitle}{-5em}
\pretitle{\begin{center}\LARGE\bfseries}
	\posttitle{\par\end{center}}
\preauthor{\begin{center}}
	\postauthor{\par\end{center}}
\predate{\begin{center}}
	\postdate{\par\end{center}}


%%%%%%%%%%%% footer / header %%%%%%%%%%%%%%%%%%%%%%%%%%%%%%%%%%%%%%%%%%%%%%%%

\usepackage{fancyhdr}

% twoside with subsection
\fancyhf{}
\fancyhead[RE]{\small\nouppercase\leftmark}
\fancyhead[LO]{\small\rightmark}
\fancyhead[LE,RO]{\thepage}
\renewcommand{\headrulewidth}{0pt}


% Does not really work...
%\setotherlanguage{greek}
%\setotherlanguage{hebrew}
%\newfontfamily\greekfont[]{Linux Libertine O}
%\newfontfamily\hebrewfont[]{Linux Libertine O}

\newcommand{\bm}{\vectorbold*} % using physics package

\newcommand{\matlab}{\textsc{Matlab}\textsuperscript{\tiny{\textregistered}}}



\setmainlanguage[]{english}

\title{11 Longing  for More}

\author{Ellen G.\ White}

\date{2021/03 Rest in Christ}

\begin{document}

\maketitle

\thispagestyle{empty}

\pagestyle{fancy}

\begin{multicols}{2}

\section*{Saturday – Longing  for More}

For hundreds of years the Scriptures had been translated into the Greek language, then widely spoken throughout the Roman Empire. The Jews were scattered everywhere, and their expectation of the Messiah’s coming was to some extent shared by the Gentiles. Among those whom the Jews styled heathen were men who had a better understanding of the Scripture prophecies concerning the Messiah than had the teachers in Israel. There were some who hoped for His coming as a deliverer from sin. Philosophers endeavored to study into the mystery of the Hebrew economy. But the bigotry of the Jews hindered the spread of the light. Intent on maintaining the separation between themselves and other nations, they were unwilling to impart the knowledge they still possessed concerning the symbolic service. The true Interpreter must come. The One whom all these types prefigured must explain their significance.

Through nature, through types and symbols, through patriarchs and prophets, God had spoken to the world. Lessons must be given to humanity in the language of humanity. The Messenger of the covenant must speak. His voice must be heard in His own temple. Christ must come to utter words which should be clearly and definitely understood. He, the author of truth, must separate truth from the chaff of man’s utterance, which had made it of no effect. The principles of God’s government and the plan of redemption must be clearly defined. The lessons of the Old Testament must be fully set before men.—The Desire of Ages, pp. 33, 34.

We should seek to follow more closely the example of Christ, the great Shepherd, as He worked with His little company of disciples, studying with them and with the people the Old Testament Scriptures. His active ministry consisted not merely in sermonizing but in educating the people. As He passed through villages, He came in personal contact with the people in their homes, teaching, and ministering to their necessities. As the crowds that followed Him increased, when He came to a favorable place, He would speak to them, simplifying His discourses by the use of parables and symbols.—Evangelism, p. 203.

Christ’s manner of teaching was beautiful and attractive, and it was ever characterized by simplicity. He unfolded the mysteries of the kingdom of heaven through the use of figures and symbols with which His hearers were familiar; and the common people heard Him gladly, for they could comprehend His words. There were no high-sounding words used, to understand which it was necessary to consult a dictionary.—Counsels to Parents, Teachers, and Students, p. 240.

The Jewish economy, bearing the signature of Heaven, had been instituted by Christ Himself. In types and symbols the great truths of redemption were veiled. Yet when Christ came, the Jews did not recognize Him to whom all these symbols pointed. They had the word of God in their hands; but the traditions which had been handed down from generation to generation, and the human interpretation of the Scriptures, hid from them the truth as it is in Jesus. The spiritual import of the sacred writings was lost. The treasure house of all knowledge was open to them, but they knew it not.—Christ’s Object Lessons, p. 104.

\section*{Sunday – Baptized Into Moses}

The example of ancient Israel is given as a warning to the people of God, that they may avoid unbelief and escape His wrath. If the iniquities of the Hebrews had been omitted from the Sacred Record, and only their virtues recounted, their history would fail to teach us the lesson that it does. …

The principles of justice required a faithful narration of facts for the benefit of all who should ever read the Sacred Record. Here we discern the evidences of divine wisdom. We are required to obey the law of God, and are not only instructed as to the penalty of disobedience, but we have narrated for our benefit and warning the history of Adam and Eve in Paradise, and the sad results of their disobedience of God’s commands. … Their example is given us as a warning against disobedience, that we may be sure that the wages of sin is death, that God’s retributive justice never fails, and that He exacts from His creatures a strict regard for His commandments. …

There before us lie the lives of the believers, with all their faults and follies, which are intended as a lesson to all the generations following them. If they had been without foible they would have been more than human, and our sinful natures would despair of ever reaching such a point of excellence. But seeing where they struggled and fell, where they took heart again and conquered through the grace of God, we are encouraged, and led to press over the obstacles that degenerate nature places in our way.—Testimonies for the Church, vol. 4, pp. 11, 12.

The Old Testament is the gospel in figures and symbols. The New Testament is the substance. One is as essential as the other. The Old Testament presents lessons from the lips of Christ, and these lessons have not lost their force in any particular.—Selected Messages, book 2, p. 104.

God commanded Moses for Israel, “Let them make Me a sanctuary; that I may dwell among them” (Exodus 25:8), and He abode in the sanctuary, in the midst of His people. Through all their weary wandering in the desert, the symbol of His presence was with them. So Christ set up His tabernacle in the midst of our human encampment. He pitched His tent by the side of the tents of men, that He might dwell among us, and make us familiar with His divine character and life. “The Word became flesh, and tabernacled among us (and we beheld His glory, glory as of the Only Begotten from the Father), full of grace and truth.” John 1:14, R. V., margin.—The Desire of Ages, p. 23.

\section*{Monday – Ritual and Sacrifices}

Nearly two thousand years ago, a voice of mysterious import was heard in heaven, from the throne of God, “Lo, I come.” … Christ was about to visit our world, and to become incarnate. He says, “A body hast Thou prepared Me.” Had He appeared with the glory that was His with the Father before the world was, we could not have endured the light of His presence. That we might behold it and not be destroyed, the manifestation of His glory was shrouded. His divinity was veiled with humanity,—the invisible glory in the visible human form.

This great purpose had been shadowed forth in types and symbols. The burning bush, in which Christ appeared to Moses, revealed God. The symbol chosen for the representation of the Deity was a lowly shrub, that seemingly had no attractions. This enshrined the Infinite. The all-merciful God shrouded His glory in a most humble type, that Moses could look upon it and live. So in the pillar of cloud by day and the pillar of fire by night, God communicated with Israel, revealing to men His will, and imparting to them His grace. God’s glory was subdued, and His majesty veiled, that the weak vision of finite men might behold it. … His glory was veiled, His greatness and majesty were hidden, that He might draw near to sorrowful, tempted men.—The Desire of Ages, p. 23.

Every morning and evening a lamb of a year old was burned upon the altar, with its appropriate meat offering, thus symbolizing the daily consecration of the nation to Jehovah, and their constant dependence upon the atoning blood of Christ. … The priests were to examine all animals brought as a sacrifice, and were to reject every one in which a defect was discovered. Only an offering “without blemish” could be a symbol of His perfect purity who was to offer Himself as “a lamb without blemish and without spot.” 1 Peter 1:19.

The apostle Paul points to these sacrifices as an illustration of what the followers of Christ are to become. He says, “I beseech you therefore, brethren, by the mercies of God, that ye present your bodies a living sacrifice, holy, acceptable unto God, which is your reasonable service.” Romans 12:1.—Patriarchs and Prophets, p. 352.

Christ was the Lamb slain from the foundation of the world. To many it has been a mystery why so many sacrificial offerings were required in the old dispensation, why so many bleeding victims were led to the altar. But the great truth that was to be kept before men, and imprinted upon mind and heart, was this, “Without shedding of blood is no remission.” In every bleeding sacrifice was typified “the Lamb of God, which taketh away the sin of the world.”

Christ Himself was the originator of the Jewish system of worship, in which, by types and symbols, were shadowed forth spiritual and heavenly things. Many forgot the true significance of these offerings; and the great truth that through Christ alone there is forgiveness of sin, was lost to them.—Ellen G. White Comments, in The SDA Bible Commentary, vol. 7, pp. 932, 933.

\section*{Tuesday – The “Example” of Rest}

There remaineth therefore a rest to the people of God. For he that is entered into his rest, he also hath ceased from his own works, as God [did] from his. Let us labour therefore to enter into that rest, lest any man fall after the same example of unbelief. [Hebrews 4:9–11].

The rest here spoken of is the rest of grace, obtained by following the prescription, Labor diligently. Those who learn of Jesus His meekness and lowliness find rest in the experience of practicing His lessons. It is not in indolence, in selfish ease and pleasure-seeking, that rest is obtained. Those who are unwilling to give the Lord faithful, earnest, loving service will not find spiritual rest in this life or in the life to come. Only from earnest labor comes peace and joy in the Holy Spirit—happiness on earth and glory hereafter.—Ellen G. White Comments, in The SDA Bible Commentary, vol. 7, p. 928.

Rest is found when all self-justification, all reasoning from a selfish standpoint, is put away. Entire self-surrender, an acceptance of His ways, is the secret of perfect rest in His love. Do just what He has told you to do, and be assured that God will do all that He has said He would do. Have you come to Him, renouncing all your makeshifts, all your unbelief, all your self-righteousness? Come just as you are, weak, helpless, and ready to die.

What is the “rest” promised?—It is the consciousness that God is true, that He never disappoints the one who comes to Him. His pardon is full and free, and His acceptance means rest to the soul, rest in His love.—Our High Calling, p. 97.

We shall be saved eternally when we enter in through the gates into the city. Then we may rejoice that we are saved, eternally saved. But until then we need to heed the injunction of the apostle, and to “fear, lest, a promise being left us of entering into his rest, any of us should seem to come short of it” (Hebrews 4:1). Having a knowledge of Canaan, singing the songs of Canaan, rejoicing in the prospect of entering into Canaan, did not bring the children of Israel into the vineyards and olive groves of the Promised Land. They could make it theirs in truth only by occupation, by complying with the conditions, by exercising living faith in God, by appropriating His promises to themselves.

Christ is the author and finisher of our faith, and when we yield to His hand we shall steadily grow in grace and in the knowledge of our Lord and Saviour. We shall make progress until we reach the full stature of men and women in Christ. Faith works by love, and purifies the soul, expelling the love of sin that leads to rebellion against, and transgression of, the law of God.—That I May Know Him, p. 162.

\section*{Wednesday – “Harden Not Your Hearts”}

God requires prompt and unquestioning obedience of His law; but men are asleep or paralyzed by the deceptions of Satan, who suggests excuses and subterfuges, and conquers their scruples, saying as he said to Eve in the garden: “Ye shall not surely die.” Disobedience not only hardens the heart and conscience of the guilty one, but it tends to corrupt the faith of others. That which looked very wrong to them at first, gradually loses this appearance by being constantly before them, till finally they question whether it is really sin and unconsciously fall into the same error. …

Many are the hindrances that lie in the path of those who would walk in obedience to the commandments of God. There are strong and subtle influences that bind them to the ways of the world, but the power of the Lord can break these chains. He will remove every obstacle from before the feet of His faithful ones or give them strength and courage to conquer every difficulty, if they earnestly beseech His help. All hindrances will vanish before an earnest desire and persistent effort to do the will of God at any cost to self, even if life itself is sacrificed. Light from heaven will illuminate the darkness of those, who, in trial and perplexity, go forward, looking unto Jesus as the Author and Finisher of their faith.—Testimonies for the Church, vol. 4, pp. 146, 147.

The coldness of ice, the hardness of iron, the impenetrable, unimpressible nature of rock—all these find a counterpart in the character of many a professed Christian. It was thus that the Lord hardened the heart of Pharaoh. God spoke to the Egyptian king by the mouth of Moses, giving him the most striking evidences of divine power; but the monarch stubbornly refused the light which would have brought him to repentance. God did not send a supernatural power to harden the heart of the rebellious king, but as Pharaoh resisted the truth, the Holy Spirit was withdrawn, and he was left to the darkness and unbelief which he had chosen. By persistent rejection of the Spirit’s influence, men cut themselves off from God. He has in reserve no more potent agency to enlighten their minds. No revelation of His will can reach them in their unbelief.—Our High Calling, p. 160.

Christ says: “I have chosen you, and ordained you, that ye should go and bring forth fruit, and that your fruit should remain” (John 15:16). As Christ’s ambassador, I would entreat of all who read these lines to take heed while it is called today. “If ye will hear his voice, harden not your hearts” (Hebrews 3:15; 4:7). Without waiting a moment, inquire, What am I to Christ? and what is Christ to me? What is my work? What is the character of the fruit I bear?—This Day With God, p. 51.

\section*{Thursday – Conquering a Heavenly City}

No distinction on account of nationality, race, or caste, is recognized by God. He is the Maker of all mankind. All men are of one family by creation, and all are one through redemption. Christ came to demolish every wall of partition, to throw open every compartment of the temple, that every soul may have free access to God. His love is so broad, so deep, so full, that it penetrates everywhere. It lifts out of Satan’s circle the poor souls who have been deluded by his deceptions. It places them within reach of the throne of God, the throne encircled by the rainbow of promise.

In Christ there is neither Jew nor Greek, bond nor free. All are brought nigh by His precious blood. (Galatians 3:28; Ephesians 2:13).

Whatever the difference in religious belief, a call from suffering humanity must be heard and answered. Where bitterness of feeling exists because of difference in religion, much good may be done by personal service. Loving ministry will break down prejudice, and win souls to God.—Christ’s Object Lessons, p. 386.

It is impossible for us, of ourselves, to escape from the pit of sin in which we are sunken. Our hearts are evil, and we cannot change them. “Who can bring a clean thing out of an unclean? not one.” “The carnal mind is enmity against God: for it is not subject to the law of God, neither indeed can be.” Job 14:4; Romans 8:7. Education, culture, the exercise of the will, human effort, all have their proper sphere, but here they are powerless. They may produce an outward correctness of behavior, but they cannot change the heart; they cannot purify the springs of life. There must be a power working from within, a new life from above, before men can be changed from sin to holiness. That power is Christ. His grace alone can quicken the lifeless faculties of the soul, and attract it to God, to holiness.—Steps to Christ, p. 18.

Many make a serious mistake in their religious life by keeping the attention fixed upon their feelings and thus judging of their advancement or decline. Feelings are not a safe criterion. We are not to look within for evidence of our acceptance with God. We shall find there nothing but that which will discourage us. Our only hope is in “looking unto Jesus the Author and Finisher of our faith.” There is everything in Him to inspire with hope, with faith, and with courage. He is our righteousness, our consolation and rejoicing.

Those who look within for comfort will become weary and disappointed. A sense of our weakness and unworthiness should lead us with humility of heart to plead the atoning sacrifice of Christ. As we rely upon His merits we shall find rest and peace and joy. He saves to the uttermost all who come unto God by Him.—Testimonies for the Church, vol. 5, pp. 199, 200.

\section*{Friday – Further Thought}

\setlength{\parindent}{0pt}The Upward Look, “Mix Faith With Hearing,” p. 75;

Spiritual Gifts, “Facts of Faith,” vol. 3, pp. 295, 296.

\end{multicols}

\end{document}

