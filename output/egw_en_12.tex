\documentclass[a4paper, 10pt, twoside, headings=small]{scrartcl}

%%%%%%%%%%%%%%%%%%%% General %%%%%%%%%%%%%%%%%%%%%%%%%%%%%%%%%%%%%%%%%%%%%%%%%%%

\usepackage[utf8]{inputenc}
\usepackage[T1]{fontenc}
%\usepackage[protrusion=true, expansion]{microtype} 

\usepackage[hidelinks]{hyperref} % Must go after the patches


%%%%%%%%%%%%%%%%%%%% Fonts %%%%%%%%%%%%%%%%%%%%%%%%%%%%%%%%%%%%%%%%%%%%%%%%%%%%%

\usepackage{fontspec}

\usepackage{polyglossia}


\addtokomafont{disposition}{\rmfamily}


\usepackage{textcomp}

\linespread{1.1}  


\usepackage[final]{microtype}
\setmainfont[Ligatures=TeX]{XCharter}


%%%%%%%%%%%%%%%%%%%% Random Packages  %%%%%%%%%%%%%%%%%%%%%%%%%%%%%%%%%%%%%%%

\usepackage{geometry}	
\usepackage{enumerate} %Erweiterung der enumerate-Umgebung
\usepackage{ifthen,calc}
\usepackage{mathrsfs,amssymb} %Zusatzzeichen
\usepackage{wrapfig}
\usepackage[retainorgcmds]{IEEEtrantools} %Besonders geeignet für einen mehrzeilige Formelsatz
\usepackage{theorem} %Theoremlayout
\usepackage{multicol}
\setlength\columnsep{20pt}
\usepackage{csquotes}





%%%%%%%%%%%%%%%%%%%% Page %%%%%%%%%%%%%%%%%%%%%%%%%%%%%%%%%%%%%%%%%%%%%%%%%%%

% twoside
\geometry{left=2.7cm, right=2.3cm, top=2.7cm, bottom=2.2cm}


%%%%%%%%%%%% PDF %%%%%%%%%%%%%%%%%%%%%%%%%%%%%%%%%%%%%%%%%%%%%%%%%%%%%%%%%%%%

\hypersetup{
	hidelinks=true,
	%	linkcolor=black,
	%	filecolor=black,      
	%	urlcolor=black,
	%	citecolor=black,
	%	allcolors=black,
	%	allbordercolors=white,
	%pdfpagemode=FullScreen,
	%	pdftitle={\mytitle},
	%	pdfauthor={\myauthor},
	pdfkeywords={},
	%	pdfcreator={Some fancy PDF-Creator...},
	bookmarksnumbered=true
}



%%%%%%%%%%%% footnotes %%%%%%%%%%%%%%%%%%%%%%%%%%%%%%%%%%%%%%%%%%%%%%%%%%%%%%

\usepackage[flushmargin, hang]{footmisc} % flush footnote mark to left margin
\usepackage{regexpatch}
\makeatletter
% 1. remove all redefinitions about footnotes done by \maketitle
%    and add \titletrue
\regexpatchcmd{\maketitle}
{\c{def}\c{@makefnmark}.*\c{if@twocolumn}}
{\c{titletrue}\c{if@twocolumn}}
{}{}
% 2. define a conditional
\newif\iftitle
%% 3. redefine \@makefnmark to print nothing when \titletrue
%\xpretocmd{\@makefnmark}{\iftitle\else}{}{}
%\xapptocmd{\@makefnmark}{\fi}{}{}
% 4. ensure \@makefntext has \titlefalse
%    that's justified by the fact that \@makefnmark
%    in \@makefntext is set in a box
\xpretocmd{\@makefntext}{\titlefalse}{}{}

\makeatother

\renewcommand{\footnotemargin}{1em}
%\addtolength{\footnotesep}{5mm}
\skip\footins=2\bigskipamount     % Determine the space above the rule
\renewcommand*{\footnoterule}{%
	\kern-3pt%
	\hrule width 1in%
	\kern 2.6pt%
	\vspace{\smallskipamount}       % The additional space below the rule
}



%%%%%%%%%%%% captions %%%%%%%%%%%%%%%%%%%%%%%%%%%%%%%%%%%%%%%%%%%%%%%%%%%%%%%

\usepackage[textfont={small},labelfont={small, bf}]{caption}
\DeclareCaptionFont{black}{ \color{white} }
\DeclareCaptionFormat{listing}{
	\colorbox[cmyk]{0.43, 0.35, 0.35,0.01 }{
		\parbox{\textwidth}{\hspace{15pt}#1#2#3}
	}
}
\captionsetup{format=plain, singlelinecheck=true}
\captionsetup[lstlisting]{labelfont={small, bf}, textfont={small}}


%%%%%%%%%%%% Title %%%%%%%%%%%%%%%%%%%%%%%%%%%%%%%%%%%%%%%%%%%%%%%%%%%%%%%%%%

\usepackage{titling}
\setlength{\droptitle}{-5em}
\pretitle{\begin{center}\LARGE\bfseries}
	\posttitle{\par\end{center}}
\preauthor{\begin{center}}
	\postauthor{\par\end{center}}
\predate{\begin{center}}
	\postdate{\par\end{center}}


%%%%%%%%%%%% footer / header %%%%%%%%%%%%%%%%%%%%%%%%%%%%%%%%%%%%%%%%%%%%%%%%

\usepackage{fancyhdr}

% twoside with subsection
\fancyhf{}
\fancyhead[RE]{\small\nouppercase\leftmark}
\fancyhead[LO]{\small\rightmark}
\fancyhead[LE,RO]{\thepage}
\renewcommand{\headrulewidth}{0pt}


% Does not really work...
%\setotherlanguage{greek}
%\setotherlanguage{hebrew}
%\newfontfamily\greekfont[]{Linux Libertine O}
%\newfontfamily\hebrewfont[]{Linux Libertine O}

\newcommand{\bm}{\vectorbold*} % using physics package

\newcommand{\matlab}{\textsc{Matlab}\textsuperscript{\tiny{\textregistered}}}



\setmainlanguage[]{english}

\title{12 The Restless Prophet}

\author{Ellen G.\ White}

\date{2021/03 Rest in Christ}

\begin{document}

\maketitle

\thispagestyle{empty}

\pagestyle{fancy}

\begin{multicols}{2}

\section*{Saturday – The Restless Prophet}

Among the cities of the ancient world in the days of divided Israel one of the greatest was Nineveh, the capital of the Assyrian realm. Founded on the fertile bank of the Tigris, soon after the dispersion from the tower of Babel, it had flourished through the centuries until it had become “an exceeding great city of three days’ journey.” Jonah 3:3.

In the time of its temporal prosperity Nineveh was a center of crime and wickedness. Inspiration has characterized it as “the bloody city, … full of lies and robbery.” In figurative language the prophet Nahum compared the Ninevites to a cruel, ravenous lion. “Upon whom,” he inquired, “hath not thy wickedness passed continually?” Nahum 3:1, 19.

Yet Nineveh, wicked though it had become, was not wholly given over to evil. He who “beholdeth all the sons of men” (Psalm 33:13) and “seeth every precious thing” (Job 28:10) perceived in that city many who were reaching out after something better and higher, and who, if granted opportunity to learn of the living God, would put away their evil deeds and worship Him. And so in His wisdom God revealed Himself to them in an unmistakable manner, to lead them, if possible, to repentance.—Prophets and Kings, p. 265.

As Jonah was three days and three nights in the belly of the whale, Christ was to be the same time “in the heart of the earth.” And as the preaching of Jonah was a sign to the Ninevites, so Christ’s preaching was a sign to His generation. But what a contrast in the reception of the word! The people of the great heathen city trembled as they heard the warning from God. Kings and nobles humbled themselves; the high and the lowly together cried to the God of heaven, and His mercy was granted unto them. “The men of Nineveh shall rise in judgment with this generation,” Christ had said, “and shall condemn it: because they repented at the preaching of Jonas; and, behold, a greater than Jonas is here.” Matthew 12:40, 41.—The Desire of Ages, p. 406.

The lesson is for God’s messengers today, when the cities of the nations are as verily in need of a knowledge of the attributes and purposes of the true God as were the Ninevites of old. Christ’s ambassadors are to point men to the nobler world, which has largely been lost sight of. … Through His ministering servants the Lord Jesus is calling upon men to strive with sanctified ambition to secure the immortal inheritance. He urges them to lay up treasure beside the throne of God. …

Our God is a God of mercy. With long-sufferance and tender compassion He deals with the transgressors of His law. And yet, in this our day, when men and women have so many opportunities for becoming familiar with the divine law as revealed in Holy Writ, the great Ruler of the universe cannot behold with any satisfaction the wicked cities, where reign violence and crime. The end of God’s forbearance with those who persist in disobedience is approaching rapidly.—Prophets and Kings, pp. 274, 275.

\section*{Sunday – Running Away}

Satan had been working to make the gulf deep and impassable between earth and heaven. By his falsehoods he had emboldened men in sin. It was his purpose to wear out the forbearance of God, and to extinguish His love for man, so that He would abandon the world to satanic jurisdiction.

Satan was seeking to shut out from men a knowledge of God, to turn their attention from the temple of God, and to establish his own kingdom. His strife for supremacy had seemed to be almost wholly successful. It is true that in every generation God had His agencies. Even among the heathen there were men through whom Christ was working to uplift the people from their sin and degradation. But these men were despised and hated. Many of them suffered a violent death. The dark shadow that Satan had cast over the world grew deeper and deeper.

Through heathenism, Satan had for ages turned men away from God; but he won his great triumph in perverting the faith of Israel. By contemplating and worshiping their own conceptions, the heathen had lost a knowledge of God, and had become more and more corrupt. So it was with Israel. The principle that man can save himself by his own works lay at the foundation of every heathen religion. … Satan had implanted this principle. Wherever it is held, men have no barrier against sin.—The Desire of Ages, p. 34, 35.

To [Jonah] came the word of the Lord, “Arise, go to Nineveh, that great city, and cry against it; for their wickedness is come up before Me.” Jonah 1:1, 2.

As the prophet thought of the difficulties and seeming impossibilities of this commission, he was tempted to question the wisdom of the call. From a human viewpoint it seemed as if nothing could be gained by proclaiming such a message in that proud city. He forgot for the moment that the God whom he served was all-wise and all-powerful. While he hesitated, still doubting, Satan overwhelmed him with discouragement. The prophet was seized with a great dread, and he “rose up to flee unto Tarshish.” …

In the charge given him, Jonah had been entrusted with a heavy responsibility; yet He who had bidden him go was able to sustain His servant and grant him success. Had the prophet obeyed unquestioningly, he would have been spared many bitter experiences, and would have been blessed abundantly. Yet in the hour of Jonah’s despair the Lord did not desert him. Through a series of trials and strange providences, the prophet’s confidence in God and in His infinite power to save was to be revived.—Prophets and Kings, p. 266.

A lost sheep never finds its way back to the fold of itself. If it is not sought for and saved by the watchful shepherd, it wanders until it perishes. What a representation of the Saviour is this! Unless Jesus, the Good Shepherd, had come to seek and to save the wandering, we should have perished. The Pharisees had taught that none but the Jewish nation would be saved, and they treated all other nationalities with contempt.—Lift Him Up, p. 212.

\section*{Monday – A Three-Day Rest}

“Now the Lord had prepared a great fish to swallow up Jonah. And Jonah was in the belly of the fish three days and three nights.

“Then Jonah prayed unto the Lord his God out of the fish’s belly, and said:

“I cried by reason of mine affliction unto the Lord, And He heard me; Out of the belly of hell cried I, And Thou heardest my voice.” …

At last Jonah had learned that “salvation belongeth unto the Lord.” Psalm 3:8. With penitence and a recognition of the saving grace of God, came deliverance. Jonah was released from the perils of the mighty deep and was cast upon the dry land.

Once more the servant of God was commissioned to warn Nineveh. “The word of the Lord came unto Jonah the second time, saying, Arise, go unto Nineveh, that great city, and preach unto it the preaching that I bid thee.” This time he did not stop to question or doubt, but obeyed unhesitatingly. He “arose, and went unto Nineveh, according to the word of the Lord.” Jonah 3:1-3.—Prophets and Kings, pp. 268, 269.

You need not go to the ends of the earth for wisdom, for God is near. It is not the capabilities you now possess or ever will have that will give you success. It is that which the Lord can do for you. We need to have far less confidence in what man can do and far more confidence in what God can do for every believing soul. He longs to have you reach after Him by faith. He longs to have you expect great things from Him. He longs to give you understanding in temporal as well as in spiritual matters. He can sharpen the intellect. He can give tact and skill. Put your talents into the work, ask God for wisdom, and it will be given you.

Take the word of Christ as your assurance. Has He not invited you to come unto Him? Never allow yourself to talk in a hopeless, discouraged way. If you do you will lose much. By looking at appearances and complaining when difficulties and pressure come, you give evidence of a sickly, enfeebled faith. Talk and act as if your faith was invincible. The Lord is rich in resources; He owns the world. Look heavenward in faith. Look to Him who has light and power and efficiency.—Christ’s Object Lessons, p. 146.

All true obedience comes from the heart. It was heart work with Christ. And if we consent, He will so identify Himself with our thoughts and aims, so blend our hearts and minds into conformity to His will, that when obeying Him we shall be but carrying out our own impulses. The will, refined and sanctified, will find its highest delight in doing His service. When we know God as it is our privilege to know Him, our life will be a life of continual obedience.—The Desire of Ages, p. 668.

\section*{Tuesday – Mission Accomplished}

As Jonah entered the city, he began at once to “cry against” it the message, “Yet forty days, and Nineveh shall be overthrown.” Verse 4. From street to street he went, sounding the note of warning.

The message was not in vain. The cry that rang through the streets of the godless city was passed from lip to lip until all the inhabitants had heard the startling announcement. The Spirit of God pressed the message home to every heart and caused multitudes to tremble because of their sins and to repent in deep humiliation. …

As king and nobles, with the common people, the high and the low, “repented at the preaching of Jonas” (Matthew 12:41) and united in crying to the God of heaven, His mercy was granted them. He “saw their works, that they turned from their evil way; and God repented of the evil, that He had said that He would do unto them; and He did it not.” Jonah 3:10. Their doom was averted, the God of Israel was exalted and honored throughout the heathen world, and His law was revered.—Prophets and Kings, p. 270.

When sin has deadened the moral perceptions, the wrongdoer does not discern the defects of his character nor realize the enormity of the evil he has committed; and unless he yields to the convicting power of the Holy Spirit he remains in partial blindness to his sin. His confessions are not sincere and in earnest. To every acknowledgment of his guilt he adds an apology in excuse of his course, declaring that if it had not been for certain circumstances he would not have done this or that for which he is reproved.—Steps to Christ, p. 40.

We cannot afford to neglect one ray of light God has given. To be sluggish in our practice of those things which require diligence is to commit sin. The human agent is to cooperate with God, and keep under those passions which should be in subjection. To do this he must be unwearied in his prayers to God, ever obtaining grace to control his spirit, temper, and actions. Through the imparted grace of Christ, he may be enabled to overcome. To be an overcomer means more than many suppose it means.

The Spirit of God will answer the cry of every penitent heart; for repentance is the gift of God, and an evidence that Christ is drawing the soul to Himself. We can no more repent of sin without Christ, than we can be pardoned without Christ, and yet it is a humiliation to man with his human passion and pride to go to Jesus straightway, believing and trusting Him for everything which he needs.

Let no man present the idea that man has little or nothing to do in the great work of overcoming; for God does nothing for man without his cooperation. … Man’s efforts alone are nothing but worthlessness; but cooperation with Christ means a victory. Of ourselves we have no power to repent of sin. Unless we accept divine aid we cannot take the first step toward the Saviour.—Selected Messages, book 1, pp. 380, 381.

\section*{Wednesday – An Angry, Restless Missionary}

When Jonah learned of God’s purpose to spare the city that, notwithstanding its wickedness, had been led to repent in sackcloth and ashes, he should have been the first to rejoice because of God’s amazing grace; but instead he allowed his mind to dwell upon the possibility of his being regarded as a false prophet. Jealous of his reputation, he lost sight of the infinitely greater value of the souls in that wretched city. The compassion shown by God toward the repentant Ninevites “displeased Jonah exceedingly, and he was very angry.” …

Once more he yielded to his inclination to question and doubt, and once more he was overwhelmed with discouragement. Losing sight of the interests of others, and feeling as if he would rather die than live to see the city spared, in his dissatisfaction he exclaimed, “Now, O Lord, take, I beseech Thee, my life from me; for it is better for me to die than to live.”—Prophets and Kings, p. 272.

Confused, humiliated, and unable to understand God’s purpose in sparing Nineveh, Jonah nevertheless had fulfilled the commission given him to warn that great city; and though the event predicted did not come to pass, yet the message of warning was nonetheless from God. And it accomplished the purpose God designed it should. The glory of His grace was revealed among the heathen. Those who had long been sitting “in darkness and in the shadow of death, being bound in affliction and iron,” “cried unto the Lord in their trouble,” and “He saved them out of their distresses. He brought them out of darkness and the shadow of death, and brake their bands in sunder.” Psalm 107:10, 13, 14.—Prophets and Kings, pp. 272, 273.

[The Saviour’s] steps are turned toward Jerusalem, where His foes have long plotted to take His life; now He will lay it down. He set His face steadfastly to go to persecution, denial, rejection, condemnation, and death.

And He “sent messengers before His face: and they went, and entered into a village of the Samaritans, to make ready for Him.” But the people refused to receive Him, because He was on His way to Jerusalem. This they interpreted as meaning that Christ showed a preference for the Jews, whom they hated with intense bitterness. Had He come to restore the temple and worship upon Mount Gerizim, they would gladly have received Him; but He was going to Jerusalem, and they would show Him no hospitality. Little did they realize that they were turning from their doors the best gift of heaven. …

It is no part of Christ’s mission to compel men to receive Him. It is Satan, and men actuated by his spirit, that seek to compel the conscience. … but Christ is ever showing mercy, ever seeking to win by the revealing of His love. He can admit no rival in the soul, nor accept of partial service; but He desires only voluntary service, the willing surrender of the heart under the constraint of love.—The Desire of Ages, pp. 486, 487.

\section*{Thursday – A Two-Way Street}

Our God is a God of mercy. With long-sufferance and tender compassion He deals with the transgressors of His law. And yet, in this our day, when men and women have so many opportunities for becoming familiar with the divine law as revealed in Holy Writ, the great Ruler of the universe cannot behold with any satisfaction the wicked cities, where reign violence and crime. The end of God’s forbearance with those who persist in disobedience is approaching rapidly. …

God’s messengers in the great cities are not to become discouraged over the wickedness, the injustice, the depravity, which they are called upon to face while endeavoring to proclaim the glad tidings of salvation. The Lord would cheer every such worker with the same message that He gave to the apostle Paul in wicked Corinth: “Be not afraid, but speak, and hold not thy peace: for I am with thee, and no man shall set on thee to hurt thee: for I have much people in this city.” Acts 18:9, 10. … In every city, filled though it may be with violence and crime, there are many who with proper teaching may learn to become followers of Jesus. Thousands may thus be reached with saving truth and be led to receive Christ as a personal Saviour.—Prophets and Kings, pp. 275, 277.

The instruction given by Jude from verse twenty to the close of the chapter, will make our work a complete whole, teaching us how to conduct the warfare in the service of Christ. No one-sided extravagance is to be revealed, no indolence of shiftlessness is to be indulged. We are not to ignore any man’s individuality, or in any way to justify cold-hearted criticism or selfish practice.

This scripture brings to view the fact that there is most earnest work to be done, and we need divine intuition that we may know how to work for souls ready to perish. There are souls to be plucked out of the fire, there are souls who are to be treated with the tenderest compassion. Workers are needed who have learned in the school of Christ His method of saving souls.—Letter 7, 1895.

Christ will impart to His messengers the same yearning love that He Himself has in seeking for the lost. We are not merely to say, “Come.” There are those who hear the call, but their ears are too dull to take in its meaning. Their eyes are too blind to see anything good in store for them. Many realize their great degradation. They say, I am not fit to be helped; leave me alone. But the workers must not desist. In tender, pitying love, lay hold of the discouraged and helpless ones. Give them your courage, your hope, your strength. By kindness compel them to come. “Of some have compassion, making a difference; and others save with fear, pulling them out of the fire.” Jude 22, 23.—Christ’s Object Lessons, p. 235.

\section*{Friday – Further Thought}

\setlength{\parindent}{0pt}The Sanctified Life, “Pride and Ambition Reproved,” pp, 57–59;

The Ministry of Healing, “Disappointments; Dangers,” pp. 177–180.

\end{multicols}

\end{document}

