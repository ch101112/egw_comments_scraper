\documentclass[a4paper, 10pt, twoside, headings=small]{scrartcl}

%%%%%%%%%%%%%%%%%%%% General %%%%%%%%%%%%%%%%%%%%%%%%%%%%%%%%%%%%%%%%%%%%%%%%%%%

\usepackage[utf8]{inputenc}
\usepackage[T1]{fontenc}
%\usepackage[protrusion=true, expansion]{microtype} 

\usepackage[hidelinks]{hyperref} % Must go after the patches


%%%%%%%%%%%%%%%%%%%% Fonts %%%%%%%%%%%%%%%%%%%%%%%%%%%%%%%%%%%%%%%%%%%%%%%%%%%%%

\usepackage{fontspec}

\usepackage{polyglossia}


\addtokomafont{disposition}{\rmfamily}


\usepackage{textcomp}

\linespread{1.1}  


\usepackage[final]{microtype}
\setmainfont[Ligatures=TeX]{XCharter}


%%%%%%%%%%%%%%%%%%%% Random Packages  %%%%%%%%%%%%%%%%%%%%%%%%%%%%%%%%%%%%%%%

\usepackage{geometry}	
\usepackage{enumerate} %Erweiterung der enumerate-Umgebung
\usepackage{ifthen,calc}
\usepackage{mathrsfs,amssymb} %Zusatzzeichen
\usepackage{wrapfig}
\usepackage[retainorgcmds]{IEEEtrantools} %Besonders geeignet für einen mehrzeilige Formelsatz
\usepackage{theorem} %Theoremlayout
\usepackage{multicol}
\setlength\columnsep{20pt}
\usepackage{csquotes}





%%%%%%%%%%%%%%%%%%%% Page %%%%%%%%%%%%%%%%%%%%%%%%%%%%%%%%%%%%%%%%%%%%%%%%%%%

% twoside
\geometry{left=2.7cm, right=2.3cm, top=2.7cm, bottom=2.2cm}


%%%%%%%%%%%% PDF %%%%%%%%%%%%%%%%%%%%%%%%%%%%%%%%%%%%%%%%%%%%%%%%%%%%%%%%%%%%

\hypersetup{
	hidelinks=true,
	%	linkcolor=black,
	%	filecolor=black,      
	%	urlcolor=black,
	%	citecolor=black,
	%	allcolors=black,
	%	allbordercolors=white,
	%pdfpagemode=FullScreen,
	%	pdftitle={\mytitle},
	%	pdfauthor={\myauthor},
	pdfkeywords={},
	%	pdfcreator={Some fancy PDF-Creator...},
	bookmarksnumbered=true
}



%%%%%%%%%%%% footnotes %%%%%%%%%%%%%%%%%%%%%%%%%%%%%%%%%%%%%%%%%%%%%%%%%%%%%%

\usepackage[flushmargin, hang]{footmisc} % flush footnote mark to left margin
\usepackage{regexpatch}
\makeatletter
% 1. remove all redefinitions about footnotes done by \maketitle
%    and add \titletrue
\regexpatchcmd{\maketitle}
{\c{def}\c{@makefnmark}.*\c{if@twocolumn}}
{\c{titletrue}\c{if@twocolumn}}
{}{}
% 2. define a conditional
\newif\iftitle
%% 3. redefine \@makefnmark to print nothing when \titletrue
%\xpretocmd{\@makefnmark}{\iftitle\else}{}{}
%\xapptocmd{\@makefnmark}{\fi}{}{}
% 4. ensure \@makefntext has \titlefalse
%    that's justified by the fact that \@makefnmark
%    in \@makefntext is set in a box
\xpretocmd{\@makefntext}{\titlefalse}{}{}

\makeatother

\renewcommand{\footnotemargin}{1em}
%\addtolength{\footnotesep}{5mm}
\skip\footins=2\bigskipamount     % Determine the space above the rule
\renewcommand*{\footnoterule}{%
	\kern-3pt%
	\hrule width 1in%
	\kern 2.6pt%
	\vspace{\smallskipamount}       % The additional space below the rule
}



%%%%%%%%%%%% captions %%%%%%%%%%%%%%%%%%%%%%%%%%%%%%%%%%%%%%%%%%%%%%%%%%%%%%%

\usepackage[textfont={small},labelfont={small, bf}]{caption}
\DeclareCaptionFont{black}{ \color{white} }
\DeclareCaptionFormat{listing}{
	\colorbox[cmyk]{0.43, 0.35, 0.35,0.01 }{
		\parbox{\textwidth}{\hspace{15pt}#1#2#3}
	}
}
\captionsetup{format=plain, singlelinecheck=true}
\captionsetup[lstlisting]{labelfont={small, bf}, textfont={small}}


%%%%%%%%%%%% Title %%%%%%%%%%%%%%%%%%%%%%%%%%%%%%%%%%%%%%%%%%%%%%%%%%%%%%%%%%

\usepackage{titling}
\setlength{\droptitle}{-5em}
\pretitle{\begin{center}\LARGE\bfseries}
	\posttitle{\par\end{center}}
\preauthor{\begin{center}}
	\postauthor{\par\end{center}}
\predate{\begin{center}}
	\postdate{\par\end{center}}


%%%%%%%%%%%% footer / header %%%%%%%%%%%%%%%%%%%%%%%%%%%%%%%%%%%%%%%%%%%%%%%%

\usepackage{fancyhdr}

% twoside with subsection
\fancyhf{}
\fancyhead[RE]{\small\nouppercase\leftmark}
\fancyhead[LO]{\small\rightmark}
\fancyhead[LE,RO]{\thepage}
\renewcommand{\headrulewidth}{0pt}


% Does not really work...
%\setotherlanguage{greek}
%\setotherlanguage{hebrew}
%\newfontfamily\greekfont[]{Linux Libertine O}
%\newfontfamily\hebrewfont[]{Linux Libertine O}

\newcommand{\bm}{\vectorbold*} % using physics package

\newcommand{\matlab}{\textsc{Matlab}\textsuperscript{\tiny{\textregistered}}}



\setmainlanguage[]{english}

\title{08 Free to Rest}

\author{Ellen G.\ White}

\date{2021/03 Rest in Christ}

\begin{document}

\maketitle

\thispagestyle{empty}

\pagestyle{fancy}

\begin{multicols}{2}

\section*{Saturday – Free to Rest}

We no longer mark out a way nor seek to bring the Lord to our wishes. If the life of the sick can glorify Him, we pray that they may live; nevertheless, not as we will but as He will. Our faith can be just as firm, and more reliable, by committing the desire to the all-wise God, and, without feverish anxiety, in perfect confidence, trusting all to Him. We have the promise. We know that He hears us if we ask according to His will. Our petitions must not take the form of a command, but of intercession for Him to do the things we desire of Him. When the church are united, they will have strength and power; but when part of them are united to the world, and many are given to covetousness, which God abhors, He can do but little for them. Unbelief and sin shut them away from God. We are so weak that we cannot bear much spiritual prosperity, lest we take the glory, and accredit goodness and righteousness to ourselves as the reason of the signal blessing of God, when it was all because of the great mercy and lovingkindness of our compassionate heavenly Father, and not because any good was found in us.—Testimonies for the Church, vol. 2, p. 149.

In praying for the sick, it should be remembered that “we know not what we should pray for as we ought.” [Romans 8:26.] We do not know whether the blessing we desire will be best or not. Therefore our prayers should include this thought: “Lord, Thou knowest every secret of the soul. Thou art acquainted with these persons. Jesus, their Advocate, gave His life for them. His love for them is greater than ours can possibly be. If, therefore, it is for Thy glory and the good of the afflicted ones, we ask, in the name of Jesus, that they may be restored to health. If it be not Thy will that they may be restored, we ask that Thy grace may comfort and Thy presence sustain them in their sufferings.”

God knows the end from the beginning. He is acquainted with the hearts of all men. He reads every secret of the soul. He knows whether those for whom prayer is offered would or would not be able to endure the trials that would come upon them should they live. He knows whether their lives would be a blessing or a curse to themselves and to the world. This is one reason why, while presenting our petitions with earnestness, we should say, “Nevertheless not my will, but Thine, be done.” [Luke 22:42.] Jesus added these words of submission to the wisdom and will of God when in the garden of Gethsemane He pleaded, “O My Father, if it be possible, let this cup pass from Me.” [Matthew 26:39.] And if they were appropriate for Him, the Son of God, how much more are they becoming on the lips of finite, erring mortals!—Gospel Workers, pp. 217, 218.

\section*{Sunday – Healing Rest}

Our Lord Jesus Christ came to this world as the unwearied servant of man’s necessity. He “took our infirmities, and bare our sicknesses,” that He might minister to every need of humanity. Matthew 8:17. The burden of disease and wretchedness and sin He came to remove. It was His mission to bring to men complete restoration; He came to give them health and peace and perfection of character.

Varied were the circumstances and needs of those who besought His aid, and none who came to Him went away unhelped. From Him flowed a stream of healing power, and in body and mind and soul men were made whole.

The Saviour’s work was not restricted to any time or place. His compassion knew no limit. … In every city, every town, every village, through which He passed, He laid His hands upon the afflicted ones and healed them. Wherever there were hearts ready to receive His message, He comforted them with the assurance of their heavenly Father’s love.—The Ministry of Healing, p. 17.

Many of those who came to Christ for help had brought disease upon themselves, yet He did not refuse to heal them. And when virtue from Him entered into these souls, they were convicted of sin, and many were healed of their spiritual disease as well as of their physical maladies.

Among these was the paralytic at Capernaum. Like the leper, this paralytic had lost all hope of recovery. His disease was the result of a sinful life, and his sufferings were embittered by remorse. In vain he had appealed to the Pharisees and doctors for relief; they pronounced him incurable, they denounced him as a sinner and declared that he would die under the wrath of God. …

His great desire was relief from the burden of sin. He longed to see Jesus and receive the assurance of forgiveness and peace with heaven. Then he would be content to live or to die, according to God’s will.—The Ministry of Healing, pp. 73, 74.

Why is it that men are so unwilling to trust Him who created man, and who can by a touch, a word, a look, heal all manner of disease? Who is more worthy of our confidence than the One who made so great a sacrifice for our redemption? Our Lord has given us definite instruction through the apostle James as to our duty in case of sickness. When human help fails, God will be the helper of His people. “Is any sick among you? let him call for the elders of the church; and let them pray over him, anointing him with oil in the name of the Lord: and the prayer of faith shall save the sick, and the Lord shall raise him up.” If the professed followers of Christ would, with purity of heart, exercise … faith in the promises of God … they would realize in soul and body the life-giving power of the Holy Spirit.—Testimonies for the Church, vo. 5, p. 196.

\section*{Monday – Root Treatment}

The Saviour looked upon the mournful countenance and saw the pleading eyes fixed upon Him. Well He knew the longing of that burdened soul. It was Christ who had brought conviction to his conscience when he was yet at home. When he repented of his sins and believed in the power of Jesus to make him whole, the mercy of the Saviour had blessed his heart. Jesus had watched the first glimmer of faith grow into a conviction that He was the sinner’s only helper, and had seen it grow stronger with every effort to come into His presence. It was Christ who had drawn the sufferer to Himself. Now, in words that fell like music on the listener’s ear, the Saviour said, “Son, be of good cheer; thy sins be forgiven thee.” Matthew 9:2.

The burden of guilt rolls from the sick man’s soul. He cannot doubt. Christ’s words reveal His power to read the heart. Who can deny His power to forgive sins? Hope takes the place of despair, and joy of oppressive gloom. The man’s physical pain is gone, and his whole being is transformed. Making no further request, he lay in peaceful silence, too happy for words.—The Ministry of Healing, pp. 75, 76.

The paralytic found in Christ healing for both the soul and the body. The spiritual healing was followed by physical restoration. This lesson should not be overlooked. There are today thousands suffering from physical disease, who, like the paralytic, are longing for the message, “Thy sins are forgiven.” The burden of sin, with its unrest and unsatisfied desires, is the foundation of their maladies. They can find no relief until they come to the Healer of the soul. The peace which He alone can give, would impart vigor to the mind, and health to the body.

Jesus came to “destroy the works of the devil.” “In Him was life,” and He says, “I am come that they might have life, and that they might have it more abundantly.” He is “a quickening spirit.” 1 John 3:8; John 1:4; 10:10; 1 Corinthians 15:45. And He still has the same life-giving power as when on earth He healed the sick, and spoke forgiveness to the sinner. He “forgiveth all thine iniquities,” He “healeth all thy diseases.” Psalm 103:3.—The Desire of Ages, p. 270.

If we surrender our lives to His service, we can never be placed in a position for which God has not made provision. Whatever may be our situation, we have a Guide to direct our way; whatever our perplexities, we have a sure Counselor; whatever our sorrow, bereavement, or loneliness, we have a sympathizing Friend. If in our ignorance we make missteps, Christ does not leave us. His voice, clear and distinct, is heard saying, “I am the Way, the Truth, and the Life.” John 14:6. “He shall deliver the needy when he crieth; the poor also, and him that hath no helper.” Psalm 72:12.—Christ’s Object Lessons, p. 173.

\section*{Tuesday – Running Away}

[Elijah] had hoped that after this display of God’s power, Jezebel would no longer have influence over the mind of Ahab, and that there would be a speedy reform throughout Israel. All day on Carmel’s height he had toiled without food. Yet when he guided the chariot of Ahab to the gate of Jezreel, his courage was strong, despite the physical strain under which he had labored.

But a reaction such as frequently follows high faith and glorious success was pressing upon Elijah. He feared that the reformation begun on Carmel might not be lasting; and depression seized him. He had been exalted to Pisgah’s top; now he was in the valley. While under the inspiration of the Almighty, he had stood the severest trial of faith; but in this time of discouragement, with Jezebel’s threat sounding in his ears, and Satan still apparently prevailing through the plotting of this wicked woman, he lost his hold on God. He had been exalted above measure, and the reaction was tremendous. Forgetting God, Elijah fled on and on, until he found himself in a dreary waste, alone. … A fugitive, far from the dwelling places of men, his spirits crushed by bitter disappointment, he desired never again to look upon the face of man. At last, utterly exhausted, he fell asleep.—Prophets and Kings, pp. 160, 161.

Keep looking unto Jesus, offering up silent prayers in faith, taking hold of His strength, whether you have any manifest feeling or not. Go right forward as if every prayer offered was lodged in the throne of God and responded to by the One whose promises never fail. Go right along, singing and making melody to God in your hearts, even when depressed by a sense of weight and sadness. I tell you as one who knows, light will come, joy will be ours, and the mists and clouds will be rolled back. …

We should daily dedicate ourselves to God and believe He accepts the sacrifice, without examining whether we have that degree of feeling that corresponds with our faith. Feeling and faith are as distinct as the east is from the west. Faith is not dependent on feeling. We must earnestly cry to God in faith, feeling or no feeling, and then live our prayers. Our assurance and evidence is God’s word, and after we have asked we must believe without doubting.—Selected Messages, book 2, pp. 242, 243.

Into the experience of all there come times of keen disappointment and utter discouragement—days when sorrow is the portion, and it is hard to believe that God is still the kind benefactor of His earthborn children; days when troubles harass the soul, till death seems preferable to life. It is then that many lose their hold on God and are brought into the slavery of doubt, the bondage of unbelief. Could we at such times discern with spiritual insight the meaning of God’s providences we should see angels seeking to save us from ourselves, striving to plant our feet upon a foundation more firm than the everlasting hills, and new faith, new life, would spring into being.—Prophets and Kings, p. 162.

\section*{Wednesday – Too Tired to Run}

[F]or those also who mourn in trial and sorrow there is comfort. The bitterness of grief and humiliation is better than the indulgences of sin. Through affliction God reveals to us the plague spots in our characters, that by His grace we may overcome our faults. Unknown chapters in regard to ourselves are opened to us, and the test comes, whether we will accept the reproof and the counsel of God. When brought into trial, we are not to fret and complain. We should not rebel, or worry ourselves out of the hand of Christ. We are to humble the soul before God.

The ways of the Lord are obscure to him who desires to see things in a light pleasing to himself. They appear dark and joyless to our human nature. But God’s ways are ways of mercy and the end is salvation. Elijah knew not what he was doing when in the desert he said that he had had enough of life, and prayed that he might die. The Lord in His mercy did not take him at his word. There was yet a great work for Elijah to do; and when his work was done, he was not to perish in discouragement and solitude in the wilderness. Not for him the descent into the dust of death, but the ascent in glory, with the convoy of celestial chariots, to the throne on high.

God’s word for the sorrowing is, “I have seen his ways, and will heal him: I will lead him also, and restore comforts unto him and to his mourners.” “I will turn their mourning into joy, and will comfort them, and make them rejoice from their sorrow.” Isaiah 57:18; Jeremiah 31:13.—The Desire of Ages, p. 301.

Whenever one is encompassed with clouds, perplexed by circumstances, or afflicted by poverty or distress, Satan is at hand to tempt and annoy. He attacks our weak points of character. He seeks to shake our confidence in God, who suffers such a condition of things to exist. We are tempted to distrust God, to question His love. Often the tempter comes to us as he came to Christ, arraying before us our weakness and infirmities. He hopes to discourage the soul and to break our hold upon God. Then he is sure of his prey. If we would meet him as Jesus did, we should escape many a defeat. By parleying with the enemy we give him an advantage.

Jesus gained the victory through submission and faith in God, and by the apostle He says to us, “Submit yourselves therefore to God. Resist the devil, and he will flee from you” (James 4:7). We cannot save ourselves from the tempter’s power; he has conquered humanity, and when we try to stand in our own strength, we shall become a prey to his devices; but “the name of the Lord is a strong tower: the righteous runneth into it, and is safe.” Satan trembles and flees before the weakest soul who finds refuge in that mighty name.—In Heavenly Places, p. 256.

\section*{Thursday – Rest and More}

Those who have not borne weighty responsibilities, or who have not been accustomed to feel very deeply, cannot understand the feelings of Elijah and are not prepared to give him the tender sympathy he deserves. God knows and can read the heart’s sore anguish under temptation and sore conflict.

As Elijah sleeps under the juniper tree, a soft touch and pleasant voice arouse him. He starts at once in his terror, as if to flee, as though the enemy who was in pursuit of his life had indeed found him. But in the pitying face of love bending over him he sees, not the face of an enemy, but of a friend. An angel has been sent with food from heaven to sustain the faithful servant of God. … Elijah was strengthened and pursued his journey to Horeb. He was in a wilderness. At night he lodged in a cave for protection from the wild beasts. …

Elijah, although a prophet of God, was a man subject to like passions as we are. We have the frailties of mortal feelings to contend with. But if we trust in God, He will never leave nor forsake us. Under all circumstances we may have firm trust in God, that He will never leave nor forsake us while we preserve our integrity.—Testimonies for the Church, vol. 3, pp. 291, 292.

Often prayer is solicited for the afflicted, the sorrowful, the discouraged; and this is right. We should pray that God will shed light into the darkened mind and comfort the sorrowful heart. But God answers prayer for those who place themselves in the channel of His blessings. While we offer prayer for these sorrowful ones, we should encourage them to try to help those more needy than themselves. The darkness will be dispelled from their own hearts as they try to help others. As we seek to comfort others with the comfort wherewith we are comforted, the blessing comes back to us.—The Ministry of Healing, p. 256.

You must not sink down discouraged. The fainthearted will be made strong; the desponding will be made to hope. God has a tender care for His people. His ear is open unto their cry. I have no fears for God’s cause. He will take care of His own cause. Our duty is to fill our lot and place, live … humble at the foot of the cross, and live faithful, holy lives before Him. While we do this we shall not be ashamed, but our souls will confide in God with holy boldness. …

… We know in whom we believe. We have not run in vain, neither labored in vain. Jesus knows us. A reckoning day is coming and all will be judged according to the deeds that are done in the body.

It is true the world is dark. Opposition may wax strong. The trifler and scorner may grow bolder and harder in their iniquity. Yet, for all this, we will not be moved. We have not run as uncertain. No, no. My heart is fixed, trusting in God. … Jesus said He would go away and prepare mansions for us, that where He is we may be also. Praise God for this. My heart leaps with joy at the cheering prospect.—Reflecting Christ, p. 351.

\section*{Friday – Further Thought}

\setlength{\parindent}{0pt}My Life Today, “A Merry Heart Makes a Cheerful Countenance,” p. 177;

Ellen G. White Comments, in The SDA Bible Commentary, “Important Lessons From Elijah,” vol. 2, pp. 1034, 1035.

\end{multicols}

\end{document}

