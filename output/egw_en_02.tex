\documentclass[a4paper, 10pt, twoside, headings=small]{scrartcl}

%%%%%%%%%%%%%%%%%%%% General %%%%%%%%%%%%%%%%%%%%%%%%%%%%%%%%%%%%%%%%%%%%%%%%%%%

\usepackage[utf8]{inputenc}
\usepackage[T1]{fontenc}
%\usepackage[protrusion=true, expansion]{microtype} 

\usepackage[hidelinks]{hyperref} % Must go after the patches


%%%%%%%%%%%%%%%%%%%% Fonts %%%%%%%%%%%%%%%%%%%%%%%%%%%%%%%%%%%%%%%%%%%%%%%%%%%%%

\usepackage{fontspec}

\usepackage{polyglossia}


\addtokomafont{disposition}{\rmfamily}


\usepackage{textcomp}

\linespread{1.1}  


\usepackage[final]{microtype}
\setmainfont[Ligatures=TeX]{XCharter}


%%%%%%%%%%%%%%%%%%%% Random Packages  %%%%%%%%%%%%%%%%%%%%%%%%%%%%%%%%%%%%%%%

\usepackage{geometry}	
\usepackage{enumerate} %Erweiterung der enumerate-Umgebung
\usepackage{ifthen,calc}
\usepackage{mathrsfs,amssymb} %Zusatzzeichen
\usepackage{wrapfig}
\usepackage[retainorgcmds]{IEEEtrantools} %Besonders geeignet für einen mehrzeilige Formelsatz
\usepackage{theorem} %Theoremlayout
\usepackage{multicol}
\setlength\columnsep{20pt}
\usepackage{csquotes}





%%%%%%%%%%%%%%%%%%%% Page %%%%%%%%%%%%%%%%%%%%%%%%%%%%%%%%%%%%%%%%%%%%%%%%%%%

% twoside
\geometry{left=2.7cm, right=2.3cm, top=2.7cm, bottom=2.2cm}


%%%%%%%%%%%% PDF %%%%%%%%%%%%%%%%%%%%%%%%%%%%%%%%%%%%%%%%%%%%%%%%%%%%%%%%%%%%

\hypersetup{
	hidelinks=true,
	%	linkcolor=black,
	%	filecolor=black,      
	%	urlcolor=black,
	%	citecolor=black,
	%	allcolors=black,
	%	allbordercolors=white,
	%pdfpagemode=FullScreen,
	%	pdftitle={\mytitle},
	%	pdfauthor={\myauthor},
	pdfkeywords={},
	%	pdfcreator={Some fancy PDF-Creator...},
	bookmarksnumbered=true
}



%%%%%%%%%%%% footnotes %%%%%%%%%%%%%%%%%%%%%%%%%%%%%%%%%%%%%%%%%%%%%%%%%%%%%%

\usepackage[flushmargin, hang]{footmisc} % flush footnote mark to left margin
\usepackage{regexpatch}
\makeatletter
% 1. remove all redefinitions about footnotes done by \maketitle
%    and add \titletrue
\regexpatchcmd{\maketitle}
{\c{def}\c{@makefnmark}.*\c{if@twocolumn}}
{\c{titletrue}\c{if@twocolumn}}
{}{}
% 2. define a conditional
\newif\iftitle
%% 3. redefine \@makefnmark to print nothing when \titletrue
%\xpretocmd{\@makefnmark}{\iftitle\else}{}{}
%\xapptocmd{\@makefnmark}{\fi}{}{}
% 4. ensure \@makefntext has \titlefalse
%    that's justified by the fact that \@makefnmark
%    in \@makefntext is set in a box
\xpretocmd{\@makefntext}{\titlefalse}{}{}

\makeatother

\renewcommand{\footnotemargin}{1em}
%\addtolength{\footnotesep}{5mm}
\skip\footins=2\bigskipamount     % Determine the space above the rule
\renewcommand*{\footnoterule}{%
	\kern-3pt%
	\hrule width 1in%
	\kern 2.6pt%
	\vspace{\smallskipamount}       % The additional space below the rule
}



%%%%%%%%%%%% captions %%%%%%%%%%%%%%%%%%%%%%%%%%%%%%%%%%%%%%%%%%%%%%%%%%%%%%%

\usepackage[textfont={small},labelfont={small, bf}]{caption}
\DeclareCaptionFont{black}{ \color{white} }
\DeclareCaptionFormat{listing}{
	\colorbox[cmyk]{0.43, 0.35, 0.35,0.01 }{
		\parbox{\textwidth}{\hspace{15pt}#1#2#3}
	}
}
\captionsetup{format=plain, singlelinecheck=true}
\captionsetup[lstlisting]{labelfont={small, bf}, textfont={small}}


%%%%%%%%%%%% Title %%%%%%%%%%%%%%%%%%%%%%%%%%%%%%%%%%%%%%%%%%%%%%%%%%%%%%%%%%

\usepackage{titling}
\setlength{\droptitle}{-5em}
\pretitle{\begin{center}\LARGE\bfseries}
	\posttitle{\par\end{center}}
\preauthor{\begin{center}}
	\postauthor{\par\end{center}}
\predate{\begin{center}}
	\postdate{\par\end{center}}


%%%%%%%%%%%% footer / header %%%%%%%%%%%%%%%%%%%%%%%%%%%%%%%%%%%%%%%%%%%%%%%%

\usepackage{fancyhdr}

% twoside with subsection
\fancyhf{}
\fancyhead[RE]{\small\nouppercase\leftmark}
\fancyhead[LO]{\small\rightmark}
\fancyhead[LE,RO]{\thepage}
\renewcommand{\headrulewidth}{0pt}


% Does not really work...
%\setotherlanguage{greek}
%\setotherlanguage{hebrew}
%\newfontfamily\greekfont[]{Linux Libertine O}
%\newfontfamily\hebrewfont[]{Linux Libertine O}

\newcommand{\bm}{\vectorbold*} % using physics package

\newcommand{\matlab}{\textsc{Matlab}\textsuperscript{\tiny{\textregistered}}}



\setmainlanguage[]{english}

\title{02 Restless and Rebellious}

\author{Ellen G.\ White}

\date{2021/03 Rest in Christ}

\begin{document}

\maketitle

\thispagestyle{empty}

\pagestyle{fancy}

\begin{multicols}{2}

\section*{Saturday – Restless and Rebellious}

I was pointed back to ancient Israel. But two of the adults of the vast army that left Egypt entered the land of Canaan. Their dead bodies were strewn in the wilderness because of their transgressions. Modern Israel are in greater danger of forgetting God and being led into idolatry than were His ancient people. Many idols are worshiped, even by professed Sabbathkeepers. God especially charged His ancient people to guard against idolatry, for if they should be led away from serving the living God, His curse would rest upon them, while if they would love Him with all their heart, with all their soul, and with all their might, He would abundantly bless them in basket and in store, and would remove sickness from the midst of them.

A blessing or a curse is now before the people of God—a blessing if they come out from the world and are separate, and walk in the path of humble obedience; and a curse if they unite with the idolatrous, who trample upon the high claims of heaven. The sins and iniquities of rebellious Israel are recorded and the picture presented before us as a warning that if we imitate their example of transgression and depart from God we shall fall as surely as did they. “Now all these things happened unto them for ensamples: and they are written for our admonition, upon whom the ends of the world are come.”—Testimonies for the Church, vol. 1, p. 609.

The history of the wilderness life of Israel was chronicled for the benefit of the Israel of God till the close of time. The record of God’s dealing with the wanderers in all their marchings to and fro, in their exposure to hunger, thirst, and weariness, and in the striking manifestations of His power for their relief, is fraught with warning and instruction for His people in this age. The varied experiences of the Hebrews was a school of preparation for their promised home in Canaan. God would have His people review in these days, with a humble heart and a teachable spirit, the trials through which ancient Israel passed, that they may be instructed in their preparation for the heavenly Canaan.—This Day With God, p. 77.

The land to which we are traveling is in every sense far more attractive than was the land of Canaan to the children of Israel. What stayed their progress just in sight of the goodly land? … It was their own willful unbelief that turned them back. … The history of the children of Israel is written as a warning to us “upon whom the ends of the world are come.” We are standing, as it were, upon the very borders of the heavenly Canaan. We may, if we will, look over on the other side and behold the attractions of the goodly land. If we have faith in the promises of God we shall show in conversation and in deportment that we are not living for this world, but are making it our first business to prepare for that holy land.—That I May Know Him, p. 169.

\section*{Sunday – Restless in a Wilderness}

When God led the children of Israel out of Egypt, it was His purpose to establish them in the land of Canaan a pure, happy, healthy people. Let us look at the means by which He would accomplish this. He subjected them to a course of discipline, which, had it been cheerfully followed, would have resulted in good, both to themselves and to their posterity. He removed flesh food from them in a great measure. He had granted them flesh in answer to their clamors, just before reaching Sinai, but it was furnished for only one day. God might have provided flesh as easily as manna, but a restriction was placed upon the people for their good. It was His purpose to supply them with food better suited to their wants than the feverish diet to which many of them had been accustomed in Egypt. The perverted appetite was to be brought into a more healthy state, that they might enjoy the food originally provided for man,—the fruits of the earth, which God gave to Adam and Eve in Eden.—Counsels on Diet and Foods, p. 377.

Satan, the author of disease and misery, will approach God’s people where he can have the greatest success. … He came with his temptations first to the mixed multitude, the believing Egyptians, and stirred them up to seditious murmurings. They would not be content with the healthful food which God had provided for them. Their depraved appetites craved a greater variety, especially flesh meats.

This murmuring soon infected nearly the whole body of the people. At first, God did not gratify their lustful appetites, but caused his judgments to come upon them, and consumed the most guilty by lightning from Heaven. Yet this, instead of humbling, only seemed to increase their murmurings. …

“And there went forth a wind from the Lord, and brought quails from the sea, and let them fall by the camp … And the people stood up all that day, and all that night, and all the next day, and they gathered the quails. … And while the flesh was yet between their teeth, the wrath of the Lord was kindled against the people, and the Lord smote the people with a very great plague.”

… They gave themselves up to seditious murmurings against Moses, and against the Lord, because they did not receive those things which would prove an injury to them. Their depraved appetites controlled them, and God gave them flesh meats, as they desired, and he let them suffer the results of gratifying their lustful appetites.—Spiritual Gifts, vol. 4a, pp. 15 – 18.

\section*{Monday – It’s Contagious}

After Miriam became jealous, she imagined that Aaron and herself had been neglected, and that Moses’ wife was the cause—that she had influenced the mind of her husband—that he did not consult them in important matters as much as formerly.

The Lord heard the words of murmuring against Moses, and he was displeased. . . . And the anger of the Lord was kindled against them, and he departed. And the cloud departed from off the tabernacle, and behold Miriam became leprous, white as snow.” . . . And Moses cried unto the Lord, saying, Heal her now, O God, I beseech thee.” “And Miriam was shut out from the camp seven days; and the people journeyed not till Miriam was brought in again.” . . .

[By] the complaints of Miriam against God’s chosen servant, she not only behaved irreverently to Moses, but toward God himself, who had chosen him. Aaron was drawn into the jealous spirit of his sister Miriam. He might have prevented the evil if he had not sympathized with her, and had presented before her the sinfulness of her conduct. But instead of this, he listened to her words of complaint. The murmurings of Miriam and Aaron are left upon record as a rebuke to all who will yield to jealousy, and complain of those upon whom God lays the burden of his work.—Spiritual Gifts, vol. 4a, pp. 20, 21.

We are to direct the weapons of our warfare against our foes, but never to turn them toward those who are under marching orders from the King of Kings, who are fighting manfully the battles of the Lord of lords. Let no one aim at a soldier whom God recognizes, whom God has sent forth to bear a special message to the world and to do a special work.

The soldiers of Christ may not always reveal perfection in their step, but their mistakes should call out from their fellow comrades not words that will weaken, but words that will strengthen, and will help them recover their lost ground. They should not turn the glory of God into dishonor, and give an advantage to the bitterest foes of their King.—Selected Messages, book 3, p. 344.

Let not fellow soldiers be severe, unreasonable judges of their comrades, and make the most of every defect. Let them not manifest satanic attributes in becoming accusers of the brethren. We shall find ourselves misrepresented and falsified by the world, while we are maintaining the truth and vindicating God’s downtrodden law; but let no one dishonor the cause of God by making public some mistake that the soldiers of Christ may make, when that mistake is seen and corrected by [the] ones who have taken some false position.

God will charge those who unwisely expose the mistakes of their brethren with sin of far greater magnitude than he will charge the one who makes a misstep. Criticism and condemnation of the brethren are counted as criticism and condemnation of Christ.—Letter 48, 1894.

\section*{Tuesday – Restlessness Leads to Rebellion}

[A]fter describing the beauty and fertility of the land, all but two of the spies enlarged upon the difficulties and dangers that lay before the Israelites should they undertake the conquest of Canaan. . . .

Hope and courage gave place to cowardly despair, as the spies uttered the sentiments of their unbelieving hearts, which were filled with discouragement prompted by Satan. Their unbelief cast a gloomy shadow over the congregation, and the mighty power of God, so often manifested in behalf of the chosen nation, was forgotten. The people did not wait to reflect; they did not reason that He who had brought them thus far would certainly give them the land; they did not call to mind how wonderfully God had delivered them from their oppressors, cutting a path through the sea and destroying the pursuing hosts of Pharaoh. They left God out of the question, and acted as though they must depend solely on the power of arms.

In their unbelief they limited the power of God and distrusted the hand that had hitherto safely guided them. And they repeated their former error of murmuring against Moses and Aaron.—Patriarchs and Prophets, pp. 387, 388.

Caleb was faithful and steadfast. He was not boastful, he made no parade of his merits and good deeds; but his influence was always on the side of right. And what was his reward? When the Lord denounced judgments against the men who refused to hearken to His voice, He said: “But My servant Caleb, because he had another spirit with him, and hath followed Me fully, him will I bring into the land whereinto he went; and his seed shall possess it.” While the cowards and murmurers perished in the wilderness, faithful Caleb had a home in the promised Canaan.—Testimonies for the Church, vol. 5, p. 303.

These men, [the ten spies], having entered upon a wrong course, stubbornly set themselves against Caleb and Joshua, against Moses, and against God. Every advance step rendered them the more determined. They were resolved to discourage all effort to gain possession of Canaan. They distorted the truth in order to sustain their baleful influence. It “is a land that eateth up the inhabitants thereof,” they said. This was not only an evil report, but it was also a lying one. It was inconsistent with itself. … But when men yield their hearts to unbelief they place themselves under the control of Satan, and none can tell to what lengths he will lead them. …

Revolt and open mutiny quickly followed; for Satan had full sway, and the people seemed bereft of reason. … [T]hey accused not only Moses, but God Himself, of deception, in promising them a land which they were not able to possess. And they went so far as to appoint a captain to lead them back to the land of their suffering and bondage, from which they had been delivered by the strong arm of Omnipotence.—Patriarchs and Prophets, p. 389.

\section*{Wednesday – An Intercessor}

Moses was the greatest man who ever stood as leader of the people of God. He was greatly honored by God, not for the experience which he had gained in the Egyptian court, but because he was the meekest of men. God talked with him face to face, as a man talks with a friend. If men desire to be honored by God, let them be humble. Those who carry forward God’s work should be distinguished from all others by their humility. Of the man who is noted for his meekness, Christ says, He can be trusted. Through him I can reveal Myself to the world. He will not weave into the web any threads of selfishness. I will manifest Myself to him as I do not to the world.—Ellen G. White Comments, in The SDA Bible Commentary, vol. 1, p. 1113.

When Moses heard the people weeping in the door of their tents, and complaining throughout their families, he was displeased. He presented before the Lord the difficulties of his situation, and the unsubmissive spirit of the Israelites, and the position in which God had placed him to the people, that of a nursing father, who should make the sufferings of the people his own. He inquired of the Lord how he could bear this great burden of continually witnessing the disobedience of Israel, and hearing their murmurings against his commands, and against God himself. He declared before the Lord that he had rather die than see Israel, by their perverseness, drawing down judgments upon themselves, while the enemies of God were rejoicing in their destruction.—Spiritual Gifts, vol. 4a, p. 16.

Sin blinds the eyes and defiles the heart. Integrity, firmness, and perseverance are qualities which all should seek earnestly to cultivate; for they clothe the possessor with a power which is irresistible, a power which makes him strong to do good, strong to resist evil, strong to bear adversity. It is here that true excellence of character shines forth with the greatest luster. …

God has given us our intellectual and moral powers, but to a great extent every person is the architect of his own character. Every day the structure is going up. The word of God warns us to take heed how we build, to see that our building is founded upon the eternal Rock. The time is coming when our work will stand revealed just as it is. Now is the time for all to cultivate the powers which God has given them, that they may form characters for usefulness here and for a higher life hereafter.

Every act of life, however unimportant, has its influence in forming the character. A good character is more precious than worldly possessions, and the work of forming it is the noblest in which men can engage.—Testimonies for the Church, vol. 4, pp. 655–657.

\section*{Thursday – Faith Versus Presumption}

We must rise to a higher standard on the subject of faith. We have too little faith. The Word of God is our endorsement. We must take it, simply believing every word. With this assurance, we may claim large things, and according to our faith it will be unto us. If we humble our hearts before God, if we seek to abide in Christ, we shall have a higher, holier experience.

True faith consists in doing just what God has enjoined, not manufacturing things He has not enjoined. Justice, truth, mercy, are the fruit of faith. We need to walk in the light of God’s law; then good works will be the fruit of our faith, the proceeds of a heart renewed every day. …

If we had exercised more faith in God and had trusted less to our own ideas and wisdom, God would have manifested His power in a marked manner on human hearts. By a union with Him, by living faith, we are privileged to enjoy the virtue and efficacy of His mediation. Hence we are crucified with Christ, dead with Christ, risen with Christ, to walk in newness of life with Him.—The Upward Look, p. 346.

As Satan seeks to break down the barriers of the soul, by tempting us to indulge in sin, we must by living faith retain our connection with God, and have confidence in His strength to enable us to overcome every besetment. We are to flee from evil, and seek righteousness, meekness, and holiness.

It is time for every one of us to decide whose side we are on. The agencies of Satan will work with every mind that will allow itself to be worked by him. But there are also heavenly agencies waiting to communicate the bright rays of the glory of God to all who are willing to receive Him.

It is ours to choose whether we will be numbered with the servants of Christ or the servants of Satan. Every day we show by our conduct whose service we have chosen.—Our High Calling, p. 15.

The prevailing spirit of our time is one of infidelity and apostasy—a spirit of avowed illumination because of a knowledge of truth, but in reality of the blindest presumption. Human theories are exalted and placed where God and His law should be. Satan tempts men and women to disobey, with the promise that in disobedience they will find liberty and freedom that will make them as gods. There is seen a spirit of opposition to the plain word of God, of idolatrous exaltation of human wisdom above divine revelation. Men have allowed their minds to become so darkened and confused by conformity to worldly customs and influences that they seem to have lost all power to discriminate between light and darkness, truth and error. So far have they departed from the right way that they hold the opinions of a few philosophers, so-called, to be more trustworthy than the truths of the Bible. The entreaties and promises of God’s word, its threatenings against disobedience and idolatry—these seem powerless to melt their hearts. A faith such as actuated Paul, Peter, and John they regard as old-fashioned, mystical, and unworthy of the intelligence of modern thinkers.—Prophets and Kings, p. 178.

\section*{Friday – Further Thought}

\setlength{\parindent}{0pt}Conflict and Courage, “Two Ways to Go,” p. 25;

Patriarchs and Prophets, “From Sinai to Kadesh,” pp. 377–386.

\end{multicols}

\end{document}

